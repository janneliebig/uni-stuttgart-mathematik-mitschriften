\section{Endlich erzeugte Gruppen}

\begin{definition}
	Sei $G$ eine Gruppe und $S \subseteq G$ eine Teilmenge. Definiere 
	\[\langle S \rangle := \bigcap_{H \leq G, S \subseteq H} H \leq G.\]
	$\langle S \rangle$ heißt \textbf{die von $S$ erzeugte Untergruppe} von $G$.
\end{definition}
Falls $G = \langle S \rangle$, so heißt $S$ \textbf{Erzeugendensystem} von $G$. Hat $G$ ein endliches Erzeugendensystem, so heißt $G$ \textbf{endlich erzeugt}. Gibt es ein $g \in G$ mit $G = \langle \{g\} \rangle =: \langle g \rangle$, so heißt $G$ \textbf{zyklisch}.
\begin{rem}
	\begin{enumerate}[label=(\roman*)]
		\item Nach Konstruktion ist $\langle S\rangle$ die kleinste Untergruppe von $G$, die $S$ enthält.
		\item Für $S \neq \emptyset$ ist $\spn{S} = \{s_1 \cdots s_t \mid t \in \N, s_i \in S \cup S^{-1}\}$. 
		
		Insbesondere ist für $g \in G$:
		\[\spn{g} = \{g^n \mid n \in \Z\} \quad\text{mit}\quad g^n := \begin{cases}
			1_G, & n = 0\\
			\underbrace{g \cdots g}_{n-\text{mal}}, & n > 0\\
			\underbrace{g^{-1} \cdots g^{-1}}_{(-n)-\text{mal}}, & n < 0 
		\end{cases}\]
\end{enumerate}
\begin{leftbar}
	Wir wollen zunächst zyklische Gruppen besser verstehen!
\end{leftbar}
\end{rem}
\begin{beispiel}\label{beispiel2_3}
	\begin{enumerate}[label=(\arabic*)]
		\item $(\Z, +) = \spn{1} = \spn{-1}$ ist eine zyklische Gruppe.
		
		$(\Z_m, +) = \spn{\bar{1}}$ ist eine zyklische Gruppe der Ordnung $m$.
		
		\item Sei $G$ eine Gruppe mit $|G| = p$ Primzahl. Dann ist $G$ zyklisch.
		\begin{proof}
			Sei $1_G \neq g \in G$ und betrachte $\spn{g} \leq G$. Nach Satz \ref{satz1_11} $|\spn{g}|$ teilt $|G| = p$. Da $\spn{g} > 1$ nach Voraussetzung folgt $|\spn{g}| = p$. Somit $G = \spn{g}$.
		\end{proof}
	\end{enumerate}
\end{beispiel}
\begin{satz}\label{satz2_4}
	\begin{enumerate}[label=(\alph*)]
		\item Eine Gruppe $G$ ist zyklisch genau dann, wenn es einen surjektiven Gruppenhomomorphismus von der Form $\Z \to G$ gibt.
		\item Für eine zyklische Gruppe $G$ gilt:
		\[G \cong \begin{cases}
			\Z, & |G| = \infty,\\
			\Z_m, & |G| = m.
		\end{cases}\]
		Zudem ist jede Untergruppe von $G$ wieder zyklisch.
	\end{enumerate}
\end{satz}
\begin{proof}
	\begin{enumerate}[label=(\alph*)]
		\item \glqq{}$\Rightarrow$\grqq: Sei $G = \spn{g} = \{g^n \mid n \in \Z\}$. Definiere einen Gruppenhomomorphismus $\Z \to G$ durch $m \mapsto g^n$. Dieser ist nach Voraussetzung surjektiv.
		
		\glqq{}$\Leftarrow$\grqq: Sei $\varphi \colon \Z \to G$ ein surjektiver Gruppenhomomorphismus. Definiere $g := \varphi(1) \in G$.
		
		\underline{Behauptung}: $G = \spn{g}$.
		
		Die Inklusion $\spn{g} \subseteq G$ ist klar. Sei nun $h \in G$ beliebig. Da $\varphi$ surjektiv ist, existiert $n \in \Z$ mit $\varphi(n) = h$. Da $\varphi$ Gruppenhomomorphismus, gilt
		\[h = \varphi(n) = \begin{cases}
			\underbrace{\varphi(1) \cdots \varphi(1)}_{n-\text{mal}}, & n \geq 0\\
			\underbrace{\varphi(1)^{-1} \cdots \varphi(1)^{-1}}_{n-\text{mal}}, &n < 0
		\end{cases}\]
		Daraus folgt $h = g^n \in \spn{g}$.
		
		\item Sei $G$ zyklisch und $\varphi \colon \Z \to G$ ein surjektiver Gruppenhomomorphismus, der nach (a) existiert. Nach Satz \ref{satz1_20} gilt
		\[G \cong \faktor{\Z}{\ker(\varphi)}.\]
		Nach Aufgabe M.1.4. wissen wir, dass $\ker(\varphi) = m\Z$ für ein $m \in \N_0$. Damit folgt der erste Teil der Behauptung.
		
		Sei nun $H \leq G$. Dann ist $\varphi^{-1}(H)$ eine Untergruppe von $\Z$ (siehe Beweis zu Satz \ref{satz1_23}) und somit erneut $\varphi^{-1}(H) = m\Z, \; m \in\N_0$. Insbesondere ist $\varphi^{-1}(H) = \spn{m} \leq \Z$. Da $\varphi$ surjektiv ist, gilt $\varphi(\varphi^{-1}(H)) = H$ und $H$ wird von $\varphi(n)$ erzeugt.
	\end{enumerate}
\end{proof}
\begin{definition}
	Sei $G$ eine Gruppe und $g \in G$. Die \textbf{Ordnung von $g$} ist definiert als die Ordnung $\spn{g}$, der von $g$ erzeugten zyklischen Untergruppe von $G$.
	Wir schreiben $\ord(g)$ für die Ordnung von $g$. 
\end{definition}
\begin{rem}
	Ist $\ord(g) = m \in \N$ bzw. $\spn{g} \cong \Z_m$ mit $\spn{g} = \{1_G, g, \dots, g^{m-1}\}$ nach Satz \ref{satz2_4} (b), so gilt $g^n = 1_G$ genau dann, wenn $n \in m\Z$. 
	
	Ist $G$ endlich, so gilt $\ord(g)$ teilt $|G|$ nach Satz \ref{satz1_11} und somit $g^{|G|} = 1_G$ (\textbf{Kleiner Fermat'scher Satz}).
	
	Ist $\ord(g) = \infty$ bzw. $\spn{g} \cong \Z$, so sind die $g^n$ mit $n \in \Z$ paarweise verschieden.
\end{rem}
\begin{beispiel}
	\begin{enumerate}[label=(\arabic*)]
		\item Für $\bar{a} \in \Z_m$ mit $m \in \N$ gilt $\ord(\bar{a}) = \frac{m}{\ggt(a,m)}$. Zum Beispiel hat $\bar{8} \in \Z_{12}$ die Ordnung $\frac{12}{\ggt(8,12)} = \frac{12}{4} = 3$.
		
		\item Für $n \geq 3$ sei $D_n$ die Symmetriegruppe eines regelmäßigen $n$-Ecks in $\R^2$. Diese heißt auch \textbf{Diedergruppe}. Für $n = 3$ gilt $D_3 \cong S_3$.
		
		Im Allgemeinen enthält $D_n$ genau $n$ Drehungen und $n$ Spiegelungen, so dass $|D_n| = 2n$. Sei $r$ eine Drehung um $\frac{2\pi}{n}$ und $s$ eine beliebige Spiegelung. Dann gilt $\ord(r) = n$ und $\ord(s) = 2$, sowie $D_n = \{\id, r, r^2, \dots, r^{n-1}, s, sr, sr^2, \dots, sr^{n-1}\} = \spn{\{r,s\}}$.
	\end{enumerate}
\end{beispiel}

\begin{center}
	\textit{Welche Gruppen können wir aus zyklischen Gruppen zusammensetzen?}
\end{center}

\begin{definition}
	Eine Gruppe $G$ heißt \textbf{inneres direktes Produkt} von $G_1$ und $G_2$, falls
	\begin{itemize}
		\item $G_1, G_2 \unlhd G$.
		\item $G_1 \cdot G_2 = G$.
		\item $G_1 \cap G_2 = \{1_G\}$.
	\end{itemize}
\end{definition}

\begin{rem}
	\begin{enumerate}[label=(\roman*)]\label{rem2_9}
		\item Ist $G = G_1 \times G_2$ (äußeres) direktes Produkt der Gruppen $G_1$ und $G_2$, so ist $G$ inneres direktes Produkt von $G_1 \times \{1_G\}$ und $\{1_G\} \times G_2$.
		
		\item Ist $G$ inneres direktes Produkt von $G_1$ und $G_2$, so gilt
		\[G \cong G_1 \times G_2.\]
		\begin{proof}
			Betrachte die Abbildung $\varphi \colon G_1 \times G_2$ mit $(g_1, g_2) \mapsto g_1g_2$. Da $\underbrace{g_1g_2g_1^{-1}}_{\in G_2}  g_2^{-1} = g_1(\underbrace{g_2g_1^{-1}g_2^{-1}}_{\in G_1}) \in G_1 \cap G_2 = \{1_G\}$, folgt $g_1 = g_2$. Somit gilt $\varphi((g_1, g_2) (h_1, h_2)) = \varphi(g_1h_1, g_2h_2) = g_1(h_1g_2)h_2 = (g_1g_2)(h_1h_2) = \varphi((g_1, g_2)) \varphi((h_1, h_2))$. $\varphi$ ist zudem bijektiv nach Voraussetzung.
		\end{proof}
	\end{enumerate}
\end{rem}
\begin{beispiel}
	\begin{enumerate}[label=(\arabic*)]
		\item Die abelsche Gruppe $\C \setminus \{0\}$ ist inneres direktes Produkt von $\R_{>0}$ und $S^1 := \{z \in \C \mid |z| = 1\}$.
		\item Nach Aufgabe S.2.2. gilt 
		\[V_4 = \{\id, (12)(34), (13)(24), (14)(23)\} \unlhd S_4\]
		mit $S_4/V_4 \cong S_3$. Die $S_3$ ist isomorph zu einer Untergruppe $H$ von $S_4$ (Beispiel \ref{beispiel1_6} (4)), so dass $V_4 \cdot H = S_4$ und $V_4 \cap H = \{\id\}$. Aber $S_4$ ist nicht inneres direktes Produkt von $V_4$ und $H$, da $H$ kein Normalteiler von $S_4$.
	\end{enumerate}
\end{beispiel}
\begin{thm}\label{thm2_11}
	Jede endlich erzeugte abelsche Gruppe ist ein endliches inneres direktes Produkt zyklischer Gruppen.
\end{thm}
\begin{proof}
	Sei $(G, +)$ abelsche Gruppe, erzeugt von $S = \{a_1, \dots, a_k\} \subseteq G$. 
	
	\textbf{Induktion über $k$:} Für $k = 1$ ist $G$ zyklisch.
	
	Betrachte den surjektiven Gruppenhomomorphismus
	\[\varphi_S \colon \Z^k \to G \quad\text{mit}\quad(n_1, \dots, n_k) \mapsto n_1a_1 + \dots + n_ka_k.\]
	Sei $\pi \colon \Z^k \to \Z$ die Projektion auf die erste Komponente, also
	\[\pi((n_1, \dots, n_k)) = n_1.\]
	Das Bild von $\ker(\varphi_S)$ unter $\pi$ ist Untergruppe von $\Z$ und somit von der Form $d\Z$ für $d \in \N_0$. Sei o.B.d.A. $S$ so gewählt, dass $d$ minimal ist. Falls $d = 0$, so ist $\spn{a_1} \cap \spn{S\setminus\{a_1\}}= \{0_G\}$ und $\ord{a_1} = \infty$, d.h. $G$ ist inneres direktes Produkt von $\spn{a_1}$ und $\spn{S\setminus\{a_1\}}$, wobei $\spn{a_1} \cong \Z$.
	
	Auf $S\setminus\{a_1\}$ können wir die Induktionsvoraussetzung anwenden. 
	
	Falls $d > 0$, wähle $(n_1, \dots, n_k) \in \ker(\varphi_S)$ mit $n_1 = d$. Sei $2 \leq i \leq k$. Division mit Rest liefert $q_i, d_i \in \Z$ mit
	\[n_i = q_i d + d_i \quad\text{und}\quad 0 \leq d_i < d.\]
	Definiere $S_i := \{b_1, \dots, b_k\}$ durch $b_1 = a_i$ und $b_i = a_1 + q_i a_i$ und $b_j = a_j$ sonst. Dann ist auch $S_i$ Erzeugendensystem von $G$. Zudem liegt der Vektor
	\[(d_i, n_2, \dots, i, \dots, n_k) \in \Z^k\]
	im Kern von $\varphi_{S_i}$, da
	\[d_ib_1 + db_i = (n_i - q_id)a_i + d(a_1 + q_ia_i) = n_ia_i + n_1a_1\]
	und weil $(n_1, \dots, n_k) \in \ker(\varphi_S)$. Wegen der Minimalität von $d$ ist $1 \leq d_i < d$ ausgeschlossen, d.h.
	\[d_i = 0 \quad\text{und}\quad d \mid n_i.\]
	Setze nun $x_i = \frac{n_i}{d}$ für $1 \leq i \leq k$. Insbesondere $x_1 = 1$. Dann wird $G$ von der Menge
	\[\bigg\{\underbrace{\sum_{i=1}^k x_ia_i}_{=: a} , a_2, \dots, a_k\bigg\}\]
	mit $\spn{a} \cap \spn{\{a_2, \dots, a_k\}} = \{0_G\}$. Denn ein Element im Schnitt hat die Form 
	\[m_1a = \sum_{i=2}^k m_ia_i\]
	für $(m_1, \dots, m_k) \in \Z^k$, so dass $m_1$ im Bild von $\ker(\varphi_S)$ unter $\pi$ liegt, also ein Vielfaches von $d$ ist. Es gilt aber bereits
	\[da = n_1 a_1 + \dots + n_ka_k = 0_G.\]
	Somit ist $G$ inneres direktes Produkt von $\spn{a}$ und $\spn{S \setminus \{a_1\}}$, wobei  $\spn{a} \cong \Z_d$. Nun können wir erneut die Induktionsvoraussetzung anwenden.
\end{proof}
\begin{kor}[Hauptsatz für endlich erzeugte abelsche Gruppen]\label{kor2_12}
	Sei $G$ eine endlich erzeugte abelsche Gruppe. Dann existieren eindeutige $r,t \in \N_0$ sowie bis auf Reihenfolge eindeutig bestimmte Primzahlpotenzen 
	\[1 < p_1^{k_1} \leq \dots \leq p_t^{k_t} \quad\text{mit}\quad G \cong \Z^r \times \Z_{p_1^{k_1}} \times \dots \times \Z_{p_t^{k_t}}.\]
\end{kor}
\begin{proof}
	Die Existenz folgt aus Theorem \ref{thm2_11} zusammen mit Bemerkung \ref{rem2_9} (ii) und Aufgabe M.3.3. Letztere besagt, dass für $m, n \in \N$ mit $\mathrm{ggT}(m,n) = 1$ gilt $\Z_{nm} \cong \Z_n \times \Z_m$. Die Eindeutigkeit können/wollen wir hier nicht beweisen.
\end{proof}
Was können wir im nicht-abelschen Fall tun? Eine Klassifikation aller endlich erzeugten oder endlichen Gruppen ist hoffnungslos. Im Jahr 1982 hat man die Klassifikation aller endlichen einfachen Gruppen abgeschlossen, also aller endlichen Gruppen mit genau zwei (trivialen) Normalteilern. Dies war ein Mammutprojekt!
\begin{itemize}
	\item mehr als 500 Fachartikel,
	\item mehr als 100 MathematikerInnen,
	\item Zeitraum von über 50 Jahren,
	\item Einsatz von Computern.
\end{itemize}
Die endlichen einfachen Gruppen sind von der Form
\begin{enumerate}[label=(\arabic*)]
	\item Zyklische Gruppe $\Z_p$ mit $p$ Primzahl.
	\item Alternierende Gruppen $A_n$ für $n \geq 5$.
	\item Endliche Gruppen vom Lie-Typ, z.B.
	\[\mathsf{PSL}_{(K)} := \faktor{\SL_n(K)}{Z(\SL_n(K))} = \faktor{\SL_n(K)}{\{\lambda \mat{E}_n \mid \lambda^n = 1\}}\]
	für $n > 2$ und einen endlichen Körper $K$.
	\item 26 sporadische Gruppen mit bis zu ungefähr $8 \cdot 10^{53}$ Elementen, die sogenannten Monster!
\end{enumerate}
\begin{beispiel}\label{beispiel2_13}
	Die alternierende Gruppe $A_4$ ist nicht einfach nach Aufgabe S.2.2. Aber für $n \geq 5$ ist $A_n$ einfach.
\end{beispiel}
\begin{proof}
	Sei $\{\id\} \neq N \unlhd A_n$. Wir zeigen, dass $N = A_n$. 
	
	\textbf{Schritt 1: } $N$ enthält einen Zykel der Länge 3. 
	
	Sei $\id \neq \sigma \in N$. Ist $\sigma$ kein Zykel der Länge 3, so gilt einer der folgenden Fälle:
	\begin{enumerate}[label=(\roman*)]
		\item $\sigma = (a_1 a_2 a_3 a_4 \dots)\dots$
		\item $\sigma = (a_1 a_2 a_3)(a_4 a_5 a_6)\dots$
		\item $\sigma = (a_1 a_2)(a_3 a_4)(a_5 a_6)\dots$
		\item $\sigma = (a_1 a_2)(a_3 a_4)$
	\end{enumerate}
	Da $N$ Normalteiler ist, gilt $(\pi \sigma \pi^{-1})\sigma^{-1} \in N$ für alle $\pi \in A_n$. Im Fall (i) wähle $\pi = (a_2 a_1 a_3)$.
	\[(\pi \sigma \pi^{-1})\sigma^{-1} = (a_2 a_1 a_3)\sigma(a_1 a_2 a_3)\sigma^{-1} = (a_1 a_3 a_4).\]
	Im Fall (ii) wähle $\pi = (a_3 a_2 a_4)$.
	\[(\pi\sigma\pi^{-1})\sigma^{-1} = (a_3 a_2 a_4)\sigma(a_2a_3a_4)\sigma^{-1} = (a_1 a_5 a_2 a_4 a_3)\]
	und weiter im Fall (i). Im Fall (iii) wähle $\pi = (a_2 a_1 a_3)$:
	\[(\pi\sigma\pi^{-1}) \sigma^{-1} = (a_2 a_1 a_3)\sigma(a_1 a_2 a_3) \sigma^{-1} = (a_1 a_4)(a_2 a_3).\]
	Im Fall (iv) wähle $\pi = (a_2 a_1 a_5)$:
	\[(\pi\sigma\pi^{-1})\sigma^{-1} = (a_2 a_1 a_5)\sigma(a_1 a_2 a_5)\sigma^{-1} = (a_1 a_2 a_5).\]
	Also enthält $N$ einen Zykel der Länge 3.
	
	\textbf{Schritt 2: } $N = A_n$.
	
	Sei $(a_1 a_2 a_3) \in N$. Da $N \unlhd A_n$ ist, gilt:
	\[(a_3 a_4 a_5)(a_1 a_2 a_3)(a_4 a_3 a_5) = (a_1 a_2 a_4) \in N.\]
	Insbesondere sind alle Zykel der Form $(a_1 a_2 x)$ in $N$ mit $x \in \{1,\dots, n\} \setminus \{a_1, a_2\}$. Wiederholen des Arguments zeigt, dass alle Zykel der Länge 3 in $N$ enthalten sind. Da $A_n$ nach Aufgabe M.3.2. von diesen Zykeln erzeugt wird, folgt $N = A_n$.
\end{proof}
\begin{leftbar}
	{Warum sind endliche einfache Gruppen so wichtig?}
\end{leftbar}

\begin{definition}
	Sei $G$ ein Gruppe. Eine Folge von Untergruppen
	\[\{1_G\} = G_0 \unlhd G_1 \unlhd G_2 \unlhd \dots \unlhd G_n = G\]
	heißt \textbf{Normalreihe der Länge $n$}. Die Faktoren $G_i/G_{i-1}$ heißen \textbf{Faktoren} der Normalreihe. Eine Normalreihe von $G$ heißt \textbf{Kompositionsreihe}, falls alle ihre Faktoren einfach sind. Die Faktoren einer Kompositionsreihe heißen \textbf{Kompositionsfaktoren}.
\end{definition}
\begin{rem}\label{rem2_15}
	\begin{enumerate}[label=(\roman*)]
		\item Jede Gruppe $G$ hat die Normalreihe $\{1_G\} \unlhd G$.
		\item Jede endliche Gruppe $G$ hat eine Kompositionsreihe.
		
		Induktion nach $|G|$. Ist $G$ einfach, dann ist $\{1_G\} \unlhd G$ Kompositionsreihe. Ist andernfalls $N \unlhd G$ maximaler echter Normalteiler. nach Satz \ref{satz1_23} ist $G/N$ einfach und $N$ hat nach Induktionsvoraussetzung eine Kompositionsreihe
		\[\{1_G\} = N_0 \unlhd N_1 \unlhd \dots \unlhd N_t = N.\]
		Dann ist $\{1_G\} = N_0 \unlhd N_1 \unlhd \dots \unlhd N_t = N \unlhd G$ eine Kompositionsreihe von $G$.
		
		\item Die Gruppe $\Z$ hat keine Kompositionsreihe (nutze Klassifikation der Untergruppen von $\Z$).
	\end{enumerate}
\end{rem}
\begin{thm}\label{Satz von Jordan-Hölder}
	Sei $G$ eine endliche Gruppe. Dann alle Kompositionsreihen von $G$ \textbf{äquivalent}, das heißt sie haben die selbe Länge und bis auf Isomorphie und Reihenfolge die selben Kompositionsfaktoren.
\end{thm}
\begin{proof}
	Betrachte die folgenden Kompositionsreihen:
	\begin{enumerate}[label=(\Roman*)]
		\item $\{1_G\} = G_0 \unlhd G_1 \unlhd \dots \unlhd G_r = G$,
		\item $\{1_G\} = H_0 \unlhd H_1 \unlhd \dots \unlhd H_s = G$.
	\end{enumerate}
	\textbf{Induktion nach $r$: } Für $r = 1$ ist $G$ einfach. Also auch $G = H_1$.
	
	Sei $r > 1$. 
	
	\textbf{Fall 1: } $G_{r-1} = H_{s-1}$. Dann hat $G_{r-1}$ die Kompositionsreihen $\{1_G\} = G_0 \unlhd \dots \unlhd G_{r-1}$ und $\{1_G\} = H_0 \unlhd \dots \unlhd H_{s-1} = G_{r-1}$. Nach Induktionsvoraussetzung sind diese beiden und somit auch die ursprünglichen beiden Kompositionsreihen \textbf{(I)} und \textbf{(II)} von $G$ äquivalent.
	
	\textbf{Fall 2: } $G_{r-1} \neq H_{s-1}$. Betrachte $G_{r-1} \unlhd G_{r-1}\dots H_{s-1} \unlhd G$. Da $G_{r-1} \neq H_{s-1}$ und $G_r / G_{r-1}$ und $G/H_{s-1}$ einfach, folgt mit Satz \ref{satz1_23}, dass $G_{r-1}H_{s-1} = G$. Sei $J = G_{r-1} \cap H_{s-1}$ mit $J \unlhd G$. Nach Satz \ref{satz1_21} (a) gilt
	\[\faktor{G}{G_{r-1}} = \faktor{G_{r-1}H_{s-1}}{G_{r-1}} \cong \faktor{H_{s-1}}{J}\]
	sowie 
	\[\faktor{G}{H_{s-1}} = \faktor{H_{s-1} G_{r-1}}{H_{s-1}} \cong \faktor{G_{r-1}}{J}\]
	d.h. $H_{s-1} / J$ und $G_{r-1}/J$ sind einfach. Nach Bemerkung \ref{rem2_15} (ii) hat $J$ eine Kompositionsreihe
	\[\{1_G\} = J_0 \unlhd J_1 \unlhd \dots \unlhd J_t = J.\]
	Diese induziert die folgenden beiden Kompositionsreihen von $G$:
	\begin{enumerate}[start=3,label=(\Roman*)]
		\item $\{1_G\} = J_0 \unlhd \dots \unlhd J_t = J \unlhd G_{r-1} \unlhd G$,
		\item $\{1_G\} = J_0 \unlhd \dots \unlhd J_t = J \unlhd H_{s-1} \unlhd G$.
	\end{enumerate}
	Diese sind äquivalent, da bis auf Isomorphie nur die letzten beiden Faktoren vertauscht werden. Nach Induktionsvoraussetzung sind auch die Kompositionsreihen \textbf{(I)} und \textbf{(III)} von $G$ äquivalent. Insbesondere gilt $r-1 = t+1$. Somit liefert \textbf{(IV)} eine Kompositionsreihe von $H_{s-1}$ der Länge $r-1$, die nach Induktionsvoraussetzung äquivalent ist zu $\{1_G\} = H_0 \unlhd \dots \unlhd H_{s-1}$. Folglich sind auch die Kompositionsreihen \textbf{(II)} und \textbf{(IV)} äquivalent. Dies liefert die gewünschte Äquivalenz von \textbf{(I)} und \textbf{(II)}.
\end{proof}
\begin{beispiel}
	$\Z_6$ hat die Kompositionsreihen $\{\bar{0}\} \unlhd \spn{\bar{2}} \unlhd \Z_6$ und $\{\bar{0}\} \unlhd \spn{\bar{3}} \leq \Z_6$ mit Kompositionsfaktoren isomorph zu $\Z_2$ und $\Z_3$.
	
	Die symmetrische Gruppe $S_3$ hat die Kompositionsreihe $\{\id\} \unlhd A_3 \unlhd S_3$, deren Kompositionsfaktoren auch isomorph zu $\Z_2$ und $\Z_3$ sind. Aber $\Z_6 \cong \Z_2 \times \Z_3 \not\cong S_3$!
\end{beispiel}
% Vorlesung vom 11.11.2024:
Wir haben gesehen, dass sich endliche abelsche Gruppen, auf im Wesentlichen, eindeutige Art und Weise, aus einfachen Gruppen zusammenkleben lassen. Aber welche Gruppen können wir aus einfachen zyklischen Gruppen zusammenkleben?
\begin{definition}
	Eine Gruppe $G$ heißt \textbf{auflösbar}, wenn $G$ eine Normalreihe mit abelschen Faktoren hat, d.h. es gibt eine Folge von Untergruppen
	\[\{1_G\} = G_0 \unlhd G_1 \unlhd \dots \unlhd G_n = G\]
	mit $G_i/G_{i-1}$ abelsch für alle $1 \leq i \leq n$.
\end{definition}
\begin{rem}
	Auflösbare Gruppen werden uns in der Algebra II helfen zu entscheiden, ob Gleichungen durch endliche Wurzelausdrücke aufgelöst werden können (siehe Kapitel 0).
\end{rem}
\begin{beispiel}\label{beispiel2_20}
	\begin{enumerate}[label=(\arabic*)]
		\item Abelsche Gruppen sind auflösbar. Nach Theorem \ref{thm2_11} bzw. \ref{kor2_12} wissen wir, dass wir endliche abelsche Gruppen aus einfachen zyklischen Gruppen zusammenkleben können.
		\item Jede einfache auflösbare Gruppe $G$ ist isomorph zu $\Z_p$, $p$ prim.
		\begin{proof}
			Da $G$ einfach, existiert nur die triviale Normalreihe $\{1_G\} \unlhd G$. Da $G$ auflösbar ist, folgt, dass $G$ abelsch ist und abelsche Gruppen ohne echten Normalteiler sind isomorph zu $\Z_p$.
		\end{proof}
		\item Die alternierende Gruppe $A_n$ mit $n \in \N$ ist auflösbar genau dann, wenn $n \leq 4$.
		\begin{proof}
			Für $n \leq 3$ ist $A_n$ abelsch und dadurch auflösbar. Für $n = 4$ betrachte die Normalreihe $\{1_G\} \unlhd V_4 \unlhd A_4$ mit Faktoren $V_4 \cong \Z_2 \times \Z_2$ und $A_4 / V_4 \cong \Z_3$ (siehe Aufgabe S.2.2.). Für $n \geq 5$ nutze Beispiel \ref{beispiel2_13}.
		\end{proof}
		\item Man kann zeigen, dass endliche Gruppen ungerader Ordnung auflösbar stets auflösbar sind. (\textbf{Satz von Feit-Thompson} (1963))
		\item Die kleinste nicht auflösbare Gruppe ist die $A_5$.
	\end{enumerate}
\end{beispiel}
\begin{prop}\label{prop2_21}
	Sei $G$ eine auflösbare Gruppe.
	\begin{enumerate}[label=(\alph*)]
		\item Jede Untergruppe von $H \leq G$ ist auflösbar.
		\item Ist $N \unlhd G$ Normalteiler, so ist auch $G/N$ auflösbar. 
	\end{enumerate}
\end{prop}
\begin{proof}
	Sei $\{1_G\} = G_0 \unlhd G_1 \unlhd \dots \unlhd G_n = G$ eine Normalreihe von $G$ mit der Eigenschaft $G_i/G_{i-1}$ abelsch für alle $i \in \{1, \dots, n\}$.
	\begin{enumerate}[label=(\alph*)]
		\item Betrachte die induzierte Normalreihe der Form
		\[\{1_G\} = G_0 \cap H \unlhd G_1 \cap H \unlhd \dots \unlhd G_n \cap H = H.\]
		Nach Satz \ref{satz1_21} (a) gilt für alle $i \in \{1, \dots, n\}$
		\[\faktor{G_i \cap H}{G_{i-1} \cap H} = \faktor{G_i \cap H}{G_{i-1} \cap G_i \cap H} \cong \faktor{G_{i-1} \cdot (G_i \cap H)}{G_{i-1}} \leq \faktor{G_i}{G_{i-1}}\]
		Da $G_i / G_{i-1}$ abelsch ist, so ist auch $(G_i \cap H) / (G_{i-1}\cap H)$ abelsch. Somit ist $H$ auflösbar.
		
		\item Betrachte die durch die kanonische Projektion $\pi \colon G \to G/N$ induzierte Normalreihe der Form $\{1_{G/N}\} = \pi(G_0) \unlhd \pi(G_1) \unlhd \dots \unlhd \pi(G_n) = G/N$. Für $i \in \{1, \dots, n\}$ erhalten wir einen surjektiven Gruppenhomomorphismus 
		\[\varphi \colon G_i \to \faktor{\pi(G_i)}{\pi(G_{i-1})}\]
		durch $g_i \mapsto \pi(g_i)\pi(G_{i-1})$ mit 
		\[G_{i-1} \unlhd \ker(\varphi) \unlhd G_i.\]
		Nach Satz \ref{satz1_20} und Satz \ref{satz1_21} (b) gilt
		\[\faktor{\pi(G_i)}{\pi(G_{i-1})} \cong \faktor{G_i}{\ker(\varphi)} \cong \faktor{G_i / G_{i-1}}{\ker(\varphi) / G_{i-1}}.\]
		Da $G_i / G_{i-1}$ abelsch ist, so sind auch deren Quotienten abelsch und somit insbesondere $\pi(G_i) / \pi(G_{i-1})$. Also ist $G/N$ auflösbar.
	\end{enumerate}
\end{proof}
\begin{satz}\label{satz2_22}
	Sei $G$ eine Gruppe mit Kompositionsreihe. Dann ist $G$ auflösbar genau dann, wenn alle Kompositionsfaktoren von $G$ isomorph zu $\Z_p$ für $p$ Primzahl.
\end{satz}
\begin{proof}
	\glqq{}$\Leftarrow$\grqq: Folgt unmittelbar, da $\Z_p$ abelsch ist.
	
	\glqq{}$\Rightarrow$\grqq: Sei $G_i / G_{i-1}$ ein Kompositionsfaktor von $G$ mit $G_{i-1} \unlhd G_i \leq G$. Da $G$ auflösbar ist, sind nach Proposition \ref{prop2_21} sowohl $G_i$ als auch $G_i / G_{i-1}$ auflösbar. Da $G_i / G_{i-1}$ zudem einfach ist, folgt die Behauptung mit Beispiel \ref{beispiel2_20} (2).
\end{proof}





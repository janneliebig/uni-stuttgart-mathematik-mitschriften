\section{Körpererweiterung}
Ein Körper ist ein kommutativer von Null verschiedener Ring, in dem jedes Element ungleich Null eine Einheit ist. Wir kennen bereits $\Q, \R, \C, \Z_p$ für $p$ Primzahl sowie den Quotientenkörper $\quot(R)$ und den Quotientenring $R/I$ für einen Integritätsbereich $R$ und $I \unlhd R$ maximal (siehe Proposition \ref{prop6_3}).

Ringhomomorphismen zwischen Körpern heißen auch \textbf{Körperhomomorphismen}. Bijektive Körperhomomorphismen heißen \textbf{Körperisomorphismen}.

Da Körper nur triviale Ideale besitzen, sind Körperhomomorphismus stets injektiv.
\begin{definition}
	Seien $K$ und $L$ Körper. 
	\begin{enumerate}[label=(\alph*)]
		\item Ist $K \leq L$ ein Unterring, so heißt $K$ \textbf{Teilkörper} von $L$ und $L$ \textbf{Erweiterungskörper} von $K$. Wir sprechen von der \textbf{Körpererweiterung} $K \leq L$ und schreiben $L / K$.
		\item Ist $L/K$ eine Körpererweiterung, so ist $L$ ein $K$-Vektorraum und $[L:K] := \dim_K(L) \in \N \cup \{\infty\}$ heißt \textbf{Grad der Körpererweiterung} (vgl. Aufgabe M.13.1).  Die Körpererweiterung $L/K$ heißt \textbf{endlich}, wenn $[L:K] < \infty$.
	\end{enumerate}
\end{definition}
\begin{beispiel}\label{beispiel7_2}
	Es gilt $[\C : \R] = 2$, da $C$ als $\R$-Vektorraum die Basis $\{1,i\}$ hat. Es ist $[\R : \Q] = \infty$. Im Allgemeinen gilt $[L : K] = 1$ nur dann, wenn $L = K$. 
\end{beispiel}
\begin{definition}
	Sei $K$ ein Körper. 
	\begin{enumerate}[label=(\alph*)]
		\item Dann heißt der Schnitt
		\[\Pi(K) := \bigcap_{K' \leq K} K'\]
		aller Teilkörper der \textbf{Primkörper} von $K$.
		\item Die \textbf{Charakteristik} von $K$ ist definiert als
		\[\chr(K) := \begin{cases}
			0 & \forall n \in \N : n \cdot 1_K := 1_K + \dots + 1_K \neq 0_K,\\
			\min\{n \in \N \mid n \cdot 1_K = 0_K\} & \text{sonst.}
		\end{cases}\]
	\end{enumerate}
\end{definition}
\begin{rem}\label{rem7_4}
	\begin{enumerate}[label=(\roman*)]
		\item $\Pi(K)$ ist nach Konstruktion der kleinste Teilkörper von $K$. 
		\item\label{rem7_4_2} Betrachte den Ringhomomorphismus $\varphi_K \colon \Z \to K$ mit
		\[n \mapsto \begin{cases}
			n \cdot 1_K & n \geq 1,\\
			0_K & n = 0,\\
			-n \cdot (-1_K) & n \leq -1.
		\end{cases}.\]
		Dann gilt $\ker(\varphi_K) = \chr(K) \cdot \Z \leq \Z$ (vgl. Aufgabe S.7.2).
	\end{enumerate}
\end{rem}
\begin{satz}\label{satz7_5}
	Sei $K$ ein Körper. Ist $\chr(K) \neq 0$, so gilt $\chr(K) = p$ für eine Primzahl $p$ und $\Pi(K) \cong \Z_p$. Ist $\chr(K) = 0$, so gilt $\Pi(K) \cong \Q$.
\end{satz}
\begin{proof}
	Nutze Bemerkung \ref{rem7_4} \ref{rem7_4_2}. Ist $\chr(K) = n > 1$, so gilt $\ker(\varphi_K) = n \Z \unlhd \Z$. Der Homomorphiesatz liefert 
	\[\Z_n = \faktor{\Z}{n \Z} = \faktor{\Z}{\ker(\varphi_K)} \cong \im(\varphi_K) \leq K.\]
	Da $K$ Körper ist, ist $\Z_n$ nullteilerfrei und $n = p$ somit Primzahl. Insbesondere ist $\Z_p \cong \im(\varphi_K)$ ein Körper mit $\Pi(K) \subseteq \im(\varphi_K)$. Da zudem $1_K \in \Pi(K)$ und $\Pi(K)$ abgeschlossen unter Addition ist, folgt $\Pi(K) = \im(\varphi_K) \cong \Z_p$, wie gewünscht.
	
	Sei nun $\chr(K) = 0$ und $\varphi_K \colon \Z \to K$ somit injektiv. Nach Aufgabe M.9.1 gilt 
	% Diagramm 1 20.01.2025
	mit $\psi_K\left(\frac{a}{b}\right) = \varphi_K(a) \cdot \varphi_K(b)^{-1}$. Da $\Q$ Körper ist, folgt $\psi_K$ injektiv und $\Q \cong \im(\psi_K) \leq K$. Somit ist $\Pi(K) \subseteq \im(\psi_K)$. Da $\Q$ (wie zuvor auch $\Z_p$) keinen echten Teilkörper hat, gilt $\Pi(K) = \im(\psi_K) \cong \Q$.
\end{proof}
\begin{beispiel}\label{beispiel7_6}
	\begin{enumerate}[label=(\arabic*)]
		\item $\Q, \R, \C$ haben Charakteristik $0$ und den Primkörper $\Q$. $\Z_p$ mit $p$ prim hat Charakteristik $p$ und ist selbst ein Primkörper.
		\item Sei $K$ ein Körper und $I \unlhd K[x]$ maximal. Nach Proposition \ref{prop6_3} ist $K[x] / I$ ein Körper und mit Satz \ref{satz7_5} folgt
		\begin{align*}
			\chr\left(\faktor{K[x]}{I}\right) &= \chr(K)\\
			\Pi\left(\faktor{K[x]}{I}\right) &\cong \Pi(K).
		\end{align*}
		\begin{proof}
			Sei $\chr(K) = 0$. Angenommen, es gibt eine Primzahl $p$ mit $p \cdot 1_{K[x] / I} 0_{K[x] / I}$ bzw. $p \cdot (1_K + I) = p\cdot 1_K + I = 0_K + I$. Dann ist $p \cdot 1_K \in I$. Nach Voraussetzung ist aber $p \cdot 1_K \neq 0$ und $p \cdot 1_K$ somit eine Einheit in $K$ bzw. in $K[x]$, sodass $I = K[x]$. Ein Widerspruch. Also $\chr(K[x] / I) = 0$. Der Fall $\chr(K) = p$ folgt analog.
		\end{proof}
	\end{enumerate}
\end{beispiel}
\begin{kor}\label{kor7_7}
	Sei $K$ ein endlicher Körper. Dann gibt es eine Primzahl $p$ und $n \in \N$ mit $|K| = p^n$.
\end{kor}
\begin{proof}
	Da $|K| < \infty$, folgt mit Satz \ref{satz7_5}, dass $\Pi(K) \cong \Z_p$ für $p$ prim. Dadurch wird $K$ zum endlich dimensionalen $\Z_p$-Vektorraum. Ist $\{v_1, \dots, v_n\}$ eine entsprechende Basis von $K$, so folgt 
	\[K = \left\{\sum_{i=1}^n \lambda_i v_i \mid \lambda_i \in \Z_p\right\}.\]
	Also $|K| = p^n$, wie gewünscht.
\end{proof}
\begin{beispiel}[Der Körper mit vier Elementen]\label{beispiel7_8}
	Sei $K := \Z_2 [x] / (x^2 + x + \bar{1})$. Da $x^2 + x + \bar{1}$ unzerlegbar in $\Z_2[x]$ ist, ist $K$ ein Körper nach Kapitel 6.
	
	Nach Beispiel \ref{beispiel7_6} (2) gilt $\chr(K) = 2$ und $\Pi(K) \cong \Z_2$. Schreibe $I := (x^2 + x + \bar{1}) \unlhd \Z_2[x]$. Als $\Z_2$-Vektorraum hat $K$ die Basis $\{\bar{1} + I, x + I\}$, so dass $K = \{\bar{0} + I, \bar{1} + I, x+I, (x+\bar{1}) + I\}$. 
	
	Es gilt z. B. $(x+I)^2 = (x+\bar{1}) + I$ und $(x+I)\cdot ((x + \bar{1}) + I) = \bar{1} + I$ (siehe Aufgabe M.13.3).
\end{beispiel}
\underline{Ausblick:} Für jede Primzahl $p$ und $n \in \N$ gibt es bis auf Isomorphie genau einen Körper mit $p^n$ Elementen.
\begin{definition}
	Sei $L / K$ eine Körpererweiterung. Ein Element heißt $\alpha \in L$ heißt \textbf{algebraisch über $K$}, wenn es ein Polynom $f \in K[x] \setminus \{0_{K[x]}\}$ mit $f(\alpha) = 0_L$ existiert. Andernfalls heißt $\alpha \in L$ \textbf{transzendent über $K$}. Die Körpererweiterung $L/K$ heißt \textbf{algebraisch}, wenn jedes $\alpha \in L$ algebraisch über $K$ ist.
\end{definition}
\begin{beispiel}\label{beispiel7_10}
	\begin{enumerate}[label=(\arabic*)]
		\item Ist $L/K$ eine Körpererweiterung, so ist jedes Element $\alpha \in K$ algebraisch über $K$, da es Nullstelle des Polynoms $x - \alpha \in K[x]$ ist.
		\item $\alpha = \sqrt{2} \in \R$ ist algebraisch über $\Q$, da $\alpha$ Nullstelle von $x^2 - 2 \in \Q[x]$ ist. Die Körpererweiterung $\R/\Q$ ist nicht algebraisch, da mit $\Q$ auch $\Q[x]$ abzählbar ist und jedes Polynom in $\Q[x]$ nur endlich viele Nullstellen hat.
		Somit gibt es in $\R$ nur abzählbar viele Elemente, die algebraisch über $\Q$ sind. Beispiele transzendenter Elemente über $\Q$ sind $\pi$ und $e$.
		\item Die Körpererweiterung $\C/\R$ ist algebraisch, da $\alpha = a+ bi \in \C$ Nullstelle von $(x-a)^2 + b^2 \in \R[x]$ ist.
	\end{enumerate}
\end{beispiel}
\begin{definition}
	Sei $L/K$ eine Körpererweiterung und $M \subseteq L$ eine Teilmenge. 
	\begin{enumerate}[label=(\alph*)]
		\item Den Schnitt
		\[K(M) := \bigcap_{K \leq K' \leq L, M \subseteq K'} K'\]
		aller Teilkörper $K'$ von $L$, die $K \cup M$ enthalten, nennen wir \textbf{$K$ adjungiert $M$}. $K(M)$ ist der kleinste Zwischenkörper $K \leq K(M) \leq L$ von $L/K$, der $M$ enthält. Ist $M = \{\alpha_1, \dots, \alpha_n\}$ endlich, schreiben wir auch $K(\alpha_1, \dots, \alpha_n)$ statt $K(M)$.
		\item Eine Körpererweiterung $K(\alpha)/K$ für $\alpha \in L$ nennen wir \textbf{einfach}. Das Element $\alpha$ heißt dann \textbf{primitiv}.
	\end{enumerate}
\end{definition}
\begin{rem}\label{rem7_12}
	 Sei $L/K$ eine Körpererweiterung. Für $\alpha \in L$ betrachten den Einsetzungshomomorphismus $\varphi_\alpha \colon K[x] \to L$ mit $f \mapsto f(\alpha)$ und $K[\alpha] = \im(\varphi_\alpha) = \{f(\alpha) \mid f \in K[x]\} \leq K(\alpha) \leq L$ (siehe Beispiel \ref{beispiel4_6} (3)).
	 
	 $\alpha$ ist transzendent über $K$ genau dann, wenn $\varphi_\alpha$ injektiv ist. In dem Fall gilt $K[\alpha] \cong K[x]$ und 
	 \[K[\alpha] \cong \quot(K[x]) = \left\{\frac{f}{g} \mid f,g \in K[x], g \neq 0_{K[x]}\right\}\]
	 mit $[K(\alpha) : K] = \infty$.	 
\end{rem}
 Ist $\alpha$ algebraisch, erhalten wir das folgende Resultat:
\begin{satz}\label{satz7_13}
	Sei $L /K$ eine Körpererweiterung und $\alpha \in L$ algebraisch über $K$.
	\begin{enumerate}[label=(\alph*)]
		\item Es gibt ein eindeutig bestimmtes normiertes Polynom $\mu_\alpha \in K[x] \setminus \{0_{K[x]}\}$ kleinsten Grades, so dass $\mu_\alpha(\alpha) = 0_L$ ($\mu_\alpha$ heißt \textbf{Minimalpolynom} von $\alpha$ über $K$).
		\item Das Minimalpolynom $\mu_\alpha$ von $\alpha$ über $K$ ist unzerlegbar in $K[x]$. Es ist das eindeutige normierte unzerlegbare Polynom in $K[x]$ mit $\alpha$ als Nullstelle.
		\item Es gilt $K[x] / (\mu_\alpha) \cong K[\alpha] = K(\alpha)$ und $[K(\alpha) : K] = \deg(\mu_\alpha)$. Insbesondere ist die einfache Körpererweiterung $K(\alpha) / K$ endlich.
	\end{enumerate}	
\end{satz}
\begin{proof}
	\begin{enumerate}[label=(\alph*)]
		\item Betrachte wieder den Einsetzungshomomorphismus $\varphi_\alpha \colon K[x] \to L$. Da $K[x]$ Hauptidealring ist, gilt $\ker(\varphi_\alpha) = (\mu_\alpha)$ für ein $\mu_\alpha \in K[x]$. Wähle $\mu_\alpha$ normiert. Da $\alpha \in L$ algebraisch über $K$, gilt $\mu_\alpha \neq 0_{K[x]}$. Sei $f \in K[x] \setminus \{0_{K[x]}\}$ mit $f(\alpha) = 0_L$. Dann gilt $f \in \ker(\varphi_\alpha)$ und somit existiert $g \in K[x] \setminus \{0_{K[x]}\}$ mit $f = \mu_\alpha \cdot g$. Die Gradformel liefert $\deg(f) = \deg(g) + \deg(\mu_\alpha) \geq \deg(\mu_\alpha)$. Angenommen, $f$ ist ebenfalls normiertes Polynom kleinsten Grades mit Nullstelle $\alpha$. Dann muss $g$ konstant sein, und da $\mu_\alpha$ und $f$ normiert, folgt $g = 1_K$ bzw. $f = \mu_\alpha$.
		\item Schreibe $\mu_\alpha = f \cdot g$ mit $f,g \in K[x] \setminus \{0_{K[x]}\}$. Wir müssen zeigen, dass $\deg(f) = 0$ oder $\deg(g) = 0$. Angenommen, $ 0 < \deg(f),\deg(g) < \deg(\mu_\alpha)$. Dann liefert 
		\[\mu_\alpha (\alpha) = f(\alpha) \cdot g(\alpha) = 0_L,\]
		dass $\alpha$ Nullstelle von $f$ oder $g$ ist. Dies widerspricht der Minimalität des Grades von $\mu_\alpha$.
		
		\item Nach Teil (b) und Kapitel 6 ist $K[x] / (\mu_\alpha)$ ein Körper. Der Homomorphiesatz liefert
		\[\faktor{K[x]}{(\mu_\alpha)} = \faktor{K[x]}{\ker(\varphi_\alpha)} \cong \im(\varphi_\alpha) = K[\alpha].\]
		Da $K[\alpha]$ also bereits Körper ist, folgt $K[x] = K(\alpha)$ und 
		\[[K(\alpha) : K] = \dim_K(K(\alpha)) = \dim_K\left(\faktor{K[x]}{(\mu_\alpha)}\right) = \deg(\mu_\alpha).\]
	\end{enumerate}
\end{proof}
\begin{kor}\label{kor7_14}
	Sei $L/K$ eine Körpererweiterung und $\alpha \in L$ algebraisch über $K$ mit $\deg(\mu_\alpha) = n$. Dann ist $B = \{1_K, \alpha, \dots, \alpha^{n-1}\}$ eine Basis von $K(\alpha)$ als $K$-Vektorraum.
\end{kor}
\begin{proof}
	Nach Satz \ref{satz7_13} gilt $K(\alpha) = K[\alpha] = \{f(\alpha) \mid f \in K[x]\}$. Insbesondere enthält $K(\alpha)$ die lineare Hülle 
	\begin{align*}
		\lin_K(1_K, \alpha, \dots, \alpha^{n-1}) &= \{\lambda_0 1_K + \lambda_1 \alpha + \dots + \lambda_{n-1} \alpha^{n-1} \mid \lambda_i \in K\}\\
		&= \{f(\alpha) \mid f \in K[x], \deg(f) \leq n - 1\}.
	\end{align*}
	Sei umgekehrt $f \in K[x]$ gegeben. Division mit Rest liefert $q,r \in K[x]$ mit $f = q \cdot \mu_\alpha + r$ und $\deg(r) < \deg{\mu_\alpha} = n$. Damit gilt 
	\begin{align*}
		f(\alpha) = q(\alpha) \cdot \underbrace{\mu_{\alpha}(\alpha)}_{=0_L} + r(\alpha) = r(\alpha) \in \lin_K(1_K, \alpha, \dots, \alpha^{n-1}).
	\end{align*}
	Es folgt $K(\alpha) = \lin_K(1_K, \alpha, \dots, \alpha^{n-1})$. Wir zeigen lineare Unabhängigkeit. Seien $\lambda_0, \lambda_1, \dots, \lambda_{n-1} \in K$ mit $\lambda_0 1_K + \lambda_1 \alpha + \dots + \lambda_{n-1} \alpha^{n-1} = 0_L$. Dann ist $f = \lambda_0 + \lambda_1 x + \dots + \lambda_{n-1} x^{n-1} \in K[x]$ mit $f(\alpha) = 0_L$ und 
	\begin{align*}
		\deg(f) \leq n - 1 < \deg(\mu_\alpha).
	\end{align*}
	Es folgt, dass $f = 0_{K[x]}$ und somit $\lambda_0 = \lambda_1 = \dots = \lambda_{n-1} = 0_K$. 
\end{proof}
\begin{beispiel}\label{beispiel7_15}
	\begin{enumerate}[label=(\arabic*)]
		\item Das Minimalpolynom von $\alpha = \sqrt{2}$ über $\Q$ ist $\mu_\alpha = x^2 - 2 \in \Q[x]$. Es folgt $\Q(\sqrt{2}) = \Q[\sqrt{2}] = \{a + b\sqrt{2} \mid a,b \in \Q\}$
		\item Das Minimalpolynom von $\alpha = \sqrt[3]{2}$ über $\Q$ ist $\mu_\alpha = x^3 - 2$. Beachte, dass $x^3 - 2$ nach Eisenstein bezüglich $p=2$ unzerlegbar in $\Z[x]$ ist und somit auch $\Q[x]$ nach Satz \ref{satz6_26}. Es gilt $\Q(\sqrt[3]{2}) = \Q[\sqrt[3]{2}] = \{a + b\sqrt[3]{2} + c\sqrt[3]{4} \mid a,b,c \in \Q\}$.
		\item Betrachte die sechste Einheitswurzel $\alpha = e^{2\pi/6} \in \Q$. Dann ist $\alpha$ Nullstelle von $f=x^6 - 1 \in \Q[x]$. Da $f = (x^3 - 1)(x^3 + 1)$ ist $f$ nicht Minimalpolynom von $\alpha$ über $\Q$. Mit $\alpha^3 = -1$ ist $\alpha$ Nullstelle von $g = x^3 + 1 = (x+1)(x^2 - x + 1)$ und wir erhalten das Minimalpolynom $\mu_\alpha = x^2 - x + 1$ von $\alpha$ über $\Q$.
	\end{enumerate}
\end{beispiel}
\begin{framed}
	\textit{Wir wollen den Zusammenhang zwischen endlichen Körpererweiterungen und der Adjunktion algebraischer Elemente besser verstehen.}
\end{framed}
\begin{satz}[Gradformel]\label{satz7_16}
	Sei $K \subseteq L \subseteq M$ Körper. Dann ist die Körpererweiterung $M/K$ endlich genau dann, wenn $M/L$ und $L/K$ endliche Körpererweiterungen sind. In diesem Fall gilt $[M:K] = [M:L] \cdot [L:K]$.
\end{satz}
\begin{proof}
	Ist die $\dim_L(M) = \infty$, so gibt es in $M$ eine unendliche Menge linear unabhängiger Vektoren über $L$. Diese sind auch linear unabhängig über $K$, d.h. $\dim_K(M) = \infty$.
	
	Ebenso folgt aus $\dim_K(L) = \infty$, dass $\dim_K(M) = \infty$.
	
	Seien $M/L$ und $L/K$ endliche Körpererweiterungen und $\{\alpha_1, \dots, \alpha_n\} \subset L$ eine Basis von $L$ als $K$-Vektorraum sowie $\{\beta_1, \dots, \beta_n\} \subset M$ eine Basis von $M$ als $L$-Vektorraum. 
	
	\underline{Genügt zu zeigen:} $B := \{\alpha_i \beta_j \mid i \in \{1, \dots, n\}, j \in \{1, \dots, m\}\}$ ist Basis von $M$ als $K$-Vektorraum. Sei $\gamma \in M$ beliebig. Schreibe $\gamma = \sum_{j=1}^m \mu_j \beta_j$ mit $\mu_1, \dots, \mu_m \in L$. Jedes $\mu_j$ lässt sich schreiben als Linearkombination
	\[\mu_j = \lambda_{1j} \alpha_1 + \dots + \lambda_{nj} \alpha_n\]
	mit $\lambda_{1n}, \dots, \lambda_{nj} \in K$. Insgesamt erhalten wir 
	\[\gamma = \sum_{j=1}^m \mu_j \beta_j = \sum_{j=1}^m \sum_{i=1}^n \lambda_{ij} \alpha_i \beta_j \in \lin_K(B).\]
	Für lineare Unabhängigkeit  betrachte $\lambda_{ij} \in K$ mit 
	\[\sum_{j=1}^m \sum_{i=1}^n \lambda_{ij} \alpha_i \beta_j = \sum_{j=1}^m \bigg(\underbrace{\sum_{i=1}^n \lambda_{ij} \alpha_i}_{=0_L} \bigg)\beta_j = 0_M.\]
	$=0_L$, da $\{\beta_1, \dots, \beta_m\}$ linear unabhängig über $L$ sind. Es folgt
	\[\lambda_{ij} = 0_K \quad\text{für alle}\quad i \in \{1, \dots, n\}, j \in \{1, \dots, m\},\;\text{da}\;\{\alpha_1,\dots, \alpha_n\} \;\text{linear unabhängig über}\; K.\]
\end{proof}
\begin{kor}\label{kor7_17}
	\begin{enumerate}[label=(\alph*)]
		\item Ist $L/K$ eine endliche Körpererweiterung, so ist $L/K$ algebraisch über $L = K(\alpha_1, \dots, \alpha_n)$ für geeignete $\alpha_1, \dots, \alpha_n \in L$.
		\item Ist $L/K$ eine Körpererweiterung mit $\alpha_1, \dots, a_n \in L$ algebraisch über $K$, so ist $K(\alpha_1, \dots, \alpha_n)/K$ endlich und somit algebraisch.
	\end{enumerate}
\end{kor}
\begin{proof}
	\begin{enumerate}[label=(\alph*)]
		\item Angenommen, es gibt $\alpha \in L$ transzendent über $K$. Wir erhalten Körpererweiterungen $K \leq K(\alpha) \leq L$, wobei $K(\alpha)/K$ nach Bemerkung \ref{rem7_12} nicht endlich ist. Also ist $L/K$ nicht endlich nach Satz \ref{satz7_16}. Ein Widerspruch. Somit ist $L/K$ algebraisch.
		
		Zudem gilt $L = K(\alpha_1, \dots, \alpha_n)$ für jede Basis $\{\alpha_1, \dots, \alpha_n\} \subset L$ von $L$ als $K$-Vektorraum.
		
		\item Betrachte die endliche Kette von Körpererweiterungen
		\[K \leq K(\alpha_1) \leq K(\alpha_1)(\alpha_2) = K(\alpha_1, \alpha_2) \leq \dots \leq K(\alpha_1, \dots, \alpha_n).\]
		Da $\alpha_i \in L$ algebraisch über $K$ ist, ist $\alpha_i$ algebraisch über $K(\alpha_1, \dots, \alpha_{i-1})$. Nach Satz \ref{satz7_13} sind die Körpererweiterungen 
		\[K(\alpha_1, \dots, \alpha_i) = \faktor{K(\alpha_1, \dots, a_{i-1})(\alpha_i)}{K(\alpha_1, \dots, \alpha_{i-1})} \]
		endlich. Mit Satz \ref{satz7_16} folgt induktiv, dass auch $K(\alpha_1, \dots, \alpha_n)/K$ endlich ist.
	\end{enumerate}
\end{proof}
\begin{beispiel}\label{beispiel7_18}
	Betrachte die Körper $\Q \leq \Q(\sqrt{2}) \leq \Q(\sqrt{2}, i)$. Nach Satz \ref{satz7_16} gilt
	\[[\Q(\sqrt{2},i) : \Q] = [\Q(\sqrt{2}, i) : \Q(\sqrt{2})] \cdot [\Q(\sqrt{2}) : \Q].\]
	Nach Beispiel \ref{beispiel7_15} (1) gilt $[\Q(\sqrt{2}) : \Q] = 2$ und $\{1, \sqrt{2}\}$ ist Basis von $\Q(\sqrt{2})$ als $\Q$-Vektorraum.
	Das Minimalpolynom von $\alpha = 1$ über $\Q(\sqrt{2})$ ist $\mu_\alpha = x^2 + 1 \in \Q(\sqrt{2})[x]$, d.h. auch $[\Q(\sqrt{2}, i) : \Q(\sqrt{2})] = 2$ und somit $[\Q(\sqrt{2}, i) : \Q] = 4$.
	
	Der Beweis von Satz \ref{satz7_16} zeigt, dass $\{1, \sqrt{2}, i, i \sqrt{2}\}$ eine Basis von $\Q(\sqrt{2}, i)$ als $\Q$-Vektorraum ist.
\end{beispiel}

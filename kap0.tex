
\section{Einführung}
\textbf{Algebra} bedeutet \textbf{Rechnen mit Gleichungen}. Wir konzentrieren uns auf Polynomialgleichungen:
\begin{itemize}
	\item Systeme linearer Gleichungen betrachtet man in der linearen Algebra
	\item Quadratische Gleichungen wie $x^2 + ax + b  = 0$ lernt man in der Schule zu lösen (Mitternachtsformel).
	\item Kubische Gleichungen wie z. B. $x^3 + ax = b$ mit $a,b > 0$ sind schon schwieriger. Eine Lösung ist gegeben durch
	\begin{align*}
		x = \sqrt[3]{\frac{b}{2} + \sqrt{\left(\frac{b}{2}\right)^2 + \left(\frac{a}{3}\right)^3}} + \sqrt[3]{\frac{b}{2} - \sqrt{\left(\frac{b}{2}\right)^2 + \left(\frac{a}{3}\right)^3}}.
	\end{align*}
	\item Gleichungen von Grad 4 können durch endliche Wurzelausdrücke aufgelöst werden (Cardano 1545).
	\item Für Gleichungen von Grad 5 ist dies im Allgemeinen nicht möglich (Abel 1884).
\end{itemize}

\paragraph{Moderner Zugang (Galois 1830): }
Gruppentheorie, Körpererweiterung, ... (Algebra)

Galoistheorie, Auflösbarkeit von Polynomgleichungen,... (Algebra 2)

\paragraph{Highlights in diesem Semester: } 
\begin{itemize}
	\item \textit{Sylowsätze zur Struktur endlicher Gruppen.}
	
	\textbf{Idee: } Untersuche Gruppen ausgehend von ihren Untergruppen, deren Ordnung eine maximale Primpotenz ist.
	\item \textit{Konstruktion mit Zirkel und Lineal.}
	
	\textbf{Fragestellung: } Welche Objekte in der Ebene erhalten wir mittels Elementarkonstruktionen? Würfelverdopplung? Quadratur des Kreises? Winkeldreiteilung? Regelmäßige $n$-Ecke?
	
	\textbf{Idee: } Nutze Theorie der Körpererweiterung.
\end{itemize}
% Algebra - Main Datei
\documentclass[12pt,a4paper]{article}
\usepackage[utf8]{inputenc}
\usepackage[T1]{fontenc}
\usepackage[left=4.0cm, right=4.0cm, top=2.5cm, bottom=2.5cm]{geometry}
\usepackage{graphicx} % Required for inserting images
\usepackage[ngerman]{babel}
\usepackage{amsmath}
\usepackage{amssymb}
\usepackage{amsthm}
\usepackage{mathrsfs}
\usepackage{thmtools}
\usepackage{amsfonts}
\usepackage{mathtools}
\usepackage{enumitem} 
\usepackage{xcolor}
\usepackage{thmtools}
\usepackage{xfrac}


\usepackage[colorlinks=true,linkcolor=blue]{hyperref}%

%\usepackage{mdframed}


% hyperref
\hypersetup{
	unicode=false,
	pdftoolbar=true,
	pdfmenubar=true,
	pdffitwindow=false,
	pdfstartview={FitH},
	pdftitle={Analysis III},
	pdfnewwindow=true,
	colorlinks=true,
	linkcolor=blue,
	citecolor=blue,
	filecolor=magenta,
	urlcolor=cyan, 
	linktoc=all,     
}
\hypersetup{linktocpage}

\geometry{
	left=3.0cm, 
	right=3.0cm}

\newcommand{\C}{\mathbb{C}} % Die Menge der komplexen Zahlen
\newcommand{\R}{\mathbb{R}} % Die Menge der reellen Zahlen
\newcommand{\Q}{\mathbb{Q}} % Die Menge der rationalen Zahlen
\newcommand{\Z}{\mathbb{Z}} % Die Menge der ganzen Zahlen
\newcommand{\N}{\mathbb{N}} % Die Menge der natürlichen Zahlen
\newcommand{\K}{\mathbb{K}} % Der Körper der reellen oder komplexen Zahlen
\renewcommand{\S}{\mathbb{S}}
\newcommand{\B}{\mathbb{B}}

\newcommand{\vektor}[1]{\underline{#1}}
\newcommand{\skalarprodukt}[2]{\langle #1, #2 \rangle}
\newcommand{\e}{\mathrm{e}}
\newcommand{\iu}{\mathrm{i}}
\newcommand{\id}{\mathrm{id}}
\renewcommand{\Re}{\mathbf{Re}}
\renewcommand{\Im}{\mathbf{Im}}
\renewcommand{\d}{\mathrm{d}}
\newcommand{\D}{\mathrm{D}}
\newcommand{\dx}{\mathrm{d}x}
\newcommand{\pot}{\mathfrak{P}}
\newcommand{\CD}{\mathscr{C}}
\newcommand{\rot}{\mathrm{rot}}


\newcommand{\im}[1]{\mathrm{im}(#1)}
%\renewcommand{\vec}[1]{\mathbf{#1}} % Vektor
\newcommand{\mat}[1]{\mathbf{#1}} % Matrix

\renewcommand{\qedsymbol}{$\blacksquare$}
\newcommand{\norm}[1]{\left\lVert#1\right\rVert} % Adjustable norm
\newcommand{\snorm}[1]{{\left\lVert#1\right\rVert}_\infty} % sup norm
\newcommand{\abs}[1]{\left|#1\right|}
\newcommand{\supp}{\mathrm{supp}}
\newcommand{\scprod}[2]{\left\langle#1,#2 \right\rangle}

\newcommand{\CF}{\mathscr{C}} % Continuous and differentiable function
\renewcommand{\div}{\mathrm{div}}
\newcommand{\tr}{\mathrm{tr}}


\newtheorem{thm}{Theorem}[section]
\theoremstyle{plain}
\newtheorem{satz}[thm]{Satz}
\newtheorem{lem}[thm]{Lemma}
\newtheorem{kor}[thm]{Korollar}
%\newtheorem{definition}[thm]{Definition}
%\newtheorem{rem}[thm]{Bemerkung}

\newtheorem*{rem}{Bemerkung}
\newtheorem*{folgerung}{Folgerung}

\theoremstyle{plain}
\newtheorem*{bemerkung}{Bemerkung}

\theoremstyle{definition}
\newtheorem*{definition}{Definition}

\theoremstyle{definition}
\newtheorem*{beispiel}{Beispiel}

\theoremstyle{plain}

% Beweisende: Schwarzes Quadrat
%\renewcommand{\qedsymbol}{$\blacksquare$}


%\setlength{\parindent}{0cm}

\begin{document}
		\begin{center}
		\vspace*{1cm}
		
		{\fontsize{40}{48} \textbf{A l g e b r a}}
		
		\vspace{0.5cm}
		
		Wintersemester 20$\sfrac{24}{25}$ \quad Prof. Dr. Frederik Marks
		
		\vspace{1.0cm}
		Version: \today
		
	\end{center}
	
	Diese Zusammenfassung enthält nicht alles, was in der Vorlesung vorgetragen wurde. Der Fokus liegt vor allem auf Definition, Sätzen und einigen wenigen Bemerkungen. Für Hinweise auf Fehler und Kritik bin ich stets dankbar.
	
	\tableofcontents
	
	\section{Gruppen}
\begin{definition}
	Eine \textbf{Gruppe} $(G, *)$ ist eine Menge $G$ mit einer binären Verknüpfung 
	\[* \colon G \times G \to G, \quad(g,h) \mapsto g*h,\]
	so dass gilt:
	\begin{enumerate}[label={\bfseries(G\arabic*)}]
		\item\label{gr1} $(a * b) * c  = a * (b * c)$ für alle $a,b,c \in G$ ($*$ assoziativ),
		\item\label{gr2} Es existiert ein $e \in G$, sodass für alle $a \in G$ gilt $a * e = a = e * a$ ($e$ neutrales Element),
		\item\label{gr3} Für alle $a \in G$ existiert ein $a' \in G$, sodass $a * a' = e = a' * a$ ($a'$ inverses Element zu $a$).
	\end{enumerate}
	Gilt zusätzlich
	\begin{enumerate}[start=4, label={\bfseries(G\arabic*)}]
		\item\label{gr4} $a * b = b * a$ für alle $a,b \in G$, so heißt $(G, *)$ \textbf{kommutativ} oder \textbf{abelsch}.
	\end{enumerate}
	Die \textbf{Ordnung} von $(G, *)$ ist $|G|$.
\end{definition}
\underline{Notation: } Wir schreiben meist $G$ statt $(G, *)$. Dabei ist $*$ oft entweder $+$ (\textbf{additive Gruppe}) oder $\cdot$ (\textbf{multiplikative Gruppe}). Dann schreibe 0 statt $e$ bzw. $-a$ statt $a'$ sowie $a-b := a + (-b)$, schreibe 1 statt $e$ bzw. $a^{-1}$ statt $a'$ sowie $ab := a \cdot b$.
\begin{rem}\label{rem1_2}
	\begin{enumerate}[label=(\roman*)]
		\item Das neutrale Element einer Gruppe $G$ ist eindeutig:
		
		Sind $e, f$ neutrale Element in $G$, so gilt $e = e * f = f$ nach \ref{gr2}.
		
		\item Inverse Elemente in $G$ sind eindeutig:
		
		Seien $a'$ und $a''$ Inverse zu $a \in G$. Dann gilt
		\[a' \overset{\text{\ref{gr2}}}{=}  a' * e \overset{\text{\ref{gr3}}}{=} a' * (a * a'') \overset{\text{\ref{gr1}}}{=}  ( a' * a) * a'' \overset{\text{\ref{gr3}}}{=} = e * a'' \overset{\text{\ref{gr2}}}{=} a''\]
		
		\item Für inverse Elemente in $G$ gilt $(a')' = a$ und $(a * b)' = b' * a'$.
	\end{enumerate}
	\begin{beispiel}\label{beispiel1_3}
		\begin{enumerate}
			\item $(R, +, \cdot)$ ein Ring $\implies$ $(R, +)$ abelsche Gruppe.
			
			 $(K, +, \cdot)$ ein Körper $\implies$ $(K, +)$ und $(K\setminus \{0\}, \cdot)$ abelsche Gruppe
			 
			$(V, +, \cdot)$ ein Vektorraum $\implies$ $(V, +)$ abelsche Gruppe.
			
			Zum Beispiel $V = M_n (K) := \{ n\times n\text{-Matrizen über einem Körper } K\}$.
			
			\item $\GL_n(K) := \{\text{invertierbare } n\times n\text{-Matrizen über einem Körper } K\}$ bildet eine Gruppe bzgl. Matrizenmultiplikation - die \textbf{allgemeine lineare Gruppe}. Diese ist für $n \geq 2$ nicht abelsch. 
			
			Weitere Beispiele: 
			\begin{itemize}
				\item $\SL_n(K) := \{\mat{A} \in \GL_n(K) \mid \det \mat{A} = 1\}$ - die \textbf{spezielle lineare Gruppe}.
				\item $\mathsf{O}_n(K) := \{\mat{A} \in \GL_n(K) \mid \mat{AA}^\top = \mat{E}_n\}$ - die \textbf{orthogonale Gruppe}.
			\end{itemize}
			
			\item $G = \{e\}$ ist die \textbf{triviale Gruppe}.
			
			Für $|G| = 2$ mit $G = \{1, g\}$, dann erhalten wir die eindeutige \textbf{Multiplikationstafel}:
			
			\begin{center}
				\begin{tabular}{c|cc}
					$\cdot$ & 1 & $g$ \\
					\hline
					1 & 1 & $g$ \\
					$g$ & $g$ & 1
				\end{tabular}
			\end{center}
				Für $|G| = 3$ mit $G = \{1, g, h\}$, dann erhalten wir die eindeutige Multiplikationstafel:
				\begin{center}
					\begin{tabular}{c|ccc}
						$\cdot$ & 1 & $g$ & $h$\\
						\hline
						1 & 1 & $g$ & $h$\\
						$g$ & $g$ & $h$ &  1\\
						$h$ & $h$ & 1 & $g$
					\end{tabular}
				\end{center}
				Für $|G| = 4$ wird es schwieriger.
			
			\item \textbf{Symmetriegruppen: } Sei $G$ die Menge der \textbf{Kongruenzabbildungen} (längenerhaltend, flächenerhaltend, winkelerhaltend) eines geometrischen Objektes auf sich selbst.
			%Bild			
			\item Sei $X \neq \emptyset$ eine Menge. Die \textbf{symmetrische Gruppe} auf $X$ ist gegeben durch $S_X = \{f\colon X \to X \mid f \text{ bijektiv}\}$ mit der gewöhnlichen Komposition von Abbildungen. 
			
			Für $X = \{1, \dots, n\}$ mit $n \in \N$ erhalten wir die \textbf{symmetrische Gruppe vom Grad $n$} und schreiben $S_X = S_n$. 
			
			Die Elemente in $S_n$ heißen \textbf{Permutationen} und es gilt 
			\[|S_n| = n!.\]
			
			\textbf{Matrixnotation: } Die Permutation $\sigma \colon \{1, \dots, n\} \to \{1, \dots, n\}$ mit $\sigma(i) = a_i$ für $1 \leq i \leq n$ schreiben wir auch als
			\[\sigma = \begin{pmatrix}
				1 & 2 & 3 & \dots & n\\
				a_1 & a_2 & a_3 & \dots & a_n
			\end{pmatrix}.\]
			
			\textbf{Zykelnotation: } Sei $\{a_1, \dots, a_r\} \subseteq \{1, \dots, n\}$ mit $a_i$ paarweise verschieden. Dann ist der \textbf{Zykel} $\sigma = (a_1, \dots, a_r)$ \textbf{der Länge $r$} definiert als die Permutation $\sigma \in S$ mit 
			\begin{align*}
				\sigma(a_1) &= a_2\\
				\sigma(a_2) &= a_3\\
				&\vdots\\
				\sigma(a_r) &= a_1\\
			\end{align*}
			und $\sigma(a) = a$ für alle $a \in \{1, \dots, n\} \setminus \{a_1, \dots, a_r\}$. Zykel der Länge 2 heißen \textbf{Transpositionen}. Zwei Zykel $(a_1 \dots a_r)$ und $(b_1 \dots b_s)$ heißen \textbf{disjunkt}, falls 
			\[\{a_1, \dots, a_r\} \cap \{b_1, \dots, b_s\} = \emptyset.\]
			Disjunkte Zykel kommutieren und jede Permutation lässt sich eindeutig als Komposition disjunkter Zykel schreiben.
			
			Zum Beispiel 
			\[\sigma = \begin{pmatrix}
				1 & 2 & 3 & 4 & 5\\
				3 & 4 & 1 & 5 & 2
			\end{pmatrix}\]
			entspricht der Permutation $(13)(245)$ in Zykelschreibweise.
			
			\item Seien $G$ und $H$ Gruppen. Dann wird auch $G \times H$ zu einer Gruppe durch
			\[(g_1, h_1) * (g_2, h_2) := (g_1 *_G g_2, h_1 *_H h_2).\]
			$G \times H$ heißt das \textbf{direkte Produkt} von $G$ und $H$.
		\end{enumerate}
	\end{beispiel}
\end{rem}
\begin{definition}
	Seien $G$ und $H$ Gruppen
	\begin{enumerate}[label=(\alph*)]
		\item Eine Abbildung $\varphi \colon G \to H$ heißt \textbf{Gruppenhomomorphismus}, falls 
		\[\varphi(g_1 *_G g_2) = \varphi(g_1) *_H \varphi(g_2)\]
		für alle $g_1, g_2 \in G$. Ist $\varphi$ auch bijektiv, sprechen wir von einem \textbf{Isomorphismus}. Die Gruppen $G$ und $H$ heißen dann \textbf{isomorph} und wir schreiben $G \cong H$.
		\item $H$ heißt \textbf{Untergruppe} von $G$, falls $H \subseteq G$ und die Inklusionsabbildung $H \to G$ ein Gruppenhomomorphismus ist. Wir schreiben $H \leq G$.
	\end{enumerate}
\end{definition}

\begin{rem}
	\begin{enumerate}[label=(\roman*)]
		\item Sei $\varphi \colon G \to H$ ein Gruppenhomomorphismus. Dann gilt
		\[\varphi(1_G) = 1_H,\]
		denn es gilt
		\[\varphi(1_G) = \varphi(1_G 1_G) = \varphi(1_G) \varphi(1_G)\]
		und Multiplikation mit $\varphi(1_G)^{-1}$ liefert 
		\[1_H = \varphi(1_G).\]
		
		Zudem gilt
		\[\varphi(g^{-1}) = \varphi(g)^{-1},\]
		denn
		\[\varphi(g) \varphi(g^{-1}) = \varphi(gg^{-1}) = \varphi(1_G) = 1_H = \dots = \varphi(g^{-1}) \varphi(g).\]
		Nutze Bemerkung \ref{rem1_2} (ii) zum Beweis der Eindeutigkeit.
		
		\item Isomorphie ist eine Äquivalenzrelation. Isomorphe Gruppen betrachten wir als wesensgleich in Hinblick auf Eigenschaften, Multiplikationstafeln, Eindeutigkeitsaussagen etc.
		
		\item Sei $G$ eine Gruppe und $H \subseteq G$. Dann ist $H \leq G$ Untergruppe genau dann, wenn 
		\begin{itemize}
			\item $1_G \in H$,
			\item $h_1, h_2 \in H \;\Rightarrow\; h_1h_2 \in H$,
			\item $h \in H \;\Rightarrow\; h^{-1} \in H$.
		\end{itemize}
	\end{enumerate}
\end{rem}

\begin{beispiel}\label{beispiel1_6}
	\begin{enumerate}
		\item $\det \colon \GL_n(K) \to K \setminus \{0\}$ ist ein Gruppenhomomorphismus, da $\det(\mat{AB}) = \det(\mat{A})\det(\mat{B})$ für alle $\mat{A},\mat{B} \in \GL_n(K)$. $\SL_n(K) \leq \GL_n(K)$ und $\mathrm{O} \leq \GL_n(K)$ sind Untergruppen.
		
		\item $\exp \colon \R \to \R_{>0}$ ist ein Gruppenisomorphismus, da $\exp(x+y) = \exp(x)\exp(y)$ für alle $x,y \in \R$.
		
		\item Sei $G = \{\id, \gamma_A, \gamma_B\}$ die Symmetriegruppe eines Rechtecks wie in Beispiel \ref{beispiel1_3} (4). Dann ist $G$ isomorph zum direkten Produkt $S_1 \times S_2$. Ein Isomorphismus ist gegeben durch $G \to S_1 \times S_2$ mit
		\begin{align*}
			\id &\mapsto (\id, \id),\\
			\gamma_A &\mapsto (\id, (12)),\\
			\gamma_B &\mapsto ((12), \id),\\
			\gamma_{180^\circ} &\mapsto ((12), (12)).
		\end{align*}
	 	Vergleiche Multiplikationstafeln.
		
		\item Für $n \in \N$ ist $\varphi \colon S_n \to S_{n+1}$ mit 
		\[\varphi(\sigma) = \begin{pmatrix}
			1 & \dots & n & n+1\\
			\sigma(1) & \dots & \sigma(n) & n+1
		\end{pmatrix}\]
		ein injektiver Gruppenhomomorphismus, d.h. $S_n$ ist isomorph zu einer Untergruppe von $S_{n+1}$.
		
		\item Sei $G$ eine Gruppe und $g \in G$. Dann ist die Abbildung 
		\[C_g \colon G \to G, \quad h \mapsto ghg^{-1},\]
		genannt \textbf{Konjugation mit $g$}, ein Gruppenisomorphismus mit
		\[(C_g)^{-1} = C_{g^{-1}}.\]
		Die Abbildungen
		\begin{align*}
			L_g &\colon G \to G, \quad h \mapsto gh\\
			R_g &\colon G \to G, \quad h \mapsto hg,
		\end{align*}
		genannt \textbf{Links-} und \textbf{Rechtsmultiplikation mit $g$}, sind im Allgemeinen keine Gruppenhomomorphismen, aber bijektiv!
		
		\item Sei $\varphi \colon G \to H$ ein Gruppenhomomorphismus. Dann sind $\ker(\varphi) := \{g \in G \mid \varphi(g) = 1_H\}$ (\textbf{Kern von $\varphi$}) und $\im(\varphi) = \{\varphi(g) \mid g \in G\}$ (\textbf{Bild von $\varphi$}) Untergruppen von $G$ bzw. $H$. Zudem ist $\varphi$ injektiv genau dann, wenn $\ker(\varphi) = \{1_G\}$.
		
		\begin{inlproof}
			\glqq{}Nur dann\grqq: Sei $g \in G$ mit $\varphi(g) = 1_H$. Da $\varphi(1_G) = 1_H$, folgt $g = 1_G$.
			
			\glqq{}Dann\grqq: Seien $a,b \in G$ mit $\varphi(a) = \varphi(b)$. Da $1_H = \varphi(a) \varphi(b)^{-1} = \varphi(ab^{-1})$, folgt $ab^{-1} = 1_G$ und somit $a = b$.
		\end{inlproof}
	\end{enumerate}
\end{beispiel}

\begin{satz}[Satz von Cayley]\label{satz1_7}
	Jede Gruppe ist isomorph zu einer Untergruppe einer symmetrischen Gruppe, d.h. einer Gruppe von Bijektionen.
\end{satz}
\begin{proof}
	Sei $G$ eine Gruppe. Wir konstruieren einen injektiven Gruppenhomomorphismus $\varphi \colon G \to S_G = \{f \colon G \to G \mid f \text{ bijektiv}\}$ durch $g \mapsto L_g$ (siehe Beispiele \ref{beispiel1_3} (5) und \ref{beispiel1_6} (5)). 
	
	\underline{Z.z.: $\varphi$ ist Gruppenhomomorphismus.} Seien $g, h \in G$. Dann gilt für alle $a \in G$
	\begin{align*}
		\varphi(gh)(a) = L_{gh}(a) = (gh)(a) = g(ha) = L_g(ha) = (L_g \circ L_h)(a) = (\varphi(g) \circ \varphi(h))(a)
	\end{align*} 
	und somit $\varphi(gh) = \varphi(g) \circ \varphi(h)$. 
	
	\underline{Z.z.: $\varphi$ ist injektiv:} Seien $g,h \in G$ mit
	\[L_g = \varphi(g) = \varphi(h) = L_h.\]
	Dann gilt 
	\[g = L_g(1_G) = L_h(1_G) = h.\]
	Ist $G$ endlich mit $|G| = n$, so ist $G$ isomorph zu einer Untergruppe von $S_n$, der symmetrischen Gruppe vom Grad $n$.
\end{proof}
% Vorlesung 21.10.2024:
\begin{beispiel}\label{beispiel1_8}
	Für $\sigma \in S_n$ definiere $\sgn(\sigma) = (-1)^{\omega(\sigma)}$, wobei 
	\[\omega(\sigma) = \left|\{(i,j) \mid 1 \leq i < j \leq n, \sigma(j) < \sigma(i)\}\right|,\]
	genannt die \textbf{Anzahl der Fehlstände von $\sigma$}. Dann ist
	\[\sgn(\sigma) = \prod_{i < j} \frac{\sigma(i) - \sigma(j)}{i - j}.\]
	Zum Beispiel ist für 
	\[\sigma = \begin{pmatrix}
		1 & 2 & 3 \\
		3 & 2 & 1
	\end{pmatrix}\]
	das Signum
	\[\sgn(\sigma) = \frac{3-2}{1-2} \cdot \frac{3-1}{1-3} \cdot \frac{2-1}{2-3} = (-1)^3 = -1.\]
	$\sigma$ hat 3 Fehlstände. Die Abbildung $\sgn \colon S_n \to (\{-1, 1\}, \cdot)$ ist ein Gruppenhomomorphismus, da für $\sigma, \pi \in S_n$ gilt
	\[\sgn(\sigma \pi) = \prod_{i < j} \frac{\sigma(\pi(i)) - \sigma(\pi(j))}{i - j} = \prod_{i < j} \frac{\sigma(\pi(i)) - \sigma(\pi(j))}{\pi(i) - \pi(j)} \cdot \prod_{i < j} \frac{\pi(i) - \pi(j)}{i - j} = \sgn(\sigma) \cdot \sgn(\pi)\]
	Nach Beispiel \ref{beispiel1_6} (6) ist $A_n := \ker(\sgn)$ Untergruppe von $S_n$. $A_n$ heißt die \textbf{alternierende Gruppe vom Grad $n$}.
	
	Zum Beispiel ist für $S_3 = \{\id, (12), (13), (23), (123), (132)\}$: 
	\[A_3 = \{\id, (123), (132)\}.\]
\end{beispiel}
\begin{definition}
	Sei $G$ eine Gruppe und $H \leq G$. Für $g \in G$ heißt
	\begin{align*}
		gH &:= \{gh \mid h \in H\} &&\textbf{Linksnebenklasse von $H$ in $G$},\\
		Hg &:= \{hg \mid h \in H\} &&\textbf{Rechtsnebenklasse von $H$ in $G$}.
	\end{align*}
	Schreibe $G / H = \{gH \mid g \in G\}$ und $H \backslash G = \{Hg \mid g \in G\}$. Definiere $[G : H] := \left|G/H\right|$.
\end{definition}
\begin{rem}
	\begin{enumerate}[label=(\roman*)]
		\item Nach Beispiel \ref{beispiel1_6} (5) gilt
		\[|gH| = |H| = |Hg|\]
		für alle $g \in G$.
		\item Nach Aufgabe M.1.3 definieren die Relationen $a \sim_1 b \Leftrightarrow a^{-1}b \in H$ bzw. $a \sim_2 b \Leftrightarrow ab^{-1} \in H$ Äquivalenzrelationen auf $G$. Die Äquivalenzklassen sind genau die Links- bzw. Rechtsnebenklassen von $H$ in $G$.
		
		Insbesondere gilt 
		\begin{align*}
			G &= \bigcup_{g \in G} gH = \bigsqcup_{N \in G/H} N,\\
			G &= \bigcup_{g \in G} Hg = \bigsqcup_{N \in H \backslash G} N
		\end{align*}
		\begin{inlproof}
			Für $[a] = \{b \mid a_1 \sim_1 b\}$ gilt: $b \in [a] \Leftrightarrow a^{-1}b \in H \;\Leftrightarrow\; \exists h \in H : a^{-1}b = h \;\Leftrightarrow\; \exists h \in H : b = ah \;\Leftrightarrow\; b \in aH$.
		\end{inlproof}
	\end{enumerate}
\end{rem}
\begin{satz}[Satz von Lagrange]\label{satz1_11}
	Sei $G$ eine endliche Gruppe und $H \leq G$. Dann gilt
	\[|G| = [G : H] \cdot |H|.\]
	Insbesondere also $|H| \big| |G|$ ($|H|$ teilt $|G|$).
\end{satz}
\begin{proof}
	Wähle $\{g_1, \dots, g_r\} \subseteq G$, so dass
	\[G = \bigsqcup_{j=1}^r g_i H\]
	gilt, d.h. $[G:H] = r$. Dann gilt
	\[|G| = \sum_{j=1}^{r} |g_jH| = \sum_{j=1}^r |H| = r|H| = [G:H] \cdot |H|.\]
\end{proof}
\begin{leftbar}
	Definiert $g_1H * g_2H := g_1g_2H$ eine Gruppenstruktur auf der $G/H$?

\underline{Problem:} Wohldefiniertheit.
\end{leftbar}
\begin{beispiel}
	Sei $G = S_3$ und $H = \{\id, (12)\}$. Da $|G|= 6$ und $|H| = 2$, folgt mit Satz \ref{satz1_11}, dass $[G:H] = 3$. Es gibt also 3 Linksnebenklassen:
	\[\id H = H, \quad (23)H = \{(23), (132)\} \quad\text{und}\quad (13)H = \{(13), (123)\}\]
	Wir erhalten
	\[(23)H * (13)H = (123)H\]
	und
	\[(23)H = (132)H * (13)H = (12)H.\]
	$(123)H \neq (12)H$ und wir folgern:
\end{beispiel}
\begin{leftbar}
	Wir brauchen eine stärkere Bedingung als $H \leq G$ Untergruppe.
\end{leftbar}
\begin{definition}
	Sei $G$ eine Gruppe und $H \leq G$. 
	\begin{enumerate}[label=(\alph*)]
		\item Für $g \in G$ heißt $gHg^{-1} = \{ghg^{-1} \mid h \in H\}$ die \textbf{zu $H$ konjugierte Untergruppe}. Nach Beispiel \ref{beispiel1_6} (5), (6) gilt $gHg^{-1} \leq G$ mit $|gHg^{-1}| = |H|$ für alle $g \in G$.
		\item $H \leq G$ heißt \textbf{normale Untergruppe} oder auch \textbf{Normalteiler}, falls 
		\[gHg^{-1} = H \quad\text{für alle } g \in G.\]
		Wir schreiben $H \unlhd G$.
	\end{enumerate}
\end{definition}
\begin{rem}\label{rem1_14}
	Die folgenden Aussagen sind äquivalent
	\begin{enumerate}[label=(\roman*)]
		\item $H \unlhd G$.
		\item $gH = Hg$ für alle $g \in G$, d.h. für jedes $g \in G$ stimmt die Linksnebenklasse mit der Rechtsnebenklasse überein.
		\item $gHg^{-1} \subseteq H$ für alle $g \in G$.
	\end{enumerate}
\end{rem}
\begin{proof}
	\glqq{}(iii) $\Rightarrow$ (ii)\grqq: Sei $g \in G$. Nach Voraussetzung gilt $gH \subseteq Hg$ sowie für $h \in H: hg = g(g^{-1}hg) \in gH$, also auch $Hg \subset gH$, wie gewünscht.
\end{proof}
\begin{beispiel}\label{beispiel1_15}
	\begin{enumerate}[label=(\arabic*)]
		\item $\{1_G\} \unlhd G$ und $G \unlhd G$ sind Normalteiler.
		
		Eine Gruppe $G \neq \{1_G\}$ heißt \textbf{einfach}, wenn sie nur die trivialen Normalteiler $\{1_G\}$ und $G$ hat.
		
		\item Ist $G$ abelsch, so ist jede Untergruppe Normalteiler.
		
		\item Sei $H \leq G$ mit $[G : H] = 2$, dann gilt $H \unlhd G$. 
		\begin{inlproof}
			\[[G:H] = 2 \;\implies\; G/H = \{H, G\backslash H\}\]
			Damit stimmen Links- und Rechtsnebenklassen überein.			
		\end{inlproof}
		Betrachte zum Beispiel die alternierende Gruppe $A_n \leq S_n$ mit $A_n = \{\sigma \in S_n \mid \sgn(\sigma) = 1\}$ für $n \geq 2$. Für $\pi \in S_n$ mit $\sgn(\pi) = -1$ gilt
		\[\pi A_n = \{\sigma \in S_n \mid \sgn(\sigma) = -1\}\]
		\begin{inlproof}
			$\subseteq$: Gilt, da $\sgn$ Gruppenhomomorphimus ist.
			
			$\supseteq$: Sei $\sigma \in S_n$ mit $\sgn(\sigma) = -1$. Dann gilt $\sigma = \pi(\pi^{-1} \sigma) \in \pi A_n$, da $\sgn(\pi^{-1} \sigma) = \sgn(\pi^{-1})\sgn(\sigma) = (-1)^2 = 1$.
		\end{inlproof}
		Es folgt, dass $[S_n : A_n] = 2$ und damit $A_n \unlhd S_n$. Insbesondere ist nach Satz \ref{satz1_11} $|A_n| = n! / 2$, 
		da z.B. $(13)H_1 (13) = H_3$.
		
		\item Die symmetrische Gruppe $S_3$ hat folgende Untergruppen:
		%\begin{center}
		%	\includegraphics[scale=0.2]{images/beispiel1_15-4}
		%\end{center}
		% https://q.uiver.app/#q=WzAsNixbMywwLCJTXzMiXSxbMCwyLCJBXzMgPSBcXHtcXGlkLCAoMTIzKSwgKDEzMilcXH0iXSxbMiwyLCJIXzEgPSBcXHtcXGlkLCAoMTIpXFx9Il0sWzQsMiwiSF8yID0gXFx7XFxpZCwgKDEzKVxcfSJdLFs2LDIsIkhfMyA9IFxce1xcaWQsICgyMylcXH0iXSxbMyw0LCJcXHtcXGlkXFx9Il0sWzAsMSwiIiwwLHsic3R5bGUiOnsiaGVhZCI6eyJuYW1lIjoibm9uZSJ9fX1dLFswLDIsIiIsMCx7InN0eWxlIjp7ImhlYWQiOnsibmFtZSI6Im5vbmUifX19XSxbMCwzLCIiLDAseyJzdHlsZSI6eyJoZWFkIjp7Im5hbWUiOiJub25lIn19fV0sWzAsNCwiIiwwLHsic3R5bGUiOnsiaGVhZCI6eyJuYW1lIjoibm9uZSJ9fX1dLFsxLDUsIiIsMCx7InN0eWxlIjp7ImhlYWQiOnsibmFtZSI6Im5vbmUifX19XSxbMiw1LCIiLDAseyJzdHlsZSI6eyJoZWFkIjp7Im5hbWUiOiJub25lIn19fV0sWzMsNSwiIiwwLHsic3R5bGUiOnsiaGVhZCI6eyJuYW1lIjoibm9uZSJ9fX1dLFs0LDUsIiIsMCx7InN0eWxlIjp7ImhlYWQiOnsibmFtZSI6Im5vbmUifX19XV0=
		\[\begin{tikzcd}[column sep=tiny]
			&&& {S_3} \\
			\\
			{A_3 = \{\id, (123), (132)\}} && {H_1 = \{\id, (12)\}} && {H_2 = \{\id, (13)\}} && {H_3 = \{\id, (23)\}} \\
			\\
			&&& {\{\id\}}
			\arrow[no head, from=1-4, to=3-1]
			\arrow[no head, from=1-4, to=3-3]
			\arrow[no head, from=1-4, to=3-5]
			\arrow[no head, from=1-4, to=3-7]
			\arrow[no head, from=3-1, to=5-4]
			\arrow[no head, from=3-3, to=5-4]
			\arrow[no head, from=3-5, to=5-4]
			\arrow[no head, from=3-7, to=5-4]
		\end{tikzcd}\]
		
		Es gilt $A_3 \unlhd S_3$. $H_1, H_2, H_3$ sind jedoch keine Normalteiler, da zum Beispiel $(13)H_1(13) = H_3$.
	\end{enumerate}
\end{beispiel}
\begin{satz}\label{satz1_16}
	Sei $G$ eine Gruppe und $H \unlhd G$.
	\begin{enumerate}[label=(\alph*)]
		\item Die Menge $G/H = \{gH \mid g \in G\}$ mit Multiplikation $(aH) \cdot (bH) = (ab)H$ für alle $a,b \in G$ ist eine Gruppe. Sie heißt \textbf{Faktorgruppe} oder \textbf{Quotientengruppe}.
		\item Die Abbildung $\pi \colon G \to G/H$ mit $g \mapsto gH$ ist ein surjektiver Gruppenhomomorphismus mit $\ker(\pi) = H$. $\pi $ heißt \textbf{kanonische Projektion}.
	\end{enumerate}
\end{satz}
\begin{proof}
	\begin{enumerate}[label=(\alph*)]
		\item Die Multiplikation ist wohldefiniert. Sei $aH = a'H$ und $bH = b'H$ bzw. $a^{-1}a' \in H$ und $b^{-1}b' \in H$. 
		
		\underline{Z.z.: $abH = a'b'H$ bzw. $(ab)^{-1} a'b' \in H$.}
		
		\begin{align*}
			(ab)^{-1} a'b' = b^{-1} a^{-1} a' b' = b^{-1} \underbrace{(a^{-1} a')}_{\in H} b \underbrace{(b^{-1} b')}_{\in H} \in H
		\end{align*}
		
		\paragraph{Gruppenaxiome:} 		
		\begin{enumerate}[label={\bfseries(G\arabic*)}]
			\item Multiplikation in $G$ und in $G/H$ assoziativ.
			\item $H$ ist neutrales Element in $G/H$.
			\item Das Inverse zu $gH$ ist $g^{-1}H$.
		\end{enumerate}
		
		\item Für alle $a,b \in G$ gilt 
		\[\pi(ab) = (ab)H = aH \cdot bH = \pi(a)\cdot\pi(b).\]
		Also ist $\pi$ ein Gruppenhomomorphismus mit
		\[\ker(\pi) = \{g \in G \mid gH = H\} = H\]
	\end{enumerate}
\end{proof}
\begin{beispiel}
	Betrachte $n\Z \unlhd \Z$ für $n \in \N$ mit
	\[\faktor{\Z}{n\Z} = \{n\Z, 1 + n\Z, \dots, (n-1) + n\Z\} =: \Z_n,\]
	wobei $(a + n\Z) + (b + n\Z) = (a + b) + n\Z$. Statt $a + n\Z$ schreiben wir auch $\bar{a}$.
	
	Wir können auch Normalteiler in $\Z/n\Z$ betrachten, zum Beispiel
	\[H = \{\bar{0}, \bar{3}\} \unlhd \Z_6.\]
	$H$ ist offensichtlich Untergruppe und auch noch Normalteiler, da $\Z_6$ abelsch ist. Es gilt
	\[\faktor{\Z_6}{H} = \{H, \bar{1} + H, \bar{2} + H\} \cong \Z_3\]
\end{beispiel}

\begin{prop}\label{prop1_18}
	Sei $H \leq G$ eine Untergruppe. Dann gilt $H \unlhd G$ genau dann, wenn $H$ Kern eines Gruppenhomomorphismus ist, der in $G$ startet.
\end{prop}
\begin{proof}
	\glqq{}$\Rightarrow$\grqq: Folgt aus Satz \ref{satz1_16} (b).
	
	\glqq{}$\Leftarrow$\grqq: Sei $\varphi \colon G \to G'$ Gruppenhomomorphismus und $H = \ker(\varphi)$. Nach Beispiel \ref{beispiel1_6} (6) ist $H \leq G$ Untergruppe. Sei nun $g \in G$ und $h \in H$. Dann gilt
	\[\varphi(ghg^{-1}) = \varphi(g) \varphi(h) \varphi(g)^{-1} = \varphi(g) 1_{G'} \varphi(g)^{-1} = 1_{G'}\]
	Es folgt $gHg^{-1} \subseteq H$ und $H \unlhd G$ ist Normalteiler nach Bemerkung \ref{rem1_14}.
\end{proof}
\begin{beispiel}\label{beispiel1_19}
	\begin{enumerate}[label=(\arabic*)]
		\item Nach Beispiel \ref{beispiel1_8} ist $\sgn \colon S_n \to (\{-1,1\}, \cdot)$ für $n > 1$ ein surjektiver Gruppenhomomorphismus. Nach Beispiel \ref{beispiel1_15} (3) gilt
		\[\faktor{S_n}{\ker(\sgn)} = \faktor{S_n}{A_n} = \{A_n, \pi A_n\},\]
		wobei $\sgn(\pi) = -1$. Insbesondere gilt 
		\[\faktor{S_n}{\ker(\sgn)} \cong \Z_2 \cong (\{-1, 1\}, \cdot) = \im(\sgn).\]
		\item Sei $\varphi \colon \R\setminus\{0\}$ mit $x\mapsto |x|$. Dann ist $\varphi$ ein Gruppenhomomorphismus mit $\ker(\varphi) =\{\pm1\}$ und $\im(\varphi) = \R_{>0}$.
		
		Es gilt
		\[\faktor{\R \setminus \{0\}}{\ker(\varphi)} = \faktor{\R \setminus \{0\}}{(\{-1, 1\}, \cdot)} \cong \R_{>0} = \im{\varphi}\]
		
		\item Nach Beispiel \ref{beispiel1_6} (1) ist $\det \colon \GL_n(K) \to K\setminus\{0\}$ ein surjektiver Gruppenhomomorphismus. Es gilt $\ker(\det) = \SL_n(K)$, sowie
		\[\faktor{\GL_n(K)}{\ker(\det)} = \faktor{\GL_n(K)}{\SL_n(K))} \cong K\setminus\{0\} = \im(\det).\]
		Anstatt diesen Isomorphismus explizit nachzuprüfen, beweisen wir
	\end{enumerate}
\end{beispiel}
\begin{satz}[Homomorphiesatz]\label{satz1_20}
	Sei $\varphi \colon G \to H$ ein Gruppenhomomorphismus. Dann gilt
	\[\faktor{G}{\ker(\varphi)} \cong \im(\varphi).\]
	Insbesondere gilt $|G| = |\ker(\varphi)| \cdot |\im(\varphi)|$ für $G$ endlich.
\end{satz}
\begin{proof}
	Betrachte die Abbildung $\bar{\varphi} \colon G/\ker(\varphi) \to \im(\varphi)$ mit $g\ker(\varphi) \mapsto \varphi(g)$. 
	
	$\bar{\varphi}$ ist wohldefiniert, da für $g\ker(\varphi) = g'\ker(\varphi)$ gilt: Es existiert ein $x \in \ker(\varphi)$ mit $g = g'x$ und somit 
	\[\varphi(g) = \varphi(g'x) = \varphi(g')\varphi(x) = \varphi(g') 1_H = \varphi(g')\]
	
	$\bar{\varphi}$ ist Gruppenhomomorphismus, da für $g, g' \in G$ gilt:
	\[\bar{\varphi}(g\ker(\varphi) g'\ker(\varphi)) = \bar{\varphi}(gg'\ker(\varphi)) = \varphi(gg') = \varphi(g) \varphi(g') = \bar{\varphi}(g\ker(\varphi)) \bar{\varphi}(g'\ker(\varphi))\]
	$\bar{\varphi}$ ist nach Konstruktion surjektiv.
	
	$\bar{\varphi}$ ist injektiv, da aus $\bar{\varphi}(g\ker(\varphi)) = \varphi(g) = 1_H$ folgt, dass $g \in \ker(\varphi)$ und somit $g\ker(\varphi) = \ker(\varphi) = 1_{(G/\ker(\varphi))}$ (siehe Beispiel \ref{beispiel1_6} (6)). $\bar{\varphi}$ ist also ein Gruppenisomorphismus, so dass
	\[\faktor{G}{\ker(\varphi)} \cong \im (\varphi).\]
	Für endliche Gruppen $G$ folgt schließlich mit Satz \ref{satz1_11}
	\[|G| = |\ker(\varphi)| \cdot |\im(\varphi)|.\]
\end{proof}

\begin{satz}[Isomorphiesätze]\label{satz1_21}
	Sei $G$ eine Gruppe und $H_1, H_2 \leq G$ Untergruppen.
	\begin{enumerate}[label=(\alph*)]
		\item Ist $H_1 \unlhd G$ Normalteiler, so gilt $H_1H_2 = H_2H_1 \leq G$ und
		\[\faktor{H_1H_2}{H_1} \cong \faktor{H_2}{(H_1 \cap H_2)}.\]
		\item Sind $H_1, H_2 \unlhd G$ Normalteiler mit $H_1 \leq H_2 \leq G$, so gilt
		\[\faktor{G/H_1}{H_2/H_1} \cong \faktor{G}{H_2}\]
	\end{enumerate}
\end{satz}
\begin{proof}
	\begin{enumerate}[label=(\alph*)]
		\item Da $H_1 \unlhd G$, gilt $h_2 H_1 = H_1 h_2$ für alle $h_2 \in H_2$. Somit $H_1H_2 = H_2H_1$. $H_1H_2$ ist Untergruppe, da $1_G \in H_1H_2$ und daher $1_G = 1_G 1_G \in H_1$. Für $h_1, h'_1 \in H$ und $h_2, h'_2 \in H_2$ gilt 
		\[(h_1h_2)(h'_1h'_2) = h_1(\underbrace{h_2h'_1}_{\in H_1H_2}) h'_2 \in H_1 H_2 \]
		Für $h_1 \in H_1, h_2 \in H_2$ gilt $(h_1h_2)^{-1} = h_2^{-1} h_1^{-1} \in H_2 H_1 = H_1 H_2$. Da $H_1 \unlhd G$, gilt auch $H_1 \unlhd H_1H_2$.
		
		Betrachte nun den Homomorphismus		
		\[
		\begin{tikzcd}[column sep=huge, row sep=huge]
			H_2 \arrow{r}{\varphi} \arrow[swap]{d}{(\textbf{Inklusion})\;i} & \faktor{H_1H_2}{H_1} \\
			H_1H_2 \arrow[swap]{ur}{\pi \; (\textbf{kanonische Bijektion})}
		\end{tikzcd}
		\]
		mit $\varphi(h_2) = h_2H_1$. Dann gilt
		\[\ker(\varphi) = \{h_2 \in H_2 \mid h_2 H_1 = H_1\} = H_1 \cap H_2\]
		Da $\varphi$ nach Konstruktion surjektiv, (nutze $H_1 H_2 = H_2 H_1$), folgt mit Satz \ref{satz1_20}
		\[\faktor{H_2}{H_1 \cap H_2} = \faktor{H_2}{\ker(\varphi)} \cong \im(\varphi) = \faktor{H_1H_2}{H_1}.\]
		\item Nach Voraussetzung gilt $H_1 \unlhd H_2$. Die Faktorgruppe $H_2 / H_1$ wird zur Untergruppe von $G/H_1$. Es gilt sogar $H_2/H_1 \unlhd G/H_1$, da für alle $g \in G, h_2 \in H_2$ gilt:
		\[gH_1 h_2H_1 g^{-1} H_1 = \underbrace{gh_2g^{-1}}_{\in H_2, \text{ da } H_2 \unlhd G} H_1.\]
		Betrachte nun den Homomorphismus
		\[
		\begin{tikzcd}[column sep=huge, row sep=huge]
			G \arrow{r}{\varphi} \arrow[swap]{d}{\text{(kanonische Projektion)}\;\pi_1} & \faktor{\sfrac{G}{H_1}}{\sfrac{H_2}{H_1}} \\
			\faktor{G}{H_1} \arrow[swap]{ur}{\pi_2 \; \text{(kanonische Projektion)}}
		\end{tikzcd}
		\]
		

		mit $\varphi(g) = gH_1(H_2/H_1)$. Dann gilt
		\[\ker(\varphi) = \{g \in G \mid gH_1 \in \faktor{H_2}{H_1}\} = H_2.\]
		Da $\varphi$ nach Konstruktion surjektiv ist (als Komposition surjektiver Abbildungen), folgt wieder mit Satz \ref{satz1_20}:
		\[\faktor{G}{H_2} = \faktor{G}{\ker(\varphi)} \cong \im(\varphi) = \faktor{G/H_1}{H_2 /H_1}.\]
	\end{enumerate}
\end{proof}
\begin{beispiel}
	\begin{enumerate}[label=(\arabic*)]
		\item Sei $G = \Z$, $H_1 = 3\Z \unlhd G$ und $H_2 = 5\Z \unlhd G$. Dann ist $H_1 \cap H_2 = 15\Z$ und $H_1 + H_2 = \Z$, da $1 = 5(-1) + 3 \cdot 2.$ Satz \ref{satz1_21} (a) liefert
		\[\Z_3 = \faktor{\Z}{3\Z} = \faktor{H_1 + H_2}{H_1} \cong \faktor{H_2}{H_1 \cap H_2} = \faktor{5\Z}{15\Z}.\]
		\item Sei $G = \Z$, $H_1 = mn\Z \unlhd G$ und $H_2 = m\Z \unlhd G$ für $m,n \in \N$. Dann liefert Satz \ref{satz1_21} (b)
		\[\Z_m = \faktor{\Z}{m\Z} = \faktor{G}{H_2} \cong \faktor{G/H_1}{H_2/H_1} = \faktor{\Z/mn\Z}{m\Z/mn\Z}.\]
 	\end{enumerate}
\end{beispiel}

Wie sehen Untergruppen von Faktorgruppen aus? Im Beweis von Satz \ref{satz1_21} (b) haben wir verwendet, dass für eine Gruppe $G$ mit $N \unlhd G$ und $N \leq H \leq G$ gilt:
\[\faktor{H}{N} \leq \faktor{G}{N}.\]
Der Beweis zeigt zudem, dass $H \unlhd G \;\Leftrightarrow\; H/N \unlhd G/N$. 

\begin{leftbar}
	{Sind alle Untergruppen von $G/N$ von dieser Form?} -- \textbf{Ja!}
\end{leftbar}

\begin{satz}\label{satz1_23}
	Die Abbildung $\{H \leq G \mid N \leq H\} \to \{\text{Untergruppen von } G/N\}$ mit $H \mapsto H/N$ ist bijektiv.
\end{satz}
\begin{proof}
	Betrachte die kanonische Projektion $\pi \colon G/N$ mit $g \mapsto gN$. Ist $H' \leq G/N$ eine Untergruppe, so gilt 
	\[N \leq \pi^{-1}(H') = \{g \in G \mid \pi(g) \in H'\} \leq G\]
	\begin{itemize}
		\item $g \in N \;\Rightarrow\; \pi(g) = 1_{G/N} \in H'$, also $g \in \pi^{-1}(H')$. Insbesondere $1_G \in \pi^{-1}(H')$.
		\item $g_1, g_2 \in \pi^{-1}(H') \;\Rightarrow\; g_1g_2 \in \pi^{-1}(H')$, da $\pi(g_1g_2) = \underbrace{\pi(g_1)}_{\in H'}\underbrace{\pi(g_2)}_{\in H'} \in H'$.
		\item $g \in \pi^{-1}(H') \;\Rightarrow\; g^{-1} \in \pi^{-1}(H')$, da $\pi(g^{-1}) = \pi(g)^{-1} \in H'$.
	\end{itemize}
	Die Umkehrabbildung zur gegebenen Abbildung in der Aussage liefert die Zuordnung $H' \mapsto \pi^{-1}(H')$, da 
	\[\faktor{\pi^{-1}(H')}{N} = H' \quad\text{sowie}\quad \pi^{-1}(H/N) = H\]
\end{proof}

\begin{rem}\label{rem1_24}
	Die Bijektion in Satz \ref{satz1_23} ist inklusionserhaltend und zeigt, dass die kanonische Projektion $\pi \colon G \to G/N$ einen Isomorphismus von Untergruppenverbänden induziert:
	\[
	% https://tikzcd.yichuanshen.de/#N4Igdg9gJgpgziAXAbVABwnAlgFyxMJZARgBoAGAXVJADcBDAGwFcYkQBxEAX1PU1z5CKcqWLU6TVuwASPPiAzY8BImQBMEhizaIQAOXn9lQtWK1TdII4oErhydeZrbpem0sGqUAFmeSdWQB6Q15jLwcAVgoLQL0OEI87UxQANn9XKySTb2RozRdLdgAdYuBiAH1gBP1uUu5siKJo8UK4624JGCgAc3giUAAzACcIAFskURAcCCQyEAALGHoodkgwNjCQEfG5mhmkJ0Xl1b11zYUdicQpg8QAZhollbWCC6HR66O7x+OXs7eNiuk32s0Qfj+p3AgK2wMQ3zBEOeUPOQM+SAhd0isPRiGi0zB6Uhrw2aN2iCJdwA7E8TiT3ttcfi7gAOWn-aGknHktkEpA04kArmXXGUsEATnZKJhIp5oKQksFnM2lG4QA
	\begin{tikzcd}
		& G \arrow[ld, no head] \arrow[d, no head] \arrow[rd, no head] &               &  &                                             & \faktor{G}{N} \arrow[rd, no head] \arrow[d, no head]         &    \\
		H \arrow[rd, no head] & {}                                                           & {} \arrow[rr]{}{\pi} &  & \faktor{H}{N} \arrow[ru, no head] \arrow[rd, no head] & {}                                                 & {} \\
		& N \arrow[u, no head] \arrow[ru, no head]                     &               &  &                                             & \{1_{G/N}\} \arrow[ru, no head] \arrow[u, no head] &   
	\end{tikzcd}
	\]
\end{rem}
	
	\section{Endlich erzeugte Gruppen}

\begin{definition}
	Sei $G$ eine Gruppe und $S \subseteq G$ eine Teilmenge. Definiere 
	\[\langle S \rangle := \bigcap_{H \leq G, S \subseteq H} H \leq G.\]
	$\langle S \rangle$ heißt \textbf{die von $S$ erzeugte Untergruppe} von $G$.
\end{definition}
Falls $G = \langle S \rangle$, so heißt $S$ \textbf{Erzeugendensystem} von $G$. Hat $G$ ein endliches Erzeugendensystem, so heißt $G$ \textbf{endlich erzeugt}. Gibt es ein $g \in G$ mit $G = \langle \{g\} \rangle =: \langle g \rangle$, so heißt $G$ \textbf{zyklisch}.
\begin{rem}
	\begin{enumerate}[label=(\roman*)]
		\item Nach Konstruktion ist $\langle S\rangle$ die kleinste Untergruppe von $G$, die $S$ enthält.
		\item Für $S \neq \emptyset$ ist $\spn{S} = \{s_1 \cdots s_t \mid t \in \N, s_i \in S \cup S^{-1}\}$. 
		
		Insbesondere ist für $g \in G$:
		\[\spn{g} = \{g^n \mid n \in \Z\} \quad\text{mit}\quad g^n := \begin{cases}
			1_G, & n = 0\\
			\underbrace{g \cdots g}_{n-\text{mal}}, & n > 0\\
			\underbrace{g^{-1} \cdots g^{-1}}_{(-n)-\text{mal}}, & n < 0 
		\end{cases}\]
\end{enumerate}
\begin{leftbar}
	Wir wollen zunächst zyklische Gruppen besser verstehen!
\end{leftbar}
\end{rem}
\begin{beispiel}\label{beispiel2_3}
	\begin{enumerate}[label=(\arabic*)]
		\item $(\Z, +) = \spn{1} = \spn{-1}$ ist eine zyklische Gruppe.
		
		$(\Z_m, +) = \spn{\bar{1}}$ ist eine zyklische Gruppe der Ordnung $m$.
		
		\item Sei $G$ eine Gruppe mit $|G| = p$ Primzahl. Dann ist $G$ zyklisch.
		\begin{proof}
			Sei $1_G \neq g \in G$ und betrachte $\spn{g} \leq G$. Nach Satz \ref{satz1_11} $|\spn{g}|$ teilt $|G| = p$. Da $\spn{g} > 1$ nach Voraussetzung folgt $|\spn{g}| = p$. Somit $G = \spn{g}$.
		\end{proof}
	\end{enumerate}
\end{beispiel}
\begin{satz}\label{satz2_4}
	\begin{enumerate}[label=(\alph*)]
		\item Eine Gruppe $G$ ist zyklisch genau dann, wenn es einen surjektiven Gruppenhomomorphismus von der Form $\Z \to G$ gibt.
		\item Für eine zyklische Gruppe $G$ gilt:
		\[G \cong \begin{cases}
			\Z, & |G| = \infty,\\
			\Z_m, & |G| = m.
		\end{cases}\]
		Zudem ist jede Untergruppe von $G$ wieder zyklisch.
	\end{enumerate}
\end{satz}
\begin{proof}
	\begin{enumerate}[label=(\alph*)]
		\item \glqq{}$\Rightarrow$\grqq: Sei $G = \spn{g} = \{g^n \mid n \in \Z\}$. Definiere einen Gruppenhomomorphismus $\Z \to G$ durch $m \mapsto g^n$. Dieser ist nach Voraussetzung surjektiv.
		
		\glqq{}$\Leftarrow$\grqq: Sei $\varphi \colon \Z \to G$ ein surjektiver Gruppenhomomorphismus. Definiere $g := \varphi(1) \in G$.
		
		\underline{Behauptung}: $G = \spn{g}$.
		
		Die Inklusion $\spn{g} \subseteq G$ ist klar. Sei nun $h \in G$ beliebig. Da $\varphi$ surjektiv ist, existiert $n \in \Z$ mit $\varphi(n) = h$. Da $\varphi$ Gruppenhomomorphismus, gilt
		\[h = \varphi(n) = \begin{cases}
			\underbrace{\varphi(1) \cdots \varphi(1)}_{n-\text{mal}}, & n \geq 0\\
			\underbrace{\varphi(1)^{-1} \cdots \varphi(1)^{-1}}_{n-\text{mal}}, &n < 0
		\end{cases}\]
		Daraus folgt $h = g^n \in \spn{g}$.
		
		\item Sei $G$ zyklisch und $\varphi \colon \Z \to G$ ein surjektiver Gruppenhomomorphismus, der nach (a) existiert. Nach Satz \ref{satz1_20} gilt
		\[G \cong \faktor{\Z}{\ker(\varphi)}.\]
		Nach Aufgabe M.1.4. wissen wir, dass $\ker(\varphi) = m\Z$ für ein $m \in \N_0$. Damit folgt der erste Teil der Behauptung.
		
		Sei nun $H \leq G$. Dann ist $\varphi^{-1}(H)$ eine Untergruppe von $\Z$ (siehe Beweis zu Satz \ref{satz1_23}) und somit erneut $\varphi^{-1}(H) = m\Z, \; m \in\N_0$. Insbesondere ist $\varphi^{-1}(H) = \spn{m} \leq \Z$. Da $\varphi$ surjektiv ist, gilt $\varphi(\varphi^{-1}(H)) = H$ und $H$ wird von $\varphi(n)$ erzeugt.
	\end{enumerate}
\end{proof}
\begin{definition}
	Sei $G$ eine Gruppe und $g \in G$. Die \textbf{Ordnung von $g$} ist definiert als die Ordnung $\spn{g}$, der von $g$ erzeugten zyklischen Untergruppe von $G$.
	
	Wir schreiben $\ord(g)$ für die Ordnung von $g$. 
\end{definition}
\begin{rem}
	Ist $\ord(g) = m \in \N$ bzw. $\spn{g} \cong \Z_m$ mit $\spn{g} = \{1_G, g, \dots, g^{m-1}\}$ nach Satz \ref{satz2_4} (b), so gilt $g^n = 1_G$ genau dann, wenn $n \in m\Z$. 
	
	Ist $G$ endlich, so gilt $\ord(g)$ teilt $|G|$ nach Satz \ref{satz1_11} und somit $g^{|G|} = 1_G$ (\textbf{Kleiner Fermat'scher Satz}).
	
	Ist $\ord(g) = \infty$ bzw. $\spn{g} \cong \Z$, so sind die $g^n$ mit $n \in \Z$ paarweise verschieden.
\end{rem}
\begin{beispiel}
	\begin{enumerate}[label=(\arabic*)]
		\item Für $\bar{a} \in \Z_m$ mit $m \in \N$ gilt $\ord(\bar{a}) = \frac{m}{\ggt(a,m)}$. Zum Beispiel hat $\bar{8} \in \Z_{12}$ die Ordnung $\frac{12}{\ggt(8,12)} = \frac{12}{4} = 3$.
		
		\item Für $n \geq 3$ sei $D_n$ die Symmetriegruppe eines regelmäßigen $n$-Ecks in $\R^2$. Diese heißt auch \textbf{Diedergruppe}. Für $n = 3$ gilt $D_3 \cong S_3$.
		
		Im Allgemeinen enthält $D_n$ genau $n$ Drehungen und $n$ Spiegelungen, so dass $|D_n| = 2n$. Sei $r$ eine Drehung um $\frac{2\pi}{n}$ und $s$ eine beliebige Spiegelung. Dann gilt $\ord(r) = n$ und $\ord(s) = 2$, sowie $D_n = \{\id, r, r^2, \dots, r^{n-1}, s, sr, sr^2, \dots, sr^{n-1}\} = \spn{\{r,s\}}$.
	\end{enumerate}
\end{beispiel}

\begin{center}
	\textit{Welche Gruppen können wir aus zyklischen Gruppen zusammensetzen?}
\end{center}

\begin{definition}
	Eine Gruppe $G$ heißt \textbf{inneres direktes Produkt} von $G_1$ und $G_2$, falls
	\begin{itemize}
		\item $G_1, G_2 \unlhd G$.
		\item $G_1 \cdot G_2 = G$.
		\item $G_1 \cap G_2 = \{1_G\}$.
	\end{itemize}
\end{definition}

\begin{rem}
	\begin{enumerate}[label=(\roman*)]\label{rem2_9}
		\item Ist $G = G_1 \times G_2$ (äußeres) direktes Produkt der Gruppen $G_1$ und $G_2$, so ist $G$ inneres direktes Produkt von $G_1 \times \{1_G\}$ und $\{1_G\} \times G_2$.
		
		\item Ist $G$ inneres direktes Produkt von $G_1$ und $G_2$, so gilt
		\[G \cong G_1 \times G_2.\]
		\begin{proof}
			Betrachte die Abbildung $\varphi \colon G_1 \times G_2$ mit $(g_1, g_2) \mapsto g_1g_2$. Da $\underbrace{g_1g_2g_1^{-1}}_{\in G_2}  g_2^{-1} = g_1(\underbrace{g_2g_1^{-1}g_2^{-1}}_{\in G_1}) \in G_1 \cap G_2 = \{1_G\}$, folgt $g_1 = g_2$. Somit gilt $\varphi((g_1, g_2) (h_1, h_2)) = \varphi(g_1h_1, g_2h_2) = g_1(h_1g_2)h_2 = (g_1g_2)(h_1h_2) = \varphi((g_1, g_2)) \varphi((h_1, h_2))$. $\varphi$ ist zudem bijektiv nach Voraussetzung.
		\end{proof}
	\end{enumerate}
\end{rem}
\begin{beispiel}
	\begin{enumerate}[label=(\arabic*)]
		\item Die abelsche Gruppe $\C \setminus \{0\}$ ist inneres direktes Produkt von $\R_{>0}$ und $S^1 := \{z \in \C \mid |z| = 1\}$.
		\item Nach Aufgabe S.2.2. gilt 
		\[V_4 = \{\id, (12)(34), (13)(24), (14)(23)\} \unlhd S_4\]
		mit $S_4/V_4 \cong S_3$. Die $S_3$ ist isomorph zu einer Untergruppe $H$ von $S_4$ (Beispiel \ref{beispiel1_6} (4)), so dass $V_4 \cdot H = S_4$ und $V_4 \cap H = \{\id\}$. Aber $S_4$ ist nicht inneres direktes Produkt von $V_4$ und $H$, da $H$ kein Normalteiler von $S_4$.
	\end{enumerate}
\end{beispiel}
\begin{thm}\label{thm2_11}
	Jede endlich erzeugte abelsche Gruppe ist ein endliches inneres direktes Produkt zyklischer Gruppen.
\end{thm}
\begin{proof}
	Sei $(G, +)$ abelsche Gruppe, erzeugt von $S = \{a_1, \dots, a_k\} \subseteq G$. 
	
	\textbf{Induktion über $k$:} Für $k = 1$ ist $G$ zyklisch.
	
	Betrachte den surjektiven Gruppenhomomorphismus
	\[\varphi_S \colon \Z^k \to G \quad\text{mit}\quad(n_1, \dots, n_k) \mapsto n_1a_1 + \dots + n_ka_k.\]
	Sei $\pi \colon \Z^k \to \Z$ die Projektion auf die erste Komponente, also
	\[\pi((n_1, \dots, n_k)) = n_1.\]
	Das Bild von $\ker(\varphi_S)$ unter $\pi$ ist Untergruppe von $\Z$ und somit von der Form $d\Z$ für $d \in \N_0$. Sei o.B.d.A. $S$ so gewählt, dass $d$ minimal ist. Falls $d = 0$, so ist $\spn{a_1} \cap \spn{S\setminus\{a_1\}}= \{0_G\}$ und $\ord{a_1} = \infty$, d.h. $G$ ist inneres direktes Produkt von $\spn{a_1}$ und $\spn{S\setminus\{a_1\}}$, wobei $\spn{a_1} \cong \Z$.
	
	Auf $S\setminus\{a_1\}$ können wir die Induktionsvoraussetzung anwenden. 
	
	Falls $d > 0$, wähle $(n_1, \dots, n_k) \in \ker(\varphi_S)$ mit $n_1 = d$. Sei $2 \leq i \leq k$. Division mit Rest liefert $q_i, d_i \in \Z$ mit
	\[n_i = q_i d + d_i \quad\text{und}\quad 0 \leq d_i < d.\]
	Definiere $S_i := \{b_1, \dots, b_k\}$ durch $b_1 = a_i$ und $b_i = a_1 + q_i a_i$ und $b_j = a_j$ sonst. Dann ist auch $S_i$ Erzeugendensystem von $G$. Zudem liegt der Vektor
	\[(d_i, n_2, \dots, i, \dots, n_k) \in \Z^k\]
	im Kern von $\varphi_{S_i}$, da
	\[d_ib_1 + db_i = (n_i - q_id)a_i + d(a_1 + q_ia_i) = n_ia_i + n_1a_1\]
	und weil $(n_1, \dots, n_k) \in \ker(\varphi_S)$. Wegen der Minimalität von $d$ ist $1 \leq d_i < d$ ausgeschlossen, d.h.
	\[d_i = 0 \quad\text{und}\quad d \big| n_i.\]
	Setze nun $x_i = \frac{n_i}{d}$ für $1 \leq i \leq k$. Insbesondere $x_1 = 1$. Dann wird $G$ von der Menge
	\[\bigg\{\underbrace{\sum_{i=1}^k x_ia_i}_{=: a} , a_2, \dots, a_k\bigg\}\]
	mit $\spn{a} \cap \spn{\{a_2, \dots, a_k\}} = \{0_G\}$. Denn ein Element im Schnitt hat die Form 
	\[m_1a = \sum_{i=2}^k m_ia_i\]
	für $(m_1, \dots, m_k) \in \Z^k$, so dass $m_1$ im Bild von $\ker(\varphi_S)$ unter $\pi$ liegt, also ein Vielfaches von $d$ ist. Es gilt aber bereits
	\[da = n_1 a_1 + \dots + n_ka_k = 0_G.\]
	Somit ist $G$ inneres direktes Produkt von $\spn{a}$ und $\spn{S \setminus \{a_1\}}$, wobei  $\spn{a} \cong \Z_d$. Nun können wir erneut die Induktionsvoraussetzung anwenden.
\end{proof}
\begin{kor}[Hauptsatz für endlich erzeugte abelsche Gruppen]\label{kor2_12}
	Sei $G$ eine endlich erzeugte Gruppe. Dann existieren eindeutige $r,t \in \N_0$ sowie bis auf Reihenfolge eindeutig bestimmte Primzahlpotenzen 
	\[1 < p_1^{k_1} \leq \dots \leq p_t^{k_t} \quad\text{mit}\quad G \cong \Z^r \times \Z_{p_1^{k_1}} \times \dots \times \Z_{p_t^{k_t}}.\]
\end{kor}
\begin{proof}
	Die Existenz folgt aus Theorem \ref{thm2_11} zusammen mit Bemerkung \ref{rem2_9} (ii) und Aufgabe M.3.3. Letztere besagt, dass für $m, n \in \N$ mit $\mathrm{ggT}(m,n) = 1$ gilt $\Z_{nm} \cong \Z_n \times \Z_m$. Die Eindeutigkeit können/wollen wir hier nicht beweisen.
\end{proof}
Was können wir im nicht-abelschen Fall tun? Eine Klassifikation aller endlich erzeugten oder endlichen Gruppen ist hoffnungslos. Im Jahr 1982 hat man die Klassifikation aller endlichen einfachen Gruppen abgeschlossen, also aller endlichen Gruppen mit genau zwei (trivialen) Normalteilern. Dies war ein Mammutprojekt!
\begin{itemize}
	\item mehr als 500 Fachartikel,
	\item mehr als 100 MathematikerInnen,
	\item Zeitraum von über 50 Jahren,
	\item Einsatz von Computern.
\end{itemize}
Die endlichen einfachen Gruppen sind von der Form
\begin{enumerate}[label=(\arabic*)]
	\item Zyklische Gruppe $\Z_p$ mit $p$ Primzahl.
	\item Alternierende Gruppen $A_n$ für $n \geq 5$.
	\item Endliche Gruppen vom Lie-Typ, z.B.
	\[\mathsf{PSL}_(K) := \faktor{\SL_n(K)}{Z(\SL_n(K))} = \faktor{\SL_n(K)}{\{\lambda E_n \mid \lambda^n = 1\}}\]
	für $n > 2$ und einen endlichen Körper $K$.
	\item 26 sporadische Gruppen mit bis zu ungefähr $8 \cdot 10^{53}$ Elementen, die sogenannten Monster!
\end{enumerate}
\begin{beispiel}\label{beispiel2_13}
	Die alternierende Gruppe $A_4$ ist nicht einfach nach Aufgabe S.2.2. Aber für $n \geq 5$ ist $A_n$ einfach.
\end{beispiel}
\begin{proof}
	Sei $\{\id\} \neq N \unlhd A_n$. Wir zeigen, dass $N = A_n$. 
	
	\textbf{Schritt 1: } $N$ enthält einen Zykel der Länge 3. 
	
	Sei $\id \neq \sigma \in N$. Ist $\sigma$ kein Zykel der Länge 3, so gilt einer der folgenden Fälle:
	\begin{enumerate}[label=(\roman*)]
		\item $\sigma = (a_1 a_2 a_3 a_4 \dots)\dots$
		\item $\sigma = (a_1 a_2 a_3)(a_4 a_5 a_6)\dots$
		\item $\sigma = (a_1 a_2)(a_3 a_4)(a_5 a_6)\dots$
		\item $\sigma = (a_1 a_2)(a_3 a_4)$
	\end{enumerate}
	Da $N$ Normalteiler ist, gilt $(\pi \sigma \pi^{-1})\sigma^{-1} \in N$ für alle $\pi \in A_n$. Im Fall (i) wähle $\pi = (a_2 a_1 a_3)$.
	\[(\pi \sigma \pi^{-1})\sigma^{-1} = (a_2 a_1 a_3)\sigma(a_1 a_2 a_3)\sigma^{-1} = (a_1 a_3 a_4).\]
	Im Fall (ii) wähle $\pi = (a_3 a_2 a_4)$.
	\[(\pi\sigma\pi^{-1})\sigma^{-1} = (a_3 a_2 a_4)\sigma(a_2a_3a_4)\sigma^{-1} = (a_1 a_5 a_2 a_4 a_3)\]
	und weiter im Fall (i). Im Fall (iii) wähle $\pi = (a_2 a_1 a_3)$:
	\[(\pi\sigma\pi^{-1}) \sigma^{-1} = (a_2 a_1 a_3)\sigma(a_1 a_2 a_3) \sigma^{-1} = (a_1 a_4)(a_2 a_3).\]
	Im Fall (iv) wähle $\pi = (a_2 a_1 a_5)$:
	\[(\pi\sigma\pi^{-1})\sigma^{-1} = (a_2 a_1 a_5)\sigma(a_1 a_2 a_5)\sigma^{-1} = (a_1 a_2 a_5).\]
	Also enthält $N$ einen Zykel der Länge 3.
	
	\textbf{Schritt 2: } $N = A_n$.
	
	Sei $(a_1 a_2 a_3) \in N$. Da $N \unlhd A_n$ ist, gilt:
	\[(a_3 a_4 a_5)(a_1 a_2 a_3)(a_4 a_3 a_5) = (a_1 a_2 a_4) \in N.\]
	Insbesondere sind alle Zykel der Form $(a_1 a_2 x)$ in $N$ mit $x \in \{1,\dots, n\} \setminus \{a_1, a_2\}$. Wiederholen des Arguments zeigt, dass alle Zykel der Länge 3 in $N$ enthalten sind. Da $A_n$ nach Aufgabe M.3.2. von diesen Zykeln erzeugt wird, folgt $N = A_n$.
\end{proof}
\begin{leftbar}
	{Warum sind endliche einfache Gruppen so wichtig?}
\end{leftbar}

\begin{definition}
	Sei $G$ ein Gruppe. Eine Folge von Untergruppen
	\[\{1_G\} = G_0 \unlhd G_1 \unlhd G_2 \unlhd \dots \unlhd G_n = G\]
	heißt \textbf{Normalreihe der Länge $n$}. Die Faktoren $G_i/G_{i-1}$ heißen \textbf{Faktoren} der Normalreihe. Eine Normalreihe von $G$ heißt \textbf{Kompositionsreihe}, falls alle ihre Faktoren einfach sind. Die Faktoren einer Kompositionsreihe heißen \textbf{Kompositionsfaktoren}.
\end{definition}
\begin{rem}\label{rem2_15}
	\begin{enumerate}[label=(\roman*)]
		\item Jede Gruppe $G$ hat die Normalreihe $\{1_G\} \unlhd G$.
		\item Jede endliche Gruppe $G$ hat eine Kompositionsreihe.
		
		Induktion nach $|G|$. Ist $G$ einfach, dann ist $\{1_G\} \unlhd G$ Kompositionsreihe. Ist andernfalls $N \unlhd G$ maximaler echter Normalteiler. nach Satz \ref{satz1_23} ist $G/N$ einfach und $N$ hat nach Induktionsvoraussetzung eine Kompositionsreihe
		\[\{1_G\} = N_0 \unlhd N_1 \unlhd \dots \unlhd N_t = N.\]
		Dann ist $\{1_G\} = N_0 \unlhd N_1 \unlhd \dots \unlhd N_t = N \unlhd G$ eine Kompositionsreihe von $G$.
		
		\item Die Gruppe $\Z$ hat keine Kompositionsreihe (nutze Klassifikation der Untergruppen von $\Z$).
	\end{enumerate}
\end{rem}
\begin{thm}\label{Satz von Jordan-Hölder}
	Sei $G$ eine endliche Gruppe. Dann alle Kompositionsreihen von $G$ \textbf{äquivalent}, das heißt sie haben die selbe Länge und bis auf Isomorphie und Reihenfolge die selben Kompositionsfaktoren.
\end{thm}
\begin{proof}
	Betrachte die folgenden Kompositionsreihen:
	\begin{enumerate}[label=(\Roman*)]
		\item $\{1_G\} = G_0 \unlhd G_1 \unlhd \dots \unlhd G_r = G$,
		\item $\{1_G\} = H_0 \unlhd H_1 \unlhd \dots \unlhd H_s = G$.
	\end{enumerate}
	\textbf{Induktion nach $r$: } Für $r = 1$ ist $G$ einfach. Also auch $G = H_1$.
	
	Sei $r > 1$. 
	
	\textbf{Fall 1: } $G_{r-1} = H_{s-1}$. Dann hat $G_{r-1}$ die Kompositionsreihen $\{1_G\} = G_0 \unlhd \dots \unlhd G_{r-1}$ und $\{1_G\} = H_0 \unlhd \dots \unlhd H_{s-1} = G_{r-1}$. Nach Induktionsvoraussetzung sind diese beiden und somit auch die ursprünglichen beiden Kompositionsreihen \textbf{(I)} und \textbf{(II)} von $G$ äquivalent.
	
	\textbf{Fall 2: } $G_{r-1} \neq H_{s-1}$. Betrachte $G_{r-1} \unlhd G_{r-1}\dots H_{s-1} \unlhd G$. Da $G_{r-1} \neq H_{s-1}$ und $G_r / G_{r-1}$ und $G/H_{s-1}$ einfach, folgt mit Satz \ref{satz1_23}, dass $G_{r-1}H_{s-1} = G$. Sei $J = G_{r-1} \cap H_{s-1}$ mit $J \unlhd G$. Nach Satz \ref{satz1_21} (a) gilt
	\[\faktor{G}{G_{r-1}} = \faktor{G_{r-1}H_{s-1}}{G_{r-1}} \cong \faktor{H_{s-1}}{J}\]
	sowie 
	\[\faktor{G}{H_{s-1}} = \faktor{H_{s-1} G_{r-1}}{H_{s-1}} \cong \faktor{G_{r-1}}{J}\]
	d.h. $H_{s-1} / J$ und $G_{r-1}/J$ sind einfach. Nach Bemerkung \ref{rem2_15} (ii) hat $J$ eine Kompositionsreihe
	\[\{1_G\} = J_0 \unlhd J_1 \unlhd \dots \unlhd J_t = J.\]
	Diese induziert die folgenden beiden Kompositionsreihen von $G$:
	\begin{enumerate}[start=3,label=(\Roman*)]
		\item $\{1_G\} = J_0 \unlhd \dots \unlhd J_t = J \unlhd G_{r-1} \unlhd G$,
		\item $\{1_G\} = J_0 \unlhd \dots \unlhd J_t = J \unlhd H_{s-1} \unlhd G$.
	\end{enumerate}
	Diese sind äquivalent, da bis auf Isomorphie nur die letzten beiden Faktoren vertauscht werden. Nach Induktionsvoraussetzung sind auch die Kompositionsreihen \textbf{(I)} und \textbf{(III)} von $G$ äquivalent. Insbesondere gilt $r-1 = t+1$. Somit liefert \textbf{(IV)} eine Kompositionsreihe von $H_{s-1}$ der Länge $r-1$, die nach Induktionsvoraussetzung äquivalent ist zu $\{1_G\} = H_0 \unlhd \dots \unlhd H_{s-1}$. Folglich sind auch die Kompositionsreihen \textbf{(II)} und \textbf{(IV)} äquivalent. Dies liefert die gewünschte Äquivalenz von \textbf{(I)} und \textbf{(II)}.
\end{proof}
\begin{beispiel}
	$\Z_6$ hat die Kompositionsreihen $\{\bar{0}\} \unlhd \spn{\bar{2}} \unlhd \Z_6$ und $\{\bar{0}\} \unlhd \spn{\bar{3}} \leq \Z_6$ mit Kompositionsfaktoren isomorph zu $\Z_2$ und $\Z_3$.
	
	Die symmetrische Gruppe $S_3$ hat die Kompositionsreihe $\{\id\} \unlhd A_3 \unlhd S_3$, deren Kompositionsfaktoren auch isomorph zu $\Z_2$ und $\Z_3$ sind. Aber $\Z_6 \cong \Z_2 \times \Z_3 \not\cong S_3$!
\end{beispiel}
% Vorlesung vom 11.11.2024:
Wir haben gesehen, dass sich endliche abelsche Gruppen, auf im Wesentlichen, eindeutige Art und Weise, aus einfachen Gruppen zusammenkleben lassen. Aber welche Gruppen können wir aus einfachen zyklischen Gruppen zusammenkleben?
\begin{definition}
	Eine Gruppe $G$ heißt \textbf{auflösbar}, wenn $G$ eine Normalreihe mit abelschen Faktoren hat, d.h. es gibt eine Folge von Untergruppen
	\[\{1_G\} = G_0 \unlhd G_1 \unlhd \dots \unlhd G_n = G\]
	mit $G_i/G_{i-1}$ abelsch für alle $1 \leq i \leq n$.
\end{definition}
\begin{rem}
	Auflösbare Gruppen werden uns in der Algebra II helfen zu entscheiden, ob Gleichungen durch endliche Wurzelausdrücke aufgelöst werden können (siehe Kapitel 0).
\end{rem}
\begin{beispiel}\label{beispiel2_20}
	\begin{enumerate}[label=(\arabic*)]
		\item Abelsche Gruppen sind auflösbar. Nach Theorem \ref{thm2_11} bzw. \ref{kor2_12} wissen wir, dass wir endliche abelsche Gruppen aus einfachen zyklischen Gruppen zusammenkleben können.
		\item Jede einfache auflösbare Gruppe $G$ ist isomorph zu $\Z_p$, $p$ prim.
		\begin{proof}
			Da $G$ einfach, existiert nur die triviale Normalreihe $\{1_G\} \unlhd G$. Da $G$ auflösbar ist, folgt, dass $G$ abelsch ist und abelsche Gruppen ohne echten Normalteiler sind isomorph zu $\Z_p$.
		\end{proof}
		\item Die alternierende Gruppe $A_n$ mit $n \in \N$ ist auflösbar genau dann, wenn $n \leq 4$.
		\begin{proof}
			Für $n \leq 3$ ist $A_n$ abelsch und dadurch auflösbar. Für $n = 4$ betrachte die Normalreihe $\{1_G\} \unlhd V_4 \unlhd A_4$ mit Faktoren $V_4 \cong \Z_2 \times \Z_2$ und $A_4 / V_4 \cong \Z_3$ (siehe Aufgabe S.2.2.). Für $n \geq 5$ nutze Beispiel \ref{beispiel2_13}.
		\end{proof}
		\item Man kann zeigen, dass endliche Gruppen ungerader Ordnung auflösbar stets auflösbar sind. (\textbf{Satz von Feit-Thompson} (1963))
		\item Die kleinste nicht auflösbare Gruppe ist die $A_5$.
	\end{enumerate}
\end{beispiel}
\begin{prop}\label{prop2_21}
	Sei $G$ eine auflösbare Gruppe.
	\begin{enumerate}[label=(\alph*)]
		\item Jede Untergruppe von $H \leq G$ ist auflösbar.
		\item Ist $N \unlhd G$ Normalteiler, so ist auch $G/N$ auflösbar. 
	\end{enumerate}
\end{prop}
\begin{proof}
	Sei $\{1_G\} = G_0 \unlhd G_1 \unlhd \dots \unlhd G_n = G$ eine Normalreihe von $G$ mit der Eigenschaft $G_i/G_{i-1}$ abelsch für alle $i \in \{1, \dots, n\}$.
	\begin{enumerate}[label=(\alph*)]
		\item Betrachte die induzierte Normalreihe der Form
		\[\{1_G\} = G_0 \cap H \unlhd G_1 \cap H \unlhd \dots \unlhd G_n \cap H = H.\]
		Nach Satz \ref{satz1_21} (a) gilt für alle $i \in \{1, \dots, n\}$
		\[\faktor{G_i \cap H}{G_{i-1} \cap H} = \faktor{G_i \cap H}{G_{i-1} \cap G_i \cap H} \cong \faktor{G_{i-1} \cdot (G_i \cap H)}{G_{i-1}} \leq \faktor{G_i}{G_{i-1}}\]
		Da $G_i / G_{i-1}$ abelsch ist, so ist auch $(G_i \cap H) / (G_{i-1}\cap H)$ abelsch. Somit ist $H$ auflösbar.
		
		\item Betrachte die durch die kanonische Projektion $\pi \colon G \to G/N$ induzierte Normalreihe der Form $\{1_{G/N}\} = \pi(G_0) \unlhd \pi(G_1) \unlhd \dots \unlhd \pi(G_n) = G/N$. Für $i \in \{1, \dots, n\}$ erhalten wir einen surjektiven Gruppenhomomorphismus 
		\[\varphi \colon G_i \to \faktor{\pi(G_i)}{\pi(G_{i-1})}\]
		durch $g_i \mapsto \pi(g_i)\pi(G_{i-1})$ mit 
		\[G_{i-1} \unlhd \ker(\varphi) \unlhd G_i.\]
		Nach Satz \ref{satz1_20} und Satz \ref{satz1_21} (b) gilt
		\[\faktor{\pi(G_i)}{\pi(G_{i-1})} \cong \faktor{G_i}{\ker(\varphi)} \cong \faktor{G_i / G_{i-1}}{\ker(\varphi) / G_{i-1}}.\]
		Da $G_i / G_{i-1}$ abelsch ist, so sind auch deren Quotienten abelsch und somit insbesondere $\pi(G_i) / \pi(G_{i-1})$. Also ist $G/N$ auflösbar.
	\end{enumerate}
\end{proof}
\begin{satz}\label{satz2_22}
	Sei $G$ eine Gruppe mit Kompositionsreihe. Dann ist $G$ auflösbar genau dann, wenn alle Kompositionsfaktoren von $G$ isomorph zu $\Z_p$ für $p$ Primzahl.
\end{satz}
\begin{proof}
	\glqq{}$\Leftarrow$\grqq: Folgt unmittelbar, da $\Z_p$ abelsch ist.
	
	\glqq{}$\Rightarrow$\grqq: Sei $G_i / G_{i-1}$ ein Kompositionsfaktor von $G$ mit $G_{i-1} \unlhd G_i \leq G$. Da $G$ auflösbar ist, sind nach Proposition \ref{prop2_21} sowohl $G_i$ als auch $G_i / G_{i-1}$ auflösbar. Da $G_i / G_{i-1}$ zudem einfach ist, folgt die Behauptung mit Beispiel \ref{beispiel2_20} (2).
\end{proof}





	
	\section{Operationen von Gruppen auf Mengen}
\begin{definition}
	Sei $G$ eine Gruppe und $X \neq \emptyset$ eine Menge. Dann heißt $X$ \textbf{$G$-Menge}, wenn es eine Abbildung $* \colon G \times X \to X$, $(g,x) \mapsto g*x$ gibt mit 
		\begin{enumerate}[label={\bfseries(O\arabic*)}]
		\item $1_G * x = x$ für alle $x \in X$. (Das neutrale Element operiert neutral)
		\item $g * (h * x) = (gh)*x$ für alle $g, h \in G$ und alle $x \in X$.
	\end{enumerate} 
	Wir sagen \textbf{$G$ operiert auf $X$} und schreiben oft $\cdot$ statt $*$.
\end{definition}
\begin{rem}\label{rem3_2}
	Sei $X$ eine $G$-Menge und $g \in G$. Dann ist $\tau_g \colon X \to X$ mit $x \mapsto g\cdot x$ bijektiv mit Inverse $\tau_{g{-1}}$. Also ist $\tau_g$ Element der symmetrischen Gruppe $S_X$. Die Abbildung $\tau \colon G \to S_X$ mit $g \mapsto \tau_g$ ist Gruppenhomomorphismus, da für alle $g, h \in G$ und $x \in X$ gilt:
	\[\tau(gh)(x) = \tau_{gh}(x) = (gh) x = g(hx) = \tau_g(\tau_h(x)) = \tau(g) \circ \tau_h(x) = \tau(g) \circ \tau(h)(x)\]
	Umgekehrt macht jeder Gruppenhomomorphismus $\varphi \colon G \to S_X$ mit $g \mapsto \varphi_g$ $X$ zu einer $G$-Menge durch $G \times X \to X$ mit $(g,x) \mapsto \varphi_g(x)$. Denn $1_G \cdot x = \varphi_{1_G}(x) = \id(x) = x$ für alle $x \in X$ und 
	\[g(hx) = \varphi_g(\varphi_h(x)) = (\varphi_g \circ \varphi_h)(x) = \varphi_{gh}(x) = (gh)x\]
	für alle $g,h \in G$ und $x \in X$.
\end{rem}

Ist die Abbildung $\tau$ injektiv bzw. ist $g = 1_G$ das einzige Element aus $G$ mit $gx = x$ für alle $x \in X$, so heißt die Operation von $G$ auf $X$ \textbf{treu}.

\begin{beispiel}\label{beispiel3_3}
	\begin{enumerate}[label=(\arabic*)]
		\item $G$ operiert auf sich selbst durch Linksmultiplikation. Sei $X = G$ und $G \times X \to X$ mit $(g,x) \mapsto gx$. \textbf{(O1)} und \textbf{(O2)} sind erfüllt, da $G$ eine Gruppe ist. Die Operation ist treu $\rightsquigarrow$ siehe Satz von Cayley (Satz \ref{satz1_7}).
		
		\item Sei $H \subseteq G$. Dann operiert $G$ auf $G/H$ durch $G \times (G/H) \to G/H$ mit $(g,xH) \mapsto (gx)H$. Diese Operation ist im Allgemeinen nicht treu, da für $g \in G$ gilt $(gx)H = xH$ für alle $x \in G$ genau dann, wenn $x^{-1} g x \in H$ für alle $x \in G$, was genau dann der Fall ist, wenn $g \in x H x^{-1}$ für alle $x \in G$. Für $H \unlhd G$ zum Beispiel gilt $xHx^{-1} = H$ für alle $x \in G$.
		
		\item Betrachte das Quadrat mit Eckpunkten $v_1, \dots, v_4$. Sei $G = \Z_4$ und $X = \{v_1, v_2, v_3, v_4\}$. Dann operiert $G$ treu auf $X$ durch Drehung, zum Beispiel
		\begin{align*}
			(\bar{1}, v_1) &\mapsto v_2\\
			(\bar{1}, v_2) &\mapsto v_3\\
			(\bar{1}, v_3) &\mapsto v_4\\
			(\bar{1}, v_4) &\mapsto v_1.
		\end{align*}
		Mit Hilfe von \textbf{(O1)} und \textbf{(O2)} legt dies die gewünschte Abbildung $G \times X \to X$ fest.
	\end{enumerate}
\end{beispiel}
\begin{definition}
	Sei $X$ eine $G$-Menge. Für $x \in X$ heißt 
	\begin{enumerate}[label=(\alph*)]
		\item $Gx := \{gx \mid g \in G\}$ die \textbf{Bahn von $x$ unter $G$}. 
		
		Die Operation heißt \textbf{transitiv}, falls die Menge $X$ unter $G$ nur eine Bahn besitzt, das heißt für alle $x,y \in X$ existiert $g \in G$ mit $gx = y$.
		\item $\stab_G(x) := \{g \in G \mid gx = x\}$ der \textbf{Stabilisator von $x$ in $G$}.
		
		Mit $\stab_G(x) = G$, das heißt $gx = x$ für alle $g \in G$, so heißt $x$ \textbf{Fixpunkt der Operation}. Schreibe $X^G$ für die Menge aller Fixpunkte der Operation.
	\end{enumerate}
\end{definition}
\begin{rem}\label{rem3_5}
	Sei $X$ eine $G$-Menge.
	\begin{enumerate}[label=(\roman*)]
		\item Definiere auf $X$ eine Äquivalenzrelation $x \sim y \;:\Leftrightarrow\; \exists g \in G : gx = y$. Als Äquivalenzklassen erhalten wir genau die Bahnen unter $G$. Für $x \in X$ gilt
		\[[x] = \{y \in X \mid x \sim y\} = \{y \in X \mid \exists g \in G : gx = y\} = \{gx \mid g \in G\} = Gx.\]
		Insbesondere ist $X$ die disjunkte Vereinigung von Bahnen
		
		\item Für $x in X$ ist $\stab_G(x) \leq G$, da
		\begin{align*}
			1_G &\in \stab_G(x)\\
			\forall g,h \in \stab_G(x) &: (gh)x = g(hx) = gx = x\\
			\forall g \in \stab_G(x) &: g^{-1}x = g^{-1}(gx) = (g^{-1}g)x = 1_G x = x.
		\end{align*}
		Es gilt für $a \in G$, dass $\stab_G(ax) = a\stab_G(x) a^{-1}$, da $g \in \stab_G(ax)$ genau dann, wenn $g(ax) = ax$ genau dann, wenn $(a^{-1}ga)x = x$ genau dann, wenn $a^{-1}ga \in \stab_G(x)$. Äquivalent zu $g \in a\stab_G(x) a^{-1}$.
	\end{enumerate}
\end{rem}
\begin{beispiel}\label{beispiel3_6}
	\begin{enumerate}[label=(\arabic*)]
		\item Die Operationen in Beispiel \ref{beispiel3_3} sind alle transitiv und, abgesehen vom trivialen Fall, fixpunktfrei.
		
		Im Beispiel \ref{beispiel3_3} (1) und (3) sind die Stabilisatoren trivial, also $\stab_G(x) = \{1_G\}$ für alle $x \in X$. In Beispiel \ref{beispiel3_3} (2) erhalten wir als Stabilisatoren die zu $H$ konjugierten Untergruppen.
		
		\item $G$ operiert auf sich selbst durch Konjugation. Sei dazu $X=G$ und $G \times X \to X$ mit $(g,x) \mapsto gxg^{-1}$. Diese Operation ist im Allgemeinen weder treu noch transitiv. Die Bahn $Gx = \{gxg^{-1} \mid g \in G\} =: C_x$ heißt \textbf{Konjugationsklasse von $x$}. Der Stabilisator $\stab_G(x) = \{g \in G \mid gx = xg\} =: C_G$ heißt \textbf{Zentralisator von $x$ in $G$}. Das Zentrum von $G$ entspricht genau der Menge $X^G$ bzw. der Vereinigung aller 1-elementigen Konjugationsklassen.
		
		Sei $G = \GL_n(K)$ für $n \in \N$ und einen Körper $K$. Dann enthält die Konjugationsklasse $C_{\mat{A}}$ einer Matrix $\mat{A} \in \GL_n(K)$ genau die zu $\mat{A}$ ähnlichen Matrizen in $\GL_n(K)$.
		
		Fixpunkte der Operation sind genau die Matrizen 
		\[\begin{bmatrix}
			\lambda & & \\
			& \ddots & \\
			& & \lambda
		\end{bmatrix}\]
		mit $\lambda \in K \setminus \{0\}$.
	\end{enumerate}
\end{beispiel}
\begin{satz}[Bahnensatz]\label{satz3_7}
	Sei $X$ eine $G$-Menge und $x \in X$. Dann erhalten wir die bijektive Abbildung 
	\[Gx \to \faktor{G}{\stab_G(x)}\quad\text{mit}\quad gx \mapsto g \stab_G(x).\]
	Ist $G$ endlich, so folgt $|G| = |Gx| \cdot |\stab_G(x)|$, die sogenannte \textbf{Bahnformel}.
\end{satz}
\begin{proof}
	Die Zuordnung ist wohldefiniert und injektiv, da für alle $g, h \in G$ gilt: 
	\[gx = hx \;\Leftrightarrow\; x=g^{-1}hx \;\Leftrightarrow\; g^{-1}h \in \stab_G(x) \;\Leftrightarrow\; g \stab_G(x) = h \stab_G(x)\]
	Zudem ist die Abbildung surjektiv nach Konstruktion. Die Bahnformel folgt mit Satz \ref{satz1_11}.
\end{proof}
\begin{kor}\label{kor3_8}
	Sei $X$ eine endliche $G$-Menge und $\{x_i\}_{i \in I}$ ein Repräsentantensystem der Bahnen von $X$ unter $G$. Dann gilt
	\[|X| \overset{\text{Bem.} \ref{rem3_5} (i)}{=}  \sum_{i \in I} |G{x_i}| = |X^G| + \sum_{x_i \notin X^G} |G{x_i}| \overset{\text{Satz.} \ref{satz3_7}}{=} |X^G| + \sum_{x_i \notin X^G} [G: \stab_G(x_i)].\]
	Ist $X =G$ und $G$ operiert durch Konjugation, so folgt
	\[|G| \overset{\text{Bsp.} \ref{beispiel3_6} (2)}{=} |Z(G)| + \sum_{x_i \notin Z(G)} |C_{x_i}| = |Z(G)| + \sum_{x_i \notin Z(G)} [G : C_G(x_i)].\]
	Die sogenannte \textbf{Klassengleichung}.
\end{kor}
\begin{beispiel}\label{beispiel3_9}
	Die Gruppe $G = \GL_n(K)$ operiert treu auf $X = K^n$ durch Matrizenmultiplikation. Bahnen:
	\[G\vec{0}_{K^n} = \{\vec{0}_{K^n}\}, \quad G\vec{e}_1 = K^n \setminus \{\vec{0}_{K^n}\}\]
	Insbesondere gilt $K^n = G\vec{0}_{K^n} \cup G\vec{e}_1$. Zudem ist
	\[\stab_G(\vec{e}_1) = \{\mat{A} \in \GL_n(K) \mid \mat{A}\vec{e}_1 = \vec{e}_1\} = \left\{\begin{bmatrix}
		1 & a_2 & \dots & a_n\\
		0 & & & \\
		\vdots & & \mat{A}' &\\
		0 & & &
	\end{bmatrix} \mid \mat{A}' \in \GL_{n-1}(K)\right\}.\]
	Sei nun $|K| = q < \infty$. Mit der Bahnformel gilt
	\begin{align*}
		|\GL_n(K)| &= |G\vec{e}_1| \cdot |\stab_G(\vec{e}_1)|\\
		&= (q^n - 1) q^{n-1} \cdot |\GL_{n-1}(K)|
	\end{align*}
	Induktiv erhalten wir
	\[|\GL_n(K)| = q^{\frac{n(n-1)}{2}} (q^n - 1)(q^{n-1} -1)\dots(q-1)\]
\end{beispiel}
\begin{center}
	\Large{\textit{Wie helfen uns Gruppenoperationen, die Struktur endlicher Gruppen zu verstehen?}}
\end{center}
\begin{definition}\label{definiton3_10}
	\begin{enumerate}[label=(\alph*)]
		\item Eine Gruppe der $G$ der Ordnung $|G| = p^n$ für eine Primzahl $p$ und $n \in \N$ heißt \textbf{$p$-Gruppe}.
		\item Sei $|G| = p^n \cdot q$ mit $p$ Primzahl und $\ggt(p,q) = 1$. Dann heißt $H \leq G$ \textbf{$p$-Sylowuntergruppe} von $G$, falls $|H| = p^n$. Schreibe $\syl_p(G)$ für die Menge aller $p$-Sylowuntergruppen von $G$.
	\end{enumerate}
\end{definition}
\begin{rem}\label{rem3_11}
\begin{enumerate}[label=(\roman*)]
	\item 	Ist $G$ eine $p$-Gruppe und $X$ eine endliche $G$-Menge, so gilt $|X| \equiv |X^G| \mod p$.
	\begin{proof}
		Nach Korollar \ref{kor3_8} gilt für ein Repräsentantensystem $\{x_i\}_{i \in I}$ der Bahnen von $X$:
		\[|X| \equiv |X^G| + \sum_{x_i \notin X^G} [G : \stab_G(x_i)].\]
			Nach Satz \ref{satz1_11} teilt $[G: \stab_G(x_i) ]$ die Ordnung von $G$. Für $x_i \notin G$ ist $[G : \stab_G(x_i)] > 1$ und somit gilt
			\[p \big| [G : \stab_G(x_i)].\]
	\end{proof}
	\item Für eine $p$-Gruppe $G$ ist das Zentrum $Z(G) \neq \{1_G\}$.
	\begin{proof}
		Die Klassengleichung aus Korollar \ref{kor3_8} liefert 
		\[0 \equiv |G| \equiv |Z(G)| \mod p.\]
		Also teilt $p$ die Ordnung von $Z(G)$.
	\end{proof}	
\end{enumerate}
\end{rem}
\begin{beispiel}\label{beispiel3_12}
	\begin{enumerate}[label=(\arabic*)]
		\item Die abelsches Gruppen $\Z_p, \Z_p \times \Z_p$ und $\Z_{p^2}$ sind $p$-Gruppen. Die Diedergruppe $D_4$ ist eine 2-Gruppe mit $|Z(D_4)| = 2$.
		\item $G = S_3$ mit $|G| = 2 \cdot 3$ ist keine $p$-Gruppe. Es gilt
		\begin{align*}
			\syl_2(G) &= \{\spn{(12)}, \spn{(13)}, \spn{(23)}\}\\
			\syl_3(G) &= \{A_3\}.
		\end{align*}
		\item Sei $G = \GL_n(\Z_p)$ für $n \in \N$. Nach Beispiel \ref{beispiel3_9} gilt
		\[|G| = p^{\frac{n(n-1)}{2}} \underbrace{(p^n - 1)(p^{n-1} - 1) \dots (p-1)}_{\equiv \pm 1 \mod p}.\]
		Sei 
		\[U_n := \left\{\mat{A} \in \GL_n(\Z_p) \mid \mat{A} = \begin{bmatrix}
			\bar{1} & & * \\
			& \ddots &\\
			0 & & \bar{1}
		\end{bmatrix}\right\} \leq \GL_n(\Z_p).\]
		Da $|U_n| = p^{\frac{n(n-1)}{2}}$, ist $U_n$ $p$-Sylowuntergruppe von $\GL_n(\Z_p)$.
	\end{enumerate}
\end{beispiel}
\begin{thm}[Sylow-Sätze]\label{thm3_13}
	Sei $|G| = p^n \cdot q$ mit $p$ Primzahl und $\ggt(p,q) = 1$.
	\begin{enumerate}[label=(\alph*)]
		\item Zu jedem $k \in \{1, \dots, n\}$ existiert eine Untergruppe $H \leq G$ mit $|H| = p^k$.
		\item Sei $H \leq G$ mit $|H| = p^k$ für $k \in \{1, \dots, n\}$. Sei $S \in \syl_p(G)$. Dann existiert $g \in G$ mit $H \leq gSg^{-1}$.
		\item $|\syl_p(G)|$ teilt $q$ und $|\syl_p(G)| \equiv 1 \mod p$.
	\end{enumerate}
\end{thm}
\begin{kor}\label{kor3_14}
	Sei $G$ eine endliche Gruppe und $p$ eine Primzahl.
	\begin{enumerate}[label=(\alph*)]
		\item $p$ teilt $|G|$ nur dann, wenn ein $g \in G$ existiert mit $\ord(g) = p$. \textbf{Satz von Cauchy}
		\item Sei $S \in \syl_p(G)$. Dann gilt
		\[S \unlhd G \;\Leftrightarrow\; \syl_p(G) = \{S\}.\]
	\end{enumerate}
\end{kor}
\begin{proof}
	\begin{enumerate}[label=(\alph*)]
		\item Nach Theorem \ref{thm3_13} (a) gibt es eine Untergruppe $H \leq G$ mit $|H| = p$. Nach Kapitel 2 gilt $H \cong \Z_p$. Somit existiert ein $g \in H$ mit $\ord(g) = p$.
		\item Nach Theorem \ref{thm3_13} (b) sind alle $p$-Sylowuntergruppen konjugiert zu $S$.
	\end{enumerate}
\end{proof}
\begin{proof}[Beweis der Sylowsätze]
	\begin{enumerate}[label=(\alph*)]
		\item \underline{Induktion über $|G| = p^n \cdot q$: } $G$ operiert auf sich selbst durch Konjugation (siehe Beispiel \ref{beispiel3_6} (2)). Sei $\{x_i\}_{i \in I}$ ein Repräsentantensystem der nicht-zentralen Konjugationsklassen. Die Klassengleichung liefert
		\[|G| = |Z(G)| + \sum_{i \in I} [G : C_G(x_i)].\]
		\underline{Fall 1:}
		
		Angenommen $p$ teilt nicht $|Z(G)|$. Da $p$ aber $|G|$ teil, existiert ein $i \in I$, so dass $p$ nicht $[G : C_G(x_i)]=\frac{|G|}{|C_G(x_i)|}$ teilt. Somit gilt $|C_G(x_i)| = p^n \cdot q'$ mit $\ggt(p, q') = 1$ und $|C_G(x_i)| < |G|$. Nach Induktionsvoraussetzung hat $C_G(x_i)$ eine Untergruppe der Ordnung $p^k$ für alle $k \in \{1, \dots, n\}$. Also gilt dies auch für $G$.
		
		\underline{Fall 2: }
		
		Angenommen $p$ teilt $|Z(G)|$. Schreibe 
		\[Z(G) \cong \Z_{n_1} \times \dots \times \Z_{n_s}\]
		mit $1 < n_1 \leq \dots \leq n_s$ und $n_1 \vert \dots \vert n_s$ (siehe Aufgabe M.4.1). Sei $j \in \{1, \dots, s\}$ mit $p \vert n_j$. In $\Z_{n_j}$ existiert somit ein Element der Ordnung $p$. Sei entsprechend $g \in Z(G)$ mit $\ord(g) = p$. Für $k=1$ folgt die Behauptung. Sei also $k > 1$. Da $g \in Z(G)$, folgt $\spn{g} \unlhd G$ mit
		\[\left|\faktor{G}{\spn{g}}\right| = p^{n-1} \cdot q.\]
		Nach Induktionsvoraussetzung existiert eine Untergruppe $U \leq G/\spn{g}$ mit $|U| = p^{k-1}$. Satz \ref{satz1_23} liefert uns eine Untergruppe $\spn{g} \leq H \leq G$ mit $H/\spn{g} = U$. Also gilt
		\[|H| = |U| \dots |\spn{g}| = p^{k-1} \cdot p = p^k.\]
		\item Sei $H \leq G$ mit $|H| = p^k$ für $k \leq n$. Sei $S \in \syl_p(G)$. Zu zeigen ist, dass ein $g \in G$ existiert mit $H \leq gSg^{-1}$.
		
		Die Gruppe $H$ operiert auf $G/S$ durch Multiplikation (vgl. Beispiel \ref{beispiel3_3} (4)). Es ist $|G/S| = q$. Mit Bemerkung \ref{rem3_11} (i) gilt für die Fixpunktmenge dieser Operation
		\[\left|{\faktor{G}{S}}^H\right| \equiv \left|\faktor{G}{S}\right| = q \mod p.\]
		Da nach Voraussetzung $p \nmid |(G/S)^H|$. Somit existiert ein Fixpunkt $gS \in (G/S)^H$ für $g \in G$, d.h. 
		\[hgS = gS\]
		für alle $h \in H$. Also $H \leq gSg^{-1}$, wie gewünscht.
		
		\item \underline{Zeige zunächst:} $|\syl_p(G)| \mid q$.
		
		$G$ operiert auf $\syl_p(G)$ durch Konjugation. Sei $S \in \syl_p(G)$. Nach Teil (b) entspricht die Bahn von $S$ unter $G$ ganz $\syl_p(G)$, d.h. die Operation ist transitiv. Der Bahnsatz liefert
		\[|\syl_p(G)| \overset{\text{Satz \ref{satz3_7}}}{=} [G:\stab_G(S)] \big| [G:\stab_G(S)] \cdot [\stab_G(S) : S] \overset{\text{Satz \ref{satz1_11}}}{=} [G:S] = q.\]
		\underline{Verbleibt zu zeigen:} $|\syl_p(G)| \equiv 1 \mod p$.

		Sei $S \in \syl_p(G)$. $S$ operiert auf $\syl_p(G)$ durch Konjugation. Inbesondere ist $S$ auch Fixpunkt dieser Operation. Sei $S' \in \syl_p(G)$ ein weiterer Fixpunkt, d.h. 
		\[gS'g^{-1} = S'\]
		für alle $g \in S$. Daraus folgt
		\[S \subseteq \stab_G(S') := \{g \in G\mid gS'g^{-1} = S'\} \quad \textbf{Normalisator von $S'$ in $G$}\]
		\underline{Behauptung:} $S \subseteq S'$ und somit $S = S'$ wegen $|S| = |S'| < \infty$.
		
		Es gilt $S' \unlhd \stab_G(S')$. Somit folgt $SS' = S'S \leq \stab_G(S')$. Nach Satz \ref{satz1_21} (a) erhalten wir 
		\[\faktor{SS'}{S'} = \faktor{S}{S \cap S'}.\]
		Da $S$ $p$-Gruppe, ist $(SS')/S$ trivial oder auch eine $p$-Gruppe. Da $S' \leq SS' \leq G$, erhalten wir
		\[[SS' : S'] \big| [G : SS'] \cdot [SS' : S'] \overset{\text{Satz \ref{satz1_11}}}{=} [G:S'] = q.\]
		Da $\ggt(p,q) = 1$, folgt $p \nmid [SS' : S']$ und somit muss $|(SS')/S'| = 1$ bzw. $SS' = S'$. Also gilt $S \subseteq S'$ und somit $S = S'$ wie gewünscht.
		
		Damit ist $S$ der einzige Fixpunkt der Operation von $S$ auf $\syl_p(G)$ durch Konjugation. Bemerkung \ref{rem3_11} (i) liefert nun $|\syl_p(G)| \equiv 1 \mod p$.
	\end{enumerate}
\end{proof}

Als Anwendung wollen wir die Struktur von Gruppen kleiner Ordnung besser verstehen und alle Gruppen bis Ordnung 15 klassifizieren!

\begin{kor}\label{kor3_15}
	\begin{enumerate}[label=(\alph*)]
		\item Sei $|G| = 2p$ mit $p \neq 2$ Primzahl. Dann gilt $G \cong \Z_{2p}$ oder $G \cong D_p$ (Diedergruppe).
		\item Sei $|G| = pq$ mit $p<q$ Primzahlen, so dass $p \nmid q-1$. Dann gilt $G \cong \Z_{pq} \cong \Z_p \times \Z_q$. 
	\end{enumerate}
\end{kor}
\begin{proof}
	\begin{enumerate}[label=(\alph*)]
		\item Nach Theorem \ref{thm3_13} (c) gilt $|\syl_p(G)| \mid 2$ und $|\syl_p(G)| \equiv 1 \mod p$, also $\syl_p(G) = \{S\}$ mit $S \cong \Z_p$. Sei $S = \spn{g}$ für ein $g \in G$ und $h \in G \setminus S$ mit $\ord(h) = 2$ (ein solches $h$ existiert zum Beispiel nach Korollar \ref{kor3_14} (a)). Es folgt, dass
		\[G = \{1_G, g, g^2, \dots, g^{p-1}, h, hg, hg^2, \dots, hg^{p-1}\}.\]
		Da $hg \notin S$, gilt $\ord(hg) = 2p$ oder $\ord(hg) = 2$. Im ersten Fall erhalten wir $G \cong \Z_{2p}$, im zweiten $G \cong D_p$.
		\item Nach Theorem \ref{thm3_13} (c) gilt:
		\begin{align*}
			|\syl_p(G)| \mid q \quad\text{und}\quad |\syl_p(G)| \equiv 1 \mod p,\\
			|\syl_q(G)| \mid p \quad\text{und}\quad |\syl_q(G)| \equiv 1 \mod q.
		\end{align*}
		Insbesondere ist $|\syl_p(G)| \in \{1,q\}$. Aber $q \equiv 1 \mod p$ bedeutet $p \mid q-1$. Ein Widerspruch!
		Daraus folgt $\syl_p(G) = \{S\}$ mit $S \cong \Z_p$. Ebenso ist $|\syl_q(G)| \in \{1,p\}$. Da $p<q$, ist $p \equiv 1 \mod q$ aber nicht möglich, d.h. $\syl_q(G) = \{H\}$ mit $H \cong \Z_q$.
		
		Nach Korollar \ref{kor3_14} (b) gilt $S,H \unlhd G$. Zudem ist $S \cdot H = G$ und $S \cap H = \{1_G\}$. $G$ ist also inneres direktes Produkt von $S$ und $H$. Damit folgt die Behauptung (vgl. Bemerkung \ref{rem2_9} (ii) und Aufgabe M.3.3).
	\end{enumerate}
\end{proof}
\begin{beispiel}\label{beispiel3_16}
	\mbox{}
	\begin{center}
		\begin{tabular}{|C{2cm}|L{8cm}|}
			\hline
			$|G|$ & Mögliche Isomorphietypen \\
			\hline
			2 & $\Z_2$ \\
			3 & $\Z_3$ \\
			4 & $\Z_4, \Z_2 \times \Z_2$ (siehe Aufgabe M.1.1) \\
			5 & $\Z_5$ \\
			6 & $\Z_6, S_3 = D_3$ (siehe Korollar \ref{kor3_15} (a)) \\
			7 & $\Z_7$ \\
			8 & $\Z_8, \Z_4 \times \Z_2, \Z_2 \times \Z_2 \times \Z_2, D_4, Q_8$\\
		\end{tabular}
	\end{center}
	$Q_8$ heißt die \textbf{Quaternionengruppe}. Sie lässt sich zum Beispiel schreiben als Untergruppe von $\SL_2(\C)$ erzeugt von den Matrizen
	\[\mat{A} = \begin{bmatrix}
		i & 0\\
		0 & -i
	\end{bmatrix}
	\quad\text{und}\quad\mat{B}=\begin{bmatrix}
		0 & -1\\
		1 & 0
	\end{bmatrix}\]
	Es gilt $Q_8 = \{\pm\mat{E}_2, \pm \mat{A}, \pm\mat{B}, \pm\mat{AB}\}$ und $\mat{A}^2 = \mat{B}^2 = (\mat{A}\mat{B})^2 = -\mat{E}_2$.
	\begin{center}
		\begin{tabular}{|C{2cm}|L{8cm}|}
			9 & $\Z_9, \Z_3 \times \Z_3$ (siehe Aufgabe M.6.1 (b))\\
			10 & $\Z_{10}, D_5$ (siehe Korollar \ref{kor3_15} (a))\\
			11 & $\Z_{11}$\\
			12 & $\Z_{12}, \Z_2 \times \Z_6, D_6, A_4, H$
		\end{tabular}
	\end{center}
	Die Gruppe $H$ können wir zum Beispiel als Untergruppe von $S_3 \times \Z_4$ realisieren. Sei dazu $H = \{(\sigma, x) \mid \sgn(\sigma) = 1 \;\Leftrightarrow\; x \text{ gerade}\} \unlhd S_3 \times \Z_4$, also
	\begin{align*}
		H = \{&(\id, \bar{0}),
		(\id, \bar{2}),
		((123), \bar{0}),
		((132), \bar{0}),
		((123), \bar{2}),
		((132), \bar{2}),\\
		&((12), \bar{1}),
		((12), \bar{3}),
		((13), \bar{1}),
		((13), \bar{3}),
		((23), \bar{1}),
		((23), \bar{3})\}
		\end{align*}
		$H$ wird zum Beispiel erzeugt von $a = ((123), \bar{2})$ und $b = ((12), \bar{1})$. mit $\ord(a) = 6, a^3 = b^2, ba = a^{-1} b$.
		
		\begin{center}
			\begin{tabular}{|C{2cm}|L{8cm}|}
				13 & $\Z_{13}$\\
				14 & $\Z_{14}, D_7$ (siehe Korollar \ref{kor3_15} (a))\\
				15 & $\Z_{15}$ (siehe Korollar \ref{kor3_15} (b))\\
				\hline
			\end{tabular}
		\end{center}
\end{beispiel}







	
	\section{Normierte Räume}
\subsection{Elementare Eigenschaften}
Sei $X$ ein Vektorraum über $\K \in \{\R, \C\}$. Eine \textbf{Norm} ist eine Abbildung $\rho \colon X \to \R$ mit folgenden Eigenschaften: Für alle $x,y \in X, \lambda \in \K$ gilt
\begin{enumerate}[label={\bfseries(N\arabic*)}]
	\item $\rho(x) \geq 0$ und $\rho(x) = 0 \;\Leftrightarrow\; x = 0$.
	\item $\rho(\lambda x) = |\lambda| \rho(x)$.
	\item\label{norm3} $\rho(x+y) \leq \rho(x) + \rho(y)$.
\end{enumerate}
Das Paar $(X, \rho)$ heißt \textbf{normierter Raum}. Wir schreiben im Folgenden meist $|x|$ oder $\norm{x}$ statt $\rho(x)$ und $X$ statt $(X,\rho)$. Aus \ref{norm3} folgt außerdem für alle $x,y \in X$
\[\abs{\norm{x} - \norm{y}} \leq \norm{x-y}.\]
Durch eine Norm wird ein Abstand $\norm{x-y}$ zwischen zwei Vektoren $x,y \in X$ und somit auch \textbf{Konvergenz} definiert: Die Folge $(x_n)$ konvergiert gegen $x \in X$, wenn $\lim_{n \to \infty} \norm{x_n-x} = 0$.

Jede konvergente Folge ist eine \textbf{Cauchy-Folge}, d.h. $\norm{x_m -x_n} < \varepsilon$, falls $n, m \geq N_\varepsilon$. Wenn jede Cauchy-Folge konvergiert, dann heißt der normierte Raum $X$ vollständig oder \textbf{Banachraum}.
\begin{beispiel}
	\begin{enumerate}[label=(\arabic*)]
		\item $\R^n$ mit $|x| = \left(\sum |x_n|^2\right)^{1/2}$ ist ein Banachraum (Thm. 11.6, Analysis 2).
		\item $\C^n$ mit $|z| = \left(\sum |z_n|^2\right)^{1/2}$ ist ein Banachraum (da norm-isomoprh zu $\R^{2n}$).
		\item $M(m \times n, \C$ mit $\norm{A} = \left(|A_{ik}|^2\right)^{1/2}$ ist ein Banachraum.
		\item $\CF([a,b])$ mit $\norm{f} = \snorm{f}$ ist ein normierter Raum. Konvergenz einer Folge $(f_n)$ in $\CF([a,b])$ entspricht gleichmäßiger Konvergenz. 
		\item\label{bsp5} Ist $X$ ein endlich-dimensionaler Vektorraum mit Basis $\{x_1, \dots, x_n\}$, dann wird durch
		\[\norm{\sum \alpha_k x_k} := \left(\sum |a_k|^2\right)^{1/2}\]
		eine Norm definiert.
	\end{enumerate}
\end{beispiel}
Aus Satz 10.1 und Theorem 10.2 folgt
\begin{thm}\label{thm4_1}
	Der Vektorraum $\CF([a,b])$ versehen mit der Supremumsnorm ist vollständig.
\end{thm}
Eine Reihe $\sum_{n=0}^\infty x_n$ mit Gliedern $x_n \in X$ heißt konvergent, wenn $\lim_{N \to \infty} \sum_{n=0}^N x_n$ existiert. Sie heißt \textbf{absolut konvergent}, wenn $\sum \norm{x_n} < \infty$.
\begin{thm}\label{thm4_2}
	In einem Banachraum ist jede absolut konvergente Reihe konvergent.
\end{thm}
Eine Teilmenge $D \subset X$ eines normiertes Raums heißt \textbf{offen}, wenn zu jedem $x_0 \in D$ ein $\varepsilon > 0$ existiert, so dass
\[\B_\varepsilon(x_0) := \{x \in X \mid \norm{x-x_0} < \varepsilon\} \subset D.\]
$D$ heißt \textbf{abgeschlossen}, wenn $X \setminus D$ offen ist und $D$ heißt \textbf{kompakt}, wenn jede Folge aus $D$ eine in $D$ konvergente Teilfolge hat.
\begin{satz}\label{satz4_3}
	Eine Teilmenge $M$ eines normierten Raumes $X$ ist genau dann abgeschlossen, wenn für jede konvergente Folge $(x_n)$ aus $M$ gilt
\[\lim_{n \to \infty} x_n = 0 \quad\Rightarrow\quad x \in M.\]
\end{satz}
\begin{thm}\label{thm4_4}
	Sei $X$ ein Banachraum, sei $M \subset X$ abgeschlossen und $f \colon M \to M$ eine Abbildung mit
	\[\norm{f(x) - f(y)} \leq L \norm{x-y}\]
	für $x,y \in M$ und $L<1$. Dann existiert genau ein $x^* \in M$ mit $f(x^* = x^*$. Außerdem konvergiert die rekursiv definierte Folge $x_{n+1} = f(x_n)$ für jeden Startpunkt $x_0 \in M$ gegen $x^*$.
\end{thm}
Zwei Normen $\norm{\cdot}_1, \norm{\cdot}_2$ heißen \textbf{äquivalent}, wenn Konstanten $c,d \in \R$ existieren, so dass für alle $x \in X$
\begin{align*}
	\norm{x}_1 &\leq c \norm{x}_2\\
	\norm{x}_2 &\leq d \norm{x}_1.
\end{align*}
Aussagen über Konvergenz, Vollständigkeit, Offenheit etc. hängen nicht von der Wahl äquivalenter Normen ab.
\begin{thm}\label{thm4_5}
	In einem endlich-dimensionalen Vektorraum sind alle Normen äquivalent.
\end{thm}
Wegen Theorem \ref{thm4_5} und Beispiel \ref{bsp5} übertragen sich die Eigenschaften der normierten Räume $\R^N, \C^N$ auf alle endlich-dimensionalen normierten Räume. Insbesondere gilt
\begin{satz}\label{satz4_6}
	Jeder endlich-dimensionale normierte Raum ist vollständig.
\end{satz}
\subsection{Beschränkte lineare Abbildungen}
Im Folgenden ist $X$ immer ein normierter Raum. Wir betrachten lineare Abbildungen $A \colon X \to X$ und schreiben $Ax := A(x)$ und $AB := A \circ B$. $A$ heißt \textbf{beschränkt}, falls ein $c \in \R$ existiert mit
\begin{equation}
	\norm{Ax} \leq c \norm{x}\label{4_2beschraenkt}
\end{equation}
für alle $x \in X$, sonst heißt $A$ unbeschränkt.
\begin{satz}\label{satz4_7}
	Folgende Aussagen sind äquivalent:
	\begin{enumerate}[label=(\alph*)]
		\item $A$ ist beschränkt.
		\item $A$ ist stetig.
		\item $A$ ist stetig in $0\in X$.
	\end{enumerate}
\end{satz}
Sei $\mathscr{L}(X)$ die Menge der beschränkten linearen Abbildungen $A \colon X \to X$. Für $A \in \mathscr{L}$ ist
\begin{equation}\label{4_2norm}
	\norm{A} := \sup_{\norm{x} = 1} \norm{Ax}
\end{equation}
endlich und $c := \norm{A}$ ist die kleinste Zahl, für welche \eqref{4_2beschraenkt} gilt. Der Vektorraum $\mathscr{L}(X)$ und \eqref{4_2norm} definieren eine Norm in $\mathscr{L}(X)$, die \textbf{Operatornorm}.
Außerdem gilt
\begin{equation}\label{4_2ungleichung}
	\norm{AB} \leq \norm{A}\cdot\norm{B}
\end{equation}
für $AB \in \mathscr{L}(X)$.
\begin{rem}
	\begin{enumerate}[label=(\arabic*)]
		\item Ist $A \colon \R^n \to \R^n$ gegeben durch die Matrix $(A_{ik})$ und $\R^n$ durch $|x| = \left(\sum |x_i|^2\right)^{1/2}$ normiert, dann gilt
		\[\norm{A} \leq \left(\sum |A_{ik}|^2\right)^{1/2} =: \norm{A}_2,\]
		denn $|Ax| \leq \norm{A}_2 |x|$ (Analysis 2, Kapitel 12.4).
		\item Ist $X$ endlich-dimensional, dann ist jede lineare Abbildung $A \colon X \to X$ beschränkt.
		\item In $\CF_0^\infty(\R)$ mit $\norm{f} = \snorm{f}$ ist die lineare Abbildung $f \mapsto f'$ unbeschränkt.
	\end{enumerate}
\end{rem}
\begin{satz}\label{satz4_8}
	Sind $A_n, B_n \in \mathscr{L}(X)$ und $A_n \to A, B_n \to B$ für $n \to \infty$, dann gilt $A_n B_n \to AB$ für $n \to \infty$.
\end{satz}
\begin{thm}\label{thm4_9}
	Ist $X$ vollständig, dann ist auch $\mathscr{L}(x)$ vollständig.
\end{thm}
\subsection{Die Exponentialabbildung}
In diesem Kapitel ist $X$ ein Banachraum.
\begin{satz}\label{satz4_10}
	Seien $A_k, B_\ell \in \mathscr{L}(X), k, \ell \in \N_0$. Sind $\sum A_k, \sum B_l$ absolut konvergent und $c_n = \sum_{k=0}^n A_k B_{n-k}$, dann ist auch $\sum c_n$ absolut konvergent und
	\[\left(\sum_{k\geq0} A_k\right)\left(\sum_{\ell \geq 0} B_\ell\right) = \sum_{n \geq 0} c_n.\]
\end{satz}
Für $A \in \mathscr{L}(X)$ definiert man $A^0 = 1$, wobei $1x = x$, $A^{n+1} = AA^n$ und
\[\exp(A) := \sum_{n = 0}^\infty \frac{1}{n!} A^n.\]
Wegen $\norm{A^n} \leq \norm{A}^n$ ist diese Reihe absolut konvergent und $\norm{\exp(A)} \leq e^{\norm{A}}$.
\begin{satz}\label{satz4_11}
	Seien $A, B, C \in \mathscr{L}(X)$, wobei $C$ bijektiv und $C^{-1} \in \mathscr{L}(X)$. Dann gilt
	\begin{enumerate}[label=(\alph*)]
		\item $\exp(C^{-1}AC) = C^{-1}\exp(A) C$.
		\item Falls $AB = BA$, dann gilt
		\[\exp(A + B) = \exp(A) + \exp(B).\]
		\item $\exp(-A) = \exp(A)^{-1}$.
	\end{enumerate}
\end{satz}
\begin{satz}\label{satz4_12}
	Sei $A \in \mathscr{L}(X)$. Für alle $t \in \R$ gilt dann
	\[\frac{\d}{\d t} e^{At} = A e^{At}.\]
\end{satz}
\begin{beispiel}
	\begin{enumerate}[label=(\arabic*)]
		\item Ist $A = \mathrm{diag}(\lambda_1, \dots, \lambda_n)$, dann ist
		\[A^k = \mathrm{diag}(\lambda_1^k, \dots, \lambda_n^k)\]
		und 
		\[e^{At} = \mathrm{diag}(e^{\lambda_1 t}, \dots, e^{\lambda_n t}).\]
		\item Ist $A$ nilpotent, also $A^{m+1} = 0$ für ein $m \in \N$, dann gilt
		\[e^A = \sum_{i=0}^m \frac{1}{k!}A^k.\]
		\item Für 
		\[A = \begin{bmatrix}
			0 & - \omega\\
			\omega & 0 
		\end{bmatrix}\]
		ist
		\[e^{At} = \begin{bmatrix}
			\cos \omega t & -\sin \omega t\\
			\sin \omega t & \cos \omega t
		\end{bmatrix}.\]
		Allgemeiner für $A \in M(n, \R)$ und $A = -A^\top$ ist $e^{At} \in \mathrm{SO}(n)$.
	\end{enumerate}
\end{beispiel}
\begin{satz}\label{satz4_13}
	Für jede $n \times n$-Matrix $A$ über $\K$ gilt
	\[\det(e^A) = e^{\mathrm{tr}(A)}.\]
\end{satz}
\begin{rem}
	Für $A = \mathrm{diag}(\lambda_1, \dots, \lambda_n)$ gilt
	\[\det(e^A) = e^{\sum_{i=1}^n \lambda_i} = e^{\mathrm{tr}(A)}.\]
\end{rem}
	
	\section{Einheiten, Nullteiler und euklidische Ringe}
Im Folgenden sei $R \neq \{0_R\}$ ein Ring.
\begin{definition}
	Elemente der Menge $R^\times := \{a \in R \mid \exists b \in R : ab = 1_R = ba\}$ heißen \textbf{Einheiten von $R$} oder \textbf{invertierbar}. Ein Ring mit $R^\times = R \setminus \{0_R\}$ heißt \textbf{Schiefkörper}. Ein kommutativer Schiefkörper heißt \textbf{Körper}.
\end{definition}
\begin{rem}\label{rem5_2}
	\begin{enumerate}[label=(\roman*)]
		\item $(R^\times)$ bildet eine Gruppe, die \textbf{Einheitengruppe in $R$}.
		\item Sei $R$ kommutativ. Es gilt $R$ ist genau dann ein Körper, wenn $R$ nur die Ideale $\{0_R\}$ und $R$ hat.
		\begin{proof}
			\glqq{}$\Leftarrow$\grqq: Sei $\{0_R\} \neq I \unlhd R$ und $x \in I \setminus \{0_R\}$. Nach Voraussetzung ist $x$ invertierbar, so dass $xx^{-1} = 1_R \in I$. Also $I=R$.
			
			\glqq{}$\Rightarrow$\grqq: Sei $a \in R \setminus \{0_R\}$ und $I = (a) \unlhd R$. Da $I \neq \{0_R\}$, gilt $I=R$ und somit existiert $b \in R$ mit $ab = 1_R$. Also $a \in R^\times$.
		\end{proof}
	\end{enumerate}
\end{rem}
\begin{beispiel}\label{beispiel5_3}
	\begin{enumerate}[label=(\arabic*)]
		\item Es ist $\Z^\times = \{1,-1\}$ und $\Q^\times = \Q \setminus \{0\}$.
		\item Es gilt $\Z[i]^\times = \{1,-1,i,-i\}$ (als Gruppe isomorph zu $\Z_4$).
		\begin{proof}
			Sei $w \in \Z[i]^\times$ und $z \in \Z[i]$ mit $wz = 1$. Komplexe Konjugation liefert $1 = 1\cdot 1 = wz \overline{wz} = |w|^2 \cdot |z|^2$, d.h. $1 = |w|^2 = |a + ib|^2 = a^2 + b^2$ für $a,b \in \Z$. Also entweder $a = \pm 1$ und $b = 0$ oder $a = 0$ und $b = \pm 1$.
		\end{proof}
		\item Es gilt $\Z_n^\times = \{\bar{a} \in \Z_n \mid \ggt(a,n) =1\}$ für $n > 1$. Insbesondere ist $\Z_n$ ein Körper genau dann, wenn $n$ Primzahl.
		\begin{proof}
			Sei $\bar{a} \in \Z_n^\times$. Dann existiert $\bar{b} \in \Z_n$ mit $\bar{a}\cdot\bar{b} = \overline{ab} = \bar{1}$, d.h. $ab \equiv 1 \mod n$. Also existiert $c \in \Z$ mit $ab + cn = 1$ und $\ggt(a,n) = 1$. Sei $\ggt(a,n) = 1$. Dann existieren $b,c \in \Z$ mit $ab + cn = 1$. Daraus folgt
			\[\bar{a}\cdot\bar{b} = \overline{ab} = \overline{1 -cn} = \bar{1} - \overline{cn} = \bar{1}.\]
			Also $\bar{a} \in \Z_n^\times$.
		\end{proof}
		Die Zuordnung $n \mapsto |\Z_n^\times|$ definiert eine Abbildung $\varphi \colon \N\to\N$, die \textbf{Eulersche $\varphi$-Funktion} genannt wird. Es gilt $\varphi(1) := 1$ sowie
		\begin{align*}
			&\varphi(2) = |\Z_2^\times| = 1& &\varphi(4) = |\Z_4^\times| = 2& &\varphi(6) = |\Z_6^\times| = 2&\\
			&\varphi(3) = |\Z_3^\times| = 2& &\varphi(5) = |\Z_5^\times| = 4& &\varphi(7) = |\Z_7^\times| = 6&
		\end{align*}
		$\varphi$ ist multiplikativ, d.h. für $n = n_1 n_2$ mit $\ggt(n_1, n_2) = 1$ gilt:
		\[\varphi(n) = \varphi(n_1) \cdot \varphi(n_2).\]
		\begin{proof}
			Nach Korollar \ref{kor4_16} gilt $\Z_n \cong \Z_{n_1} \times \Z_{n_2}$ und somit
			\[\Z_n^\times  \cong (\Z_{n_1} \times \Z_{n_2})^\times = \Z_{n_1}^\times \times \Z_{n_2}^\times.\]
		\end{proof}
	\end{enumerate}
\end{beispiel}
\begin{leftbar}
		Wir wollen im Folgenden wesentliche Eigenschaften von $\Z$ abstrahieren:
	\begin{itemize}
		\item Für alle $a,b,c \in \Z$ gilt: $a \cdot b = 0 \;\Rightarrow\; a=0 \lor b=0$.
		\item Existenz einer Division mit Rest.
		\item Existenz einer eindeutigen Primfaktorzerlegung.
	\end{itemize}
\end{leftbar}
\begin{definition}
	Ein Element $a \in R \setminus \{0_R\}$ heißt \textbf{Nullteiler}, falls ein Element $b \in R \setminus \{0_R\}$ existiert mit $ab = 0_R$ oder $ba = 0_R$. Ein kommutativer Ring ohne Nullteiler heißt \textbf{Integritätsbereich}.
\end{definition}
\begin{beispiel}\label{beispiel5_5}
	\begin{enumerate}[label=(\arabic*)]
		\item Jeder Körper $K$ ist Integritätsbereich, da für $a,b \in K$ mit $b \neq 0_K$ gilt: $ab = 0_K \;\Rightarrow\; abb^{-1} = 0_K b^{-1} = 0_K$. Allgemein sind Einheiten niemals Nullteiler.
		
		Ist $R$ Integritätsbereich und $S \leq R$, so ist auch $S$ Integritätsbereich. Insbesondere ist zum Beispiel $\Z[\sqrt{n}] \leq \C$ ein Integritätsbereich für $n \in \Z$. Endliche Integritätsbereiche sind Körper. Insbesondere ist $\Z_n$ ein Integritätsbereich genau dann, wenn $n$ Primzahl (siehe Beispiel \ref{beispiel5_3} (3)).
		\begin{proof}
			Betrachte $a \in R \setminus \{0_R\}$ und die Abbildung $\varphi_a \colon R \to R$ mit $r \mapsto ar$. $\varphi_a$ ist injektiv, da aus $ar_1 = ar_2$ folgt $0_R = ar_1 - ar_2 = a(r_1 - r_2)$ und somit $r_1 - r_2 = 0_R$ bzw. $r_1 = r_2$. Da $R$ endlich, ist $\varphi_a$ sogar bijektiv, d.h. es existiert $r \in R$ mit $\varphi_a(r) = ar = 1_R$. Also ist $a \in R^\times$ und $R$ ein Körper.
		\end{proof}
		\item Sind $R$ und $S$ nicht-triviale Ringe, so hat $R \times S$ stets Nullteiler, da $(r,0_S) \cdot (0_R, S) = 0_{R \times S}$ für $r \neq 0_R$ und $s \neq 0_S$.
		\item Die Standardmatrizen $\mat{E}_{ij}$ sind Nullteiler in $M_n(K)$ für $n \geq 2$ und einem Körper $K$.
	\end{enumerate}
\end{beispiel}
Analog zur Einbettung $\Z \to \Q$ können wir jeden Integritätsbereich $R$ in einen Körper einbetten. Betrachte dazu die Äquivalenzrelation auf $R \times R \setminus\{0_R\}$ gegeben durch
\[(r,s) \sim (x,y) \;:\Leftrightarrow\; sx = ry.\]
Reflexivität und Symmetrie gelten, da $R$ kommutativ ist. Für Transitivität betrachte $(a,b) \sim (r,s)$ und $(r,s) \sim (x,y)$, d.h. $br=as$ und $sx = ry$. Dann gilt $say = asy = bry = bsx = sbx$. Da $s \neq 0_R$ und $R$ Integritätsbereich, folgt $ay = bx$ und somit $(a,b) \sim (x,y)$, wie gewünscht.

Schreibe $\frac{r}{s} := [(r,s)]$ für die Äquivalenzklasse von $(r,s)$. Dann ist
\[\frac{r}{s} = \frac{x}{y} \;\Leftrightarrow\; sx = ry.\]
$\quot(R) := \{\frac{r}{s} \mid r \in R, s \in R \setminus \{0_R\}\}$ heißt \textbf{Quotientenkörper von $R$}. 
\begin{satz}\label{satz5_6}
	Sei $R$ ein Integritätsbereich. Dann ist $\quot(R)$ ein Körper durch
	\begin{align*}
		\frac{r}{s} + \frac{x}{y} := \frac{ry + sx}{sy}\quad\text{und}\quad \frac{r}{s} \cdot \frac{x}{y} := \frac{rx}{sy}.
	\end{align*}
	Die Abbildung $i \colon R \to \quot(R)$ mit $r \mapsto \frac{r}{1_R}$ ist ein injektiver Ringhomomorphismus. $i$ heißt \textbf{kanonische Einbettung}.
\end{satz}
\begin{proof}
	Die Operationen sind wohldefiniert:
	
	Sei $\frac{r}{s} = \frac{r'}{s'}$ und $\frac{x}{y} = \frac{x'}{y'}$, d.h. $sr' = rs'$ und $yx' = xy'$. Dann gilt
	\begin{align*}
		\frac{ry + sx}{sy} = \frac{rys'y' + sxs'y'}{sys'y'} = \frac{syr'y' + sys'x'}{sys'y'} = \frac{r'y' + s'x'}{s'y'}
	\end{align*}
	sowie
	\begin{align*}
		\frac{rx}{sy} = \frac{rxs'y'}{sys'y'} = \frac{syr'x'}{sys'y'} = \frac{r'x'}{s'y'}.
	\end{align*}
	Die Ringaxiome sind leicht nachzurechnen. Es gilt $0_{\quad(R)} = \frac{0_R}{1_R}$, $1_{\quot(R)} = \frac{1_R}{1_R}$ und für $\frac{r}{s} \in \quot(R)$ mit $r,s \neq 0_R$ ist $-\frac{r}{s} = \frac{-r}{s}$ und $\left(\frac{r}{s}\right)^{-1} = \frac{s}{r}$. Damit wird $\quot(R)$ zum Körper. Die Abbildung $i$ ist offensichtlich ein Ringhomomorphismus und injektiv, da $\frac{r}{1_R} = \frac{s}{1_R}$ genau dann gilt, wenn $r = s$.
\end{proof}
\begin{rem}\label{rem5_7}
	Wir können $R$ als Unterring von $\quot(R)$ betrachten. $\quot(R)$ ist der kleinste Körper (eindeutig bis auf Isomorphie), der $R$ enthält.
\end{rem}


Zurück zu Polynomen und weiter mit
\begin{definition}\label{definition5_8}
	Sei $R$ ein kommutativer Ring und $f = \sum a_i x^i \in R[x]$. Der \textbf{Grad von $f$} ist gegeben durch $\deg(f) := \max\{i \mid a_i \neq 0_R\}$. Setze $\deg(0_{R[x]}) := -\infty$. Ist $\deg(f) = n$, so heißt $a_n$ \textbf{Leitkoeffizient} von $f$. Das Polynom heißt \textbf{normiert}, falls der Leitkoeffizient $1_R$ ist.
\end{definition}

\begin{rem}\label{rem5_9}
	\begin{enumerate}[label=(\roman*)]
		\item Seien $f$ und $g$ Polynome in $R[x]$. Dann gilt
		\begin{align*}
			\deg(f + g) &\leq \max\{\deg(f), \deg(g)\}\\
			\deg(f\cdot g) &\leq \deg(f) + \deg(g).
		\end{align*}
		Ist das Produkt der Leitkoeffizienten von $f$ und $g$ ungleich $0_R$, so gilt $\deg(f\cdot g) = \deg(f) + \deg(g)$ - \textbf{Gradformel} genannt.
		Dies ist stets erfüllt, wenn $R$ Integritätsbereich ist. Andererseits gilt zum Beispiel in $\Z_6 [x]$:
		\[\deg((\bar{2}x^7 + \bar{1})\cdot(\bar{3}x^2)) = \deg(\bar{3}x^2) = 2 < 9 = \deg(\bar{2}x^7 + \bar{1}) + \deg(\bar{3}x^2).\]
		\item $R$ ist Integritätsbereich genau dann, wenn $R[x]$ Integritätsbereich (siehe Gradformel und Beispiel \ref{beispiel5_5} (1)). In diesem Fall gilt $R[x]^\times = R^\times$.
		\begin{proof}
			Sei $f \in R[x]^\times$, d.h. es existiert $g \in R[x]$ mit $f \cdot g = 1_{R[x]}$. Die Gradformel liefert $\deg(f) = \deg(g) = 0$, d.h. $f = a_0 \in R$ und $g = b_0 \in R$ mit $a_0 b_0 = 1_R$.
		\end{proof} 
	\end{enumerate}
\end{rem}

\begin{satz}[Division mit Rest in Polynomringen]\label{satz5_10}
	Seien $f,g \in R[x]$, wobei der Leitkoeffizient $b_m$ von $g$ eine Einheit in $R$ ist. Dann existieren eindeutige $q,r \in R[x]$ mit $\deg(r) < m$ und
	\[f = q \cdot g + r.\]
\end{satz}
\begin{proof}
	\textbf{Existenz: } Induktion nach $n := \deg(f)$.
	
	Ist $n < m$, wähle $q = 0_{R[x]}$ und $r=f$. Sei also $n \geq m$. Für $f = \sum_i a_i x^i$ setze $f_1 := f - a_n b_m^{-1} x^{n-m} g \in R[x]$. Dann ist $\deg(f_1) < \deg(f)$. Nach Induktionsvoraussetzung existieren $q_1, r_1 \in R[x]$ mit $\deg(r_1) < m$ und $f_1 = q_1 \cdot g + r_1$. Es folgt, dass 
	\begin{align*}
	f = f_1 + a_n b_m^{-1} x^{n-m} g &= q_1 g + r_1 + a_n b_m^{-1} x^{n-m} g\\
	&= 	\underbrace{(q_1 + a_n b_m^{-1} x^{n-m})}_{=: q} g + \underbrace{r_1}_{=:r}
	\end{align*}
	\textbf{Eindeutigkeit: } Angenommen, $f = q\cdot g + r = q' g + r'$ mit $\deg(r), \deg(r') < m$. Dann ist $(q- q')g = r' - r$. Es folgt 
	\begin{align*}
		m > \deg(r'-r) = \deg((q - q')\cdot g) = \deg(q-q') + \underbrace{\deg(g)}_{=m}
	\end{align*}
	(Gradformel). Daraus folgt $q-q' = 0_{R[x]}$ und somit $q = q'$ und somit $r=r'$.
\end{proof}
Wir interessieren uns allgemeiner für Ringe, die eine Division mit Rest zulassen.
% Definition 5.11.
\begin{definition}[Euklidische Ringe]
	Ein Integritätsbereich $R$ heißt \textbf{euklidischer Ring} oder kurz \textbf{euklidisch}, wenn es eine Abbildung $\delta \colon R \setminus \{0\} \to \N_0$ gibt, so dass für alle $a,b \in R$ mit $b\neq 0_R$ existieren $q,r \in R$ mit $a = q\cdot b + r$ und $r= 0_R$ oder $\delta(r) < \delta(b)$. Wir nennen $\delta$ \textbf{Gradfunktion}.
\end{definition}
\begin{beispiel}\label{beispiel5_12}
	\begin{enumerate}[label=(\arabic*)]
		\item Sei $K$ ein Körper und seien $a,b \in K$ mit $b \neq 0_K$. Dann ist
		\[a= (\underbrace{ab^{-1}}_{=:q})b + \underbrace{0_K}_{=:r}.\]
		Also ist $K$ euklidisch mit beliebiger Gradfunktion
		\item $\Z$ mit $\delta(n) := |n|$ für $n \in \Z\setminus\{0\}$ ist euklidisch.
		\begin{proof}
			Seien $a,b \in \Z$ mit $b \neq 0$. Sei $r := \min\{m \in \N_0 \mid m = a-nb, n \in \Z\}$. Wähle $q := \frac{a-r}{b} \in \Z$. Es folgt $a = qb + r$ mit $0 \leq r < |b|$.
		\end{proof}
		Die Eindeutigkeit von $q$ und $r$ bei der Division mit Rest ist weder gefordert noch ist sie hier gegeben! Zum Beispiel gilt
		\[7 = 2 \cdot 3 + 1 = 3 \cdot 3 + (-2).\]
		\item Ist $K$ ein Körper, so ist $K[x]$ euklidisch mit $\delta(f) := \deg(f)$ für $f \in K[x] \setminus \{0_{K[x]}\}$ (siehe Satz \ref{satz5_10}).
		\item $\Z[i]$ mit $\delta(a + bi) := a^2 + b^2$ für $a+bi \in \Z[i] \setminus \{0\}$ ist euklidisch. 
		\begin{proof}
			Für alle $z = x+yi \in \C$ existieren $a,b \in \Z$ mit $|x-a| \leq 1/2$ und $|y-b| \leq 1/2$.  Dann gilt $|z-(a+bi)|^2 = |(x-a) + (y-b)i|^2 \leq 2 \cdot 1/4 < 1$. Insbesondere gibt es für $f,g \in \Z[i]$ mit $g \neq 0$ ein $q := a + bi \in \Z[i]$, so dass 
			\begin{align*}
				\left|\frac{f}{g} - q\right|^2 < 1.
			\end{align*}
			Setze $r := f - qg \in \Z[i]$. Falls $r \neq 0$, so gilt $\delta(r) = |f-qg|^2 < |g|^2 = \delta(g)$, wie gewünscht. 
		\end{proof}
		Dies lässt sich veranschaulichen mit $f = 2+i$, $g = -1 -i$. $fg^{-1} = -3/2 + 1/2 i$.
		% TODO Bild Komplexe Zahlenebene
		Wähle z.B. $q_1 = -1 + i$ und $r_1 = f-q_1 g = 2+i -2= i$ oder $q_2 = -2$ und $r_2 = (2+i)-(2+ 2i) = -i$.
	\end{enumerate}
\end{beispiel}
\begin{definition}
	Ein Integritätsbereich $R$ heißt \textbf{Hauptidealring}, wenn jedes Ideal $I \unlhd R$ ein Hauptideal ist, d.h. $I = (r)$ für ein $r \in R$.
\end{definition}
\begin{satz}\label{satz5_14}
	Jeder euklidische Ring ist Hauptidealring.
\end{satz}
\begin{proof}
	Sei $R$ euklidisch mit $\{0_R\} \neq I \unlhd R$. Wähle $b \in I \setminus \{0_R\}$ mit $\delta(b)$ minimal. Es gilt $(b) \subseteq I$. Sei nun $a \in I$. Dann existieren $q,r \in R$ mit $a = qb + r$ und $r = 0_R$ oder $\delta(r) < \delta(b)$. Da $r = a - qb \in I$ und $\delta(b)$ minimal, folgt  $r = 0_R$. Somit ist $a = qb \in (b)$ und $(b) = I$.
\end{proof}
\begin{beispiel}\label{beispiel5_15}
	\begin{enumerate}[label=(\arabic*)]
		\item Ein Körper $K, \Z, K[x]$ und $\Z[i]$ sind Hauptidealringe nach Beispiel \ref{beispiel5_12}.
		\item $\Z[x]$ ist kein Hauptidealring und insbesondere nicht euklidisch. 
		\begin{proof}
			Betrachte $I := (2,x) \unlhd \Z[x]$. Da $1 \notin I$, ist $I \neq \Z[x]$. Angenommen $I = (f)$ für ein Polynom $f \in \Z[x] \setminus \{0\}$. Dann existiert $g \in \Z[x]$ mit $f \cdot g = 2$. Mit der Gradformel folgt $\deg(f) = 0$, also $f = a_0 \in \Z$ mit $a_0 \mid 2$. Da $I \neq \Z[x]$, ist $a_0 \in \{\pm 1\}$ ausgeschlossen. Also $a_0 \in \{\pm 2\}$. Aber dann gilt $x \notin (a_0) = (f) = I$. Ein Widerspruch.
		\end{proof}
		\item $\Z\left[\frac{1}{2} + \frac{1}{2}\sqrt{-19}\right]$ ist nicht euklidisch, aber ein Hauptidealring,
		
		$\Z\left[\frac{1}{2} + \frac{1}{2}\sqrt{-11}\right]$ hingegen ist euklidisch.
		\begin{proof}[Beweisidee]
			Sei $w = \frac{1}{2} + \frac{1}{2}\sqrt{-19} \in \C$. Dann gilt $w \bar{w} = \frac{1}{4} + \frac{19}{4} = \frac{20}{4} = 5$, sowie $w + \bar{w} = 1$. Für $a,b \in \Z$ folgt
			\begin{align*}
				(a+bw) \cdot \overline{(a+bw)} &= (a+bw)\cdot(a+b\bar{w}) = a^2 + ab(w+\bar{w}) + b^2 w\bar{w} = a^2 + ab + 5b^2\\
				&= \frac{1}{2}(a+b)^2 + \frac{1}{2}a^2 + \frac{9}{2} b^2 \geq 0.
			\end{align*}
			Die Abbildung $N \colon \Z[w] \to \N_0$ mit $a + bw \mapsto a^2 + ab + 5b^2$ ist multiplikativ, da komplexe Konjugation multiplikativ ist. 
			
			\textbf{Behauptung: } $\Z[w]^\times = \{\pm 1\}$. 
			
			Sei $x \in \Z[w]^\times$ und $y \in \Z[w]$ mit $xy = 1$. Dann gilt
			\begin{align*}
				1 = N(1) = N(xy) = N(x)N(y).
			\end{align*}
			Also $N(x)= 1$. Schreibe $x = a + bw$. Dann folgt
			\begin{align*}
				\frac{1}{2}(a+b)^2 + \frac{1}{2}a^2+ \frac{9}{2}b^2 = 1
			\end{align*}
			und somit $b = 0$ und $a \in \{\pm 1\}$. Das zeigt die Behauptung. 
			
			Angenommen, $\Z[w]$ ist euklidisch mit Gradfunktion $\delta \colon \Z[w] \setminus \{0\} \to \N_0$ Wähle $x \in \Z[w] \setminus \{\pm 1, 0\}$ mit $\delta(x)$ minimal. Sei $y \in \Z[w]$. Dann existiert $q,r \in \Z[w]$ mit $y = q x + r$ und $r = 0$ oder $\delta(r) < \delta(x)$. Nach Wahl von $x$ muss $r=0$ oder $r \in \{\pm 1\}$. Somit gilt für den Quotientenring $\Z[w] / (x)$: $|\Z[w] / (x)| \in \{2,3\}$. Daraus folgt
			\[\faktor{\Z[w]}{(x)} \cong \Z_2\quad\text{oder}\quad \faktor{\Z[w]}{(x)} \cong \Z_3\]
			(Isomorphie von Ringen!). Wir führen dies zu einem Widerspruch!
			
			Für $w = \frac{1}{2} + \frac{1}{2}\sqrt{-19}$ gilt
			\begin{align*}
				w^2 - w + 5 = \frac{1}{4} + \frac{1}{2}\sqrt{-19} - \frac{19}{4} - \frac{1}{2} - \frac{1}{2}\sqrt{-19} + 5 = 0.
			\end{align*}
			Insbesondere gilt für $\bar{w} = w + (x) \in \Z[w] /(x)$ (Überstrich heißt hier Restklasse) : $\bar{w}^2 - \bar{w} + \bar{5} = \bar{0}$. Aber kein Element in $\Z_2$ oder $\Z_3$ erfüllt diese Gleichung:
			
			\underline{in $\Z_2$:} $\bar{0}^2 - \bar{0} + \bar{5} =  \bar{1}$ und $\bar{1}^2 - \bar{1} + \bar{5} = \bar{1}$.
			
			\underline{in $\Z_3$: } $\bar{0}^2 - \bar{0} + \bar{5} = \bar{2}$, $\bar{1}^2 - \bar{1} + \bar{5} = \bar{2}$ und $\bar{2}^2 - \bar{2} + \bar{5} = \bar{1}$.
			
			Ein Widerspruch. Insbesondere liefert die obige Abbildung $N \colon \Z[w] \to \N_0$ mit $a + bw \mapsto a^2 + ab + 5b^2$ keine gewünschte Gradfunktion. Mit Hilfe dieser Funktion lässt sich aber zeigen, dass $\Z[w]$ Hauptidealring ist. Dazu verallgemeinert man das Vorgehen aus dem Beweis vom Satz \ref{satz5_14}. Für $w' = \frac{1}{2} + \frac{1}{2}\sqrt{-11} \in \C$ liefert die Abbildung $N' : \Z[w'] \to \N_0$ mit $a + bw' \mapsto (a+bw')(a+b\bar{w'})$ aber eine Gradfunktion, die $\Z[w']$ zum euklidischen Ring macht. Dabei gilt
			\begin{align*}
				w' \cdot \bar{w'} = \frac{1}{4} + \frac{11}{4} = \frac{12}{4} = 3.
			\end{align*}
			und 
			\begin{align*}
				(a + bw')(a+b\bar{w'}) = a^2 + ab(w' + \bar{w'}) + b^2 w' \bar{w'} = a^2 + ab + 3b^2 \geq 0.
			\end{align*}
			Nun können wir ähnlich argumentieren wie in Beispiel \ref{beispiel5_12} (4).
		\end{proof}
	\end{enumerate}
\end{beispiel}






	
\end{document}
\section{Gewöhnliche Differentialgleichungen}

\subsection{Einführung}
Wir betrachten einerseits Systeme erster Ordnung, d. h.
\begin{align*}
	x_1 &= f_1(t, x_1, \dots, x_n)\\
	&\vdots\\
	x_n &= f_n(t, x_1, \dots, x_n)
\end{align*}
mit gegebenen Funktionen $f_1, \dots, f_n \colon R \subset \R^{n+1} \to \R$ und gesuchten $\CF^1$-Funktionen $x_1, \dots, x_n \colon I \to \R$ und andererseits Differentialgleichungen höherer Ordnung
\[x^{(n)} = f(t, x^{(1)}, \dots, x^{(n-1)})\]
mit $f \colon D \subset \R^{n+1} \to \R$ gegeben und gesucht ist $\CF^n \ni x \colon I \to \R$. 

Das zweite System lässt sich immer auf das erste reduzieren.
\begin{beispiel}
	Die Newton-Gleichung $\ddot{q} = F(t, q, \dot{q})$ für $q \colon I \to \R$ wird durch Einführen der Funktion $p = \dot{q}$ äquivalent zum System
	\begin{align*}
		\dot{q} &= p\\
		\dot{p} &= F(t,q,p)
	\end{align*}
\end{beispiel}
\subsection{Lineare Systeme}
Ein lineares System 1. Ordnung ist von der Form
\begin{equation}\label{eq:5_2_eq0}
	\dot{x} = A(t)x + b(t), \quad x(t_0) = x_0
\end{equation}
mit $A(t) \in M(n, \K)$ und $b(t) \in \K^n$. Das System heißt \textbf{homogen}, wenn $b(t) \equiv 0$, sonst \textbf{inhomogen}.

Sei $J \subset \R$ ein Intervall und $t_0 \in J$. Sei $t \mapsto A(t) \in M(n,\K)$ stetig auf $J$ und sei $x \in \CF^1(J, \K^n)$ eine Lösung des AWP
\begin{equation}\label{eq:5_2_eq1}
	\dot{x} = A(t)x, \quad x(t_0) = x_0,
\end{equation}
dann gilt
\begin{align*}
	x(t) &= x_0 + \int_{t_0}^t \dot{x}(t_1) \d t_1 = x_0 + \int_{t_0}^t \d t_1 A(t_1)x(t_1).
\end{align*}
Die Gleichung wollen wir iterieren, d. h. wir nutzen 
\[x(t_1) = x_0 + \int_{t_0}^{t_1} \d t_2 A(t_2) x(t_2)\]
und bekommen
\[x(t) = x_0 + \int_{t_0}^{t} \d t_1 A(t_1) x_0 + \int_{t_0}^{t} \d t_1 \int_{t_0}^{t_1} A(t_1) A(t_2) x(t_2).\]
Nach $n$ Iterationen:
\begin{align}\label{eq:5_2_eq2}
	x(t) = &x_0 + \sum_{k=1}^N \int_{t_0}^{t} \d t_1 \int_{t_0}^{t_1} \d t_2 \dots \int_{t_0}^{t_{k-1}} \d t_k A(t_1) \dots A(t_k) x_0\\
	 &+ \int_{t_0}^{t} \d t_1 \dots \int_{t_0}^{t_N} \d t_{N+1} A(t_1)\dots A(t_{N+1}) x(t_{N+1}).\nonumber
\end{align}
\begin{satz}\label{satz5_1}
	Sei $t \mapsto A(t) \in M(n,\K)$ stetig auf dem Intervall $J \subset \R$ und $t_0 \in J$. Dann hat das AWP
	\[\dot{x} = A(t)x, \quad x(t_0) = x_0\]
	genau eine Lösung, die auf ganz $J$ existiert. Sie ist gegeben durch 
	\begin{align*}
		x(t) = x_0 + \sum_{k=1}^{\infty} \int_{t_0}^{t} \d t_1 \int_{t_0}^{t_1} \d t_2 \dots \int_{t_0}^{t_{k-1}} \d t_k A(t_1) \dots A(t_k) x_0.
	\end{align*}
\end{satz}
In der Regel wird angenommen, dass $A(t), b(t)$ stetige Funktionen von $t$ sind. Unter einer Lösung von \eqref{eq:5_2_eq0} versteht man dann eine \underline{differenzierbare} Abbildung $t \mapsto x(t)$, die auf einem Intervall $I \subset \R$ definiert ist und \eqref{eq:5_2_eq0} löst. Dann folgt aus \eqref{eq:5_2_eq0}, dass $t \mapsto x(t)$ stetig differenzierbar ist.


\begin{kor}\label{kor5_2}
	Zusätzlich zu den Annahmen von Satz \ref{satz5_1} sei $A(t)A(s) = A(s)A(t)$ für alle $s,t \in J$. Dann ist die eindeutige Lösung von \eqref{eq:5_2_eq1} gegeben durch
	\[x(t) = \exp\left(\int_{t_0}^{t} A(s) \d s\right)x_0.\]
\end{kor}
\begin{satz}\label{satz5_3}
	Sei $A \colon J \to M(n, \K)$ stetig und $J \subset \R$ ein Intervall. Dann gilt
	\begin{enumerate}[label=(\alph*)]
		\item Die Menge der Lösungen von $\dot{x} = A(t)x$ bilden einen $n$-dimensionalen Vektorraum über $\K$.
		\item Ist $t \mapsto x(t)$ eine Lösung mit $x(t_0) = 0$, dann ist $x \equiv 0$.
		\item Für Lösungen $x_1, \dots, x_k$ von $\dot{x}= A(t)x$ sind folgende Aussagen äquivalent:
		\begin{enumerate}[label=(\roman*)]
			\item Die Funktionen $x_1, \dots, x_k$ sind linear unabhängig.
			\item Es gibt ein $t_0 \in J$, so dass $x_1(t_0), \dots, x_k(t_0)$ linear unabhängig sind.
			\item Für alle $t \in J$ sind $x_1(t), \dots, x_k(t)$ linear unabhängig.
		\end{enumerate}
	\end{enumerate}
\end{satz}
Eine Basis $\{x_1, \dots, x_n\}$ des Lösungsraums von
\[\dot{x}= A(t) x\]
heißt \textbf{Fundamentalsystem} von $\dot{x}= A(t) x$. Um das AWP $\dot{x}= A(t) x, x(t_0) = x_0$ zu lösen entwickeln wir
\begin{equation}\label{eq:5_2_eq4}
	x_0 = \sum_{k=0}^n \alpha_k x_k(x_0), \quad \alpha_k \in \K
\end{equation}
bezüglich der Basis $\{x_1(t), \dots, x_n(t)\}$ von $\K^n$.Dann ist
\begin{equation}\label{eq:5_2_eq5}
	x(t) = \sum_{k=1}^n \alpha_k x_k (t) \quad t \in J
\end{equation}
die eindeutige Lösung des AWP. Die $n \times n$-Matrix $X(t) = (x_1(t) \dots x_n(t))$ gebildet aus den Lösungen $x_1, \dots, x_n$ heißt Lösungsmatrix bzw. \textbf{Fundamentalmatrix} von $\dot{x} = A(t) x$, letzteres, wenn $x_1, \dots, x_n$ ein Fundamentalsystem ist. 

Mit der Fundamentalmatrix $X(t) = (x_1(t) \dots x_n(t))$ nehmen \eqref{eq:5_2_eq4} und \eqref{eq:5_2_eq5} folgende Form an:
\[x_0 = X(t_0) \alpha, \quad \alpha = \begin{bmatrix}
	\alpha_1\\\vdots\\\alpha_n
\end{bmatrix}\]
und $x(t) = X(t) \alpha = X(t) X(t_0)^{-1} x_0$.
\begin{itemize}
	\item Jede Lösungsmatrix $Y(t)$ erfüllt $\dot{Y} = A(t) Y$ und $Y$ ist durch $Y(t_0)$ für ein $t_0 \in J$ eindeutig bestimmt.
	\item Ist $C \in \mathrm{GL}(n,\K)$ und $X(t)$ eine Fundamentalmatrix, dann ist auch $X(t)C$ eine Fundamentalmatrix.
	\item Insbesondere ist $Y(t) = X(t) X(t_0)^{-1}$ eine Fundamentalmatrix, nämlich diejenige Fundamentalmatrix, die durch
	\[\dot{Y} = A(t) Y \quad Y(t_0) = E_n\]
	eindeutig bestimmt ist. 
\end{itemize}
Die Determinante
\[W(T) := \det(x_1(t) \dots x_n(t))\]
gebildet aus $n$ Lösungen von $\dot{x} = A(t) x$ heißt \textbf{Wronski-Determinante} des Lösungssystems $x_1, \dots, x_n$.
\begin{satz}\label{satz5_4}
	Sei $A \colon J \to M(n,\K)$ stetig und sei $W(t)$ die Wronski-Determinante eines Lösungssystems von $\dot{x} = A(t) x$. Dann gilt für $t_0 \in J$:
	\[W(t) = W(t_0) \exp\left(\int_{t_0}^{t} \tr(A(s)) \d s\right).\]
\end{satz}
\begin{rem}
	Für $A(t) \equiv A$ folgt das aus Satz \ref{satz4_13}.
\end{rem}